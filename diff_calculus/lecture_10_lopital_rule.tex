\section{ Раскрытие неопределенностей (правило Лопиталя)}

\subsection{Виды неопределенностей}

Раскрытие неопределённостей - методы вычисления пределов функций, заданных формулами, которые в результате формальной подстановки в них предельных значений аргумента теряют смысл, то есть переходят в выражения типа:

\begin{itemize}
\item 
$ \infty-\infty $
\item 
$ \frac{\infty}{\infty} $
\item 
$ \frac{0}{0} $
\item 
$ 0^0 $
\item 
$ 1^\infty $
\item 
$ \infty^0 $
\item 
$ 0\cdot\infty $
\end{itemize}


\subsection{Точная формулировка}

Условия:
\begin{itemize}
\item 
$\lim\limits_{x\to a}{f(x)}=\lim\limits_{x\to a}{g(x)}=0$ или $\infty$;
\item 
$~f(x)$ и $~g(x)$ дифференцируемы в проколотой окрестности $~a$;
\item 
$g'(x)\neq 0$ в проколотой окрестности $~a$
\item 
существует $\lim\limits_{x\to a}{\frac{f'(x)}{g'(x)}}$,
\end{itemize}

тогда существует $\lim\limits_{x\to a}{\frac{f(x)}{g(x)}} = \lim\limits_{x\to a}{\frac{f'(x)}{g'(x)}}$

Пределы также могут быть односторонними

\subsection{правило Лопиталя раскрытия неопределенностей 0/0 (1-е правило Лопиталя)}

Докажем теорему для случая, когда пределы функций равны нулю (то есть неопределённость вида $\left(\frac{0}{0}\right)$

Поскольку мы рассматриваем функции $f$ и $g$ только в правой проколотой полуокрестности точки $a$, мы можем непрерывным образом их доопределить в этой точке: пусть $f(a)=g(a)=0$. Возьмём некоторый $x$ из рассматриваемой полуокрестности и применим к отрезку $[a,\;x]$ теорему Коши. По этой теореме получим:

$$
\exists c \in [a,x]\!:\frac{f(x)-f(a)}{g(x)-g(a)}=\frac{f'(c)}{g'(c)}
$$

но $f(a)=g(a)=0$, поэтому $\forall x\, \exists c \in [a,\;x]\!:\frac{f(x)}{g(x)}=\frac{f'(c)}{g'(c)}$.

\subsection{Замечание о существовании пределов отношения функций и производных}

В доказанной теореме из существования отношения пределов \textbf{производных функции} 
следует существование пределов \textbf{функции}.

Обратное строго говоря неверно.


\subsection{2-е правило Лопиталя}

Без доказательства.

\subsection{Раскрытие других неопределенностей}

\subsubsection{Неопределенность $ 0 \cdot \infty $}

Разрешается представлением $ \dfrac{\infty}{1/0} = \dfrac{\infty}{\infty}
$ или $ \dfrac{0}{1/\infty} = \dfrac{0}{0}$

\subsubsection{Неопределенность $ \infty - \infty  $}

Разрешается представлением 
$ f(x)= \infty, g(x) = \infty$

$ f(x)-g(x) = \dfrac{1/g(x) - 1/f(x)}{1/f(x)1/g(x)} = \dfrac{0}{0} $

\subsubsection{Неопределенности $~0^0$, $1^\infty$, $\infty^0$}


Для раскрытия неопределённостей видов $0^0$, $1^\infty$, $\infty^0$ пользуются следующим приёмом: находят предел (натурального) логарифм а выражения, содержащего данную неопределённость. В результате вид неопределённости меняется. После нахождения предела от него берут экспоненту.

$0^0=e^{0\cdot ln{0}}=e^{0\cdot\infty}$

$1^\infty=e^{\infty\cdot ln{1}}=e^{\infty\cdot 0}$

$\infty^0=e^{0\cdot ln{\infty}}=e^{0\cdot\infty}$
