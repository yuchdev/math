\section{Дифференцирование функций, заданных параметрически. Производная вектор-функции. Производные и дифференциалы высших порядков}

\subsection{Производная функции, заданной параметрически}

$\begin{cases}
y = \ln (1+t^{2}) \\
x = \arctg t
\end{cases}$

$ y'_{x} = \dfrac{y'_{t}}{x'_{t}} $

$ x'_{t} = (\arctg t)' = \dfrac{1}{1+t^{2}} $

$ y'_{t} = (\ln (1+t^{2}))' = \dfrac{2t}{1+t^{2}} $

$ y'_{x} = \dfrac{2t}{1+t^{2}} (1+t^{2}) = 2t $

\subsection{Производная вектор-функции}

$ f(t) = ie^{t} + j \ln t  - k\sin t $

$ f'(t) = ie^{t} + \dfrac{j}{t}  - k \cos t $

\subsection{Производные высшего порядка}

Вычислить производную $ n $-го порядка для функции

$ y = \ln x $

Используем метод мат. индукции

$ y' = \dfrac{1}{x} $

$ y'' = -\dfrac{1}{x^{2}} $

$ y''' = \dfrac{2}{x^{3}} $

$ y^{(4)} = -\dfrac{6}{x^{4}} $

$ \ldots $

$ y^{(n)} = (-1)^{n-1}\dfrac{(n-1)!}{x^{n}} $

\subsection{Производная второго порядка функции, заданной параметрически}

$\begin{cases}
y = \sin t \\
x = \cos t
\end{cases}$

Найти вторую производную по $ x $ ($ y''_{xx} $)

Формула для вычисления производной

$ y''_{xx} = \dfrac{(y'_{x})'_{t}}{x'_{t}}$


Шаг 1

$ (y'_{x})'_{t} = ? $

$ y'_{x} = \dfrac{y'_{t}}{x'_{t}} = \dfrac{y''_{tt}x'_{t} - x''_{tt}y'_{t} }{(x'_{t})^{2}} $

$ x'_{t} = (\cos t)' = -\sin t$

$x''_{tt} = -(\sin t)' = -\cos t $

$ y'_{t} = (\sin t)' = \cos t$

$y''_{tt} = (\cos t)' = - \sin t $

$ (y'_{x})'_{t} = \dfrac{(- \sin t)(- \sin t) - (- \cos t)(\cos t) }{\sin^{2} t} $
$ = \dfrac{1}{\sin^{2} t} $

$ y''_{xx} = \dfrac{\frac{1}{\sin^{2} t}}{x'_{t}} = - \dfrac{1}{\sin^{3} t} $

\subsection{Физический смысл производной второго порядка}

Скорость прямолинейно движущегося тела изменяется по закону

$ v = k\sqrt{s} $, где $ s = s(t) $ - закон изменения координат.

Доказать, что сила, приложенная к телу, постоянна

$ F = ma $

$ a = (k\sqrt{s})' = k \dfrac{1}{2\sqrt{s}} s'(t)$, т.к. $ s $ - сложная функция

$ a = k \dfrac{1}{2\sqrt{s}} k\sqrt{s} = \dfrac{k^{2}}{2} - const$ 

$ F = ma = \dfrac{mk^{2}}{2} $

Вообще, если ускорение постоянно, то и приложенная к телу сила постоянна.


\subsection{Дифференциал высшего порядка}

$ f(x) = x^{2} \ln x $

Найти $ d^{3}f $

По формуле $ d^{3}f = f^{(n)}(x) dx^{3} $

Вычислим третью производную

$ f'(x) = (x^{2} \ln x)' = 2x \ln x + \dfrac{x^{2}}{x} = 2x \ln x + x = x(1 + 2 \ln x) $

$ f''(x) = (x(1 + 2 \ln x))' = (1 + 2 \ln x) + x \dfrac{2}{x} = 2 \ln x + 2 = 2 (\ln x + 1) $

$ f'''(x) = (2 (\ln x + 1)) = \dfrac{2}{x} $

$ d^{3}f = \dfrac{dx^{3}}{x} $