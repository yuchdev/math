\section{Формула Тейлора. Разложение по формуле Маклорена некоторых элементарных функций}

\subsection{Введение}

Ряды Тейлора применяются при аппроксимации функции многочленами. В частности, линеаризация уравнений происходит путём разложения в ряд Тейлора и отсечения всех членов выше первого порядка.


\subsection{Вывод формулы Тейлора для многочлена степени n}

$ \SeriesSimple{P}{b}{x} $

Представим $ x = a+t $

$ \SeriesSimple{P}{b}{(a+t)} $

Раскроем скобки и приведем все подобные

$ \SeriesSimple{P}{A}{t} $

Сделем обратную замену $ t = x-a $

$ \SeriesSimple{P}{A}{(x-a)} $

Найдем коэффициенты $ A $

Очевидно, при последовательном дифференцировании

$ P'_n = A_1 + 2 A_2 (x-a) + 3 A_3 (x-a)^{2} + \ldots + n A_n (x-a)^{n-1}$

$ P''_n = 1 \cdot 2 A_2 + 1 \cdot 2 \cdot 3 A_3 (x-a) + \ldots + n (n-1) A_n (x-a)^{n-2}$

\ldots

$ P^{(n)}_n = n! A_n $

Взяв $ x = a $, мы сократим все члены со степенью выше 1.

Таким образом получим

\begin{tabbing}
\=aaaaaaaaaaaaaaaaa \=bbbbbbbbbbbbbbbbb \kill
\>$ P(a) = A_0 $		\>$ A_0 = P(a) $ \\
\>$ P'(a) = 1! A_1$		\>$A_1 = \dfrac{P'(a)}{1!} $ \\
\>$ P''(a) = 2! A_2$	\>$A_2 = \dfrac{P''(a)}{2!} $ \\
\>$ \ldots $ \\
\>$ P^{(n)}(a) = n! A_n$	\>$A_n = \dfrac{P^{(n)}(a)}{n!} $
\end{tabbing}

$ \\\\ $
Теперь сведем всё в одну формулу

$ P(x) = P(a) + \dfrac{P'(a)}{1!}(x-a) + \dfrac{P''(a)}{2!}(x-a)^{2} + \ldots + \dfrac{P^{(n)}(a)}{n!}(x-a)^{n}$

$ \\\\ $
Взяв $ x = 0 $, получим формулу Маклорена

$ P(0) = P(0) + \dfrac{P'(0)}{1!}x + \dfrac{P''(0)}{2!}x^{2} + \ldots + \dfrac{P^{(n)}(0)}{n!}x^{n}$

\subsubsection{Пример для многочлена}

Разложить многочлен $ P(x) = x^{2} - 3x + 2 $ по степеням $ x-1 $

Сначала найдем все производные (их две)

\begin{tabbing}
\=aaaaaaaaaaaaaaaaa		\=bbbbbbbbbbbbbbbbb \kill
\>$ P'(x) = 2x - 3 $	\>$ P'(1) = -1 $\\
\>$ P''(x) = 2 $		\>$ P''(1) = 2 $\\
\end{tabbing}

Очевидно  $ P(1) = 0 $

Теперь применим их в формуле Тейлора

$ P(x) = P(a) + \dfrac{P'(a)}{1!}(x-a) + \dfrac{P''(a)}{2!}(x-a)^{2} + \ldots + \dfrac{P^{(n)}(a)}{n!}(x-a)^{n}$

$ P(x) = 0 - \dfrac{1}{1!}(x-1) + \dfrac{2}{2!}(x-1)^{2} = (1-x) + (x-1)^{2} $

\subsection{Многочлен Тейлора для произвольной функции с остаточным членом}

Выпишем формулу Тейлора для $ n-1 $ первых членов

$ Q_{n-1} = 
f(a) + \dfrac{f'(a)}{1!}(x-a) 
+ \dfrac{f''(a)}{2!}(x-a)^{2} 
+ \ldots 
+ \dfrac{f^{(n-1)}(a)}{n!}(x-a)^{n-1}$

Если функция $ f(x) $ не является многочленом, в общем случае $ f(x) \ne Q_{n-1} $

Разница значений между ними назывется остаточным членом и обозначается $ R_n(x) $

Тогда

$$
f(x) = Q_{n-1} + R_n(x)
$$

\subsection{Вычисление остаточного члена формулы Тейлора}

Рассмотрим значение формулы Тейлора $ f(x) = Q_{n-1} + R_n(x) $ с остаточным членом на отрезке $ [a,b] $,
и возьмем $ x = b $

$ f(b) = Q_{n-1}(b) + R_n(b) $

Или подробно

$ f(b) = 
f(a) + \dfrac{f'(a)}{1!}(b-a) 
+ \dfrac{f''(a)}{2!}(b-a)^{2} 
+ \ldots 
+ \dfrac{f^{(n-1)}(a)}{n-1!}(b-a)^{n-1} + R_n(b) $

Очевидно, остаточный член следует искать в виде

$ R_n = M(b-a)^{n} $

Осталось найти коэффициент $ M $

Введем для этого вспомогательную функцию

$ \phi(x) = f(b) - Q_{n-1}(x) $

$ \phi(x) = f(b) - \left(  f(x) + \dfrac{f'(x)}{1!}(b-x)
+ \dfrac{f''(x)}{2!}(b-x)^{2} 
+ \ldots 
+ \dfrac{f^{(n-1)}(x)}{n-1!}(b-x)^{n-1} + M (b-x)^{n}\right)  $

Возьмем производную от этой функции. Очевидно, каждое слагаемое - это произведение функций, и должно раскладываться по соответствующей формуле.

$ \phi'(x) = 
f'(x)
+ \left( \dfrac{f''(x)}{1!}(b-x) - \dfrac{f'(x)}{1!} \right) 
+ \left(  \dfrac{f'''(x)}{2!}(b-x)^{2} - 2(b-x)\dfrac{f''(x)}{2!} \right) 
+ \ldots$

$+ \left(  \dfrac{f^{(n)}(x)}{n-1!}(b-x)^{n-1} -  \dfrac{f^{(n-1)}(x)}{n-1!}(n-1)(b-x)^{n-2} \right)
- M n(b-x)^{n-1}$

После группировки сокращаются все члены, кроме двух последних

$ \phi'(x) = \dfrac{f^{(n)}(x)}{n-1!}(b-x)^{n-1} - M n(b-x)^{n-1} $

Найдем $ M $, используя теорему Ролля (при выполнении условий теоремы существует точка, где производная $ = 0 $)

Первые два условия непрерывности и дифференцируемости очевидно выполняются.
Значения функции на концах отрезка равно нулю (именно для этого 'поворачивали' исходную функцию как в теореме Лагранжа или Коши).

Значит, $ \exists \phi(\xi) = 0 $

$ \phi(\xi) = \dfrac{f^{(n)}(\xi)}{n-1!}(b-\xi)^{n-1} - M n(b-\xi)^{n-1}  $

$ \phi(\xi) = -(b-\xi)^{n-1} \left( \dfrac{f^{(n)}(\xi)}{n-1!} - M n \right) =0 $

Очевидно $ -(b-\xi)^{n-1} \ne 0 $, т.к. $ \xi \in [a,b] $

Значит,

$ \dfrac{f^{(n)}(\xi)}{n-1!} - M n =0 $

$ M = \dfrac{f^{(n)}(\xi)}{n!} $

Теперь можно выразить остаточный член (он носит название остаточного члена в форме Лагранжа)

$ R_n = \dfrac{f^{(n)}(\xi)}{n!}(x-a)^{n} $

И выразить формулу Тейлора для произвольной функции $ f(x) $

$$
f(x) =
f(a) + \dfrac{f'(a)}{1!}(x-a) 
+ \dfrac{f''(a)}{2!}(x-a)^{2} 
+ \ldots
+ \dfrac{f^{(n-1)}(a)}{(n-1)!}(x-a)^{n-1} + \dfrac{f^{(n)}(\xi)}{n!}(x-a)^{n}
$$

Формула Тейлора при $ n=1 $ является обощением \textbf{формулы Лагранжа}

$ f(b) = f(a) + \dfrac{f(\xi)}{1!}(b-a) $

$ f(b) - f(a) = f(\xi)(b-a) $

\subsubsection{Альтернативная форма записи}

Остаточный член $ R_n = \dfrac{f^{(n)}(\xi)}{n!}(x-x_0)^{n} $ можно также записать в форме

$ R_n = \dfrac{f^{(n)}(x_0 + \theta (x-x_0))}{n!}(x-x_0)^{n} $

где $ 0 < \xi < 1 $

\subsubsection{Формула Маклорена}

Положив $ x_0 = 0 $, получим формулу Маклорена с остаточным членом в форме Лагранжа

$$
f(x) =
f(0) + \dfrac{f'(0)}{1!}x
+ \dfrac{f''(0)}{2!}x^{2} 
+ \ldots
+ \dfrac{f^{(n-1)}(0)}{(n-1)!}x^{n-1} + \dfrac{f^{(n)}(\xi)}{n!}x^{n}
$$

\subsubsection{Остаточный член в форме Пеано}

Возьмем альтернативную запись остаточного члена в форме Лагранжа

$ f^{(n)}(x_0 + \theta (x-x_0)) $

$ \theta (x-x_0) $ в данном случае $ \Delta x $. Поэтому по непрерывности

$ f^{(n)}(x_0 + \theta (x-x_0)) = f^{(n)}(x_0) + \alpha(x); x \to x_0, \alpha(x) \to 0 $

Заменим это выражение в формуле остаточного члена

$ R_n = \dfrac{f^{(n)}(x_0 + \theta (x-x_0))}{n!}(x-x_0)^{n} = 
\dfrac{f^{(n)}(x_0) + \alpha(x)}{n!}(x-x_0)^{n} =$

$ = \dfrac{f^{(n)}(x_0)}{n!}(x-x_0)^{n} + \dfrac{\alpha(x)}{n!}(x-x_0)^{n}
= \dfrac{f^{(n)}(x_0)}{n!}(x-x_0)^{n} + o( (x-x_0)^{n})
$

Полученная формула называется остаточным членом в форме Пеано

$$
R_n = \dfrac{f^{(n)}(x_0)}{n!}(x-x_0)^{n} + o( (x-x_0)^{n})
$$

Можно переписать формулу Тейлора с новым остаточным членом

$$
f(x) =
f(a) + \dfrac{f'(a)}{1!}(x-a) 
+ \dfrac{f''(a)}{2!}(x-a)^{2} 
+ \ldots
+ \dfrac{f^{(n-1)}(a)}{(n-1)!}(x-a)^{n-1} + \dfrac{f^{(n)}(a)}{n!}(x-a)^{n} + o( (x-a)^{n})
$$


\subsection{Разложение по формуле Маклорена некоторых элементарных функций}

TODO: графики приближения функций в Matlab

\begin{enumerate}
\item
$e^{x} = 1 + \dfrac{x}{1!} + \dfrac{x^2}{2!} + \dfrac{x^3}{3!} + \cdots + \dfrac{x^n}{n!} + o(x^{n})
= \sum^{\infty}_{n=0} \dfrac{x^n}{n!}$

\item
$ \ln(1+x) = x - \dfrac{x^2}{2} + \dfrac{x^3}{3} - \cdots + (-1)^n\dfrac{ x^{n+1}}{n+1} + o(x^{n})
= \sum^{\infty}_{n=0} (-1)^n\dfrac{ x^{n+1}}{n+1} =  \sum^{\infty}_{n=1} (- 1)^{n-1} \dfrac{x^n}{n} $

\item
$ \sin x =  x - \dfrac{x^3}{3!} + \dfrac{x^5}{5!} - \cdots\ + (-1)^n\dfrac{x^{2n+1}}{(2n+1)!} + o(x^{2n})
= \sum^{\infty}_{n=0} (-1)^n \dfrac{x^{2n+1}}{(2n+1)!} $

\item
$ \cos x =  1 - \dfrac{x^2}{2!} + \dfrac{x^4}{4!} - \cdots + \dfrac{(-1)^n}{(2n)!} x^{2n} + o(x^{2n})
= \sum^{\infty}_{n=0} \dfrac{(-1)^n}{(2n)!} x^{2n} $

\item
$ (1+x)^\alpha = 1 + mx + \dfrac{m(m-1)}{2!}x^2 + \cdots + \dfrac{m(m-1)\cdots(m-n+1)}{n!}x^n + o(x^{n})
= \sum^{\infty}_{n=0} \binom{m}{n} x^n = \sum^{\infty}_{n=0} \dfrac{m!}{n!(n-m)!}x^n $

\end{enumerate}

Пользуясь этими формулами, можно с заданной точностью вычислить число $ e $, тригонометрические функции и логарифмы.

$ e = 2 + \dfrac{1}{2!} + \dfrac{1}{3!} + \cdots $

Можно оценить остаточный член функции для уточнения значения.

\subsection{Использование формулы Маклорена для получения асимптотических оценок элементарных функций и вычисления пределов}

Асимптотические формулы для БМФ:

$\sin\alpha(x) = \alpha(x) + o(x)$

$\log_a(1+\alpha(x)) = \alpha(x)\dfrac{1}{\ln{a}} + o(x)$

$e^{\alpha(x)} = 1 + \alpha(x) + o(x)$

$(1+\alpha (x))^{m} =  1 + m\alpha(x) + o(x)$

Являются обобщением формулы Маклорена для $ n = 1 $

Можно применять при нахождении пределов.

\subsubsection{Пример}

$ \lim\limits_{x \to 0} \dfrac{x - \sin x}{x^{3}} =
\lim\limits_{x \to 0} \dfrac{x -  \left( x - \dfrac{x^{3}}{3!} + o(x^{4}) \right)  }{x^{3}} =
\lim\limits_{x \to 0} \dfrac{\dfrac{x^{3}}{3!} + o(x^{4}) }{x^{3}} =
\dfrac{1}{3!} + \lim\limits_{x \to 0} \dfrac{o(x^{4})}{x^{3}} =
\dfrac{1}{6}$

