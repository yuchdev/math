\section{Теоремы о среднем значении}

\subsection{Теорема Ролля}

Если вещественная функция непрерывна на отрезке $[a; b]$ и дифференцируема на интервале $(a; b)$, принимает на концах этого интервала одинаковые значения, то на этом интервале найдётся хотя бы одна точка, в которой производная функции равна нулю.

\subsubsection{Доказательство}

Если функция на отрезке постоянна, то утверждение очевидно, поскольку производная функции равна нулю в любой точке интервала.

Если же нет, поскольку значения функции в граничных точках сегмента равны, то согласно теореме Вейерштрасса, она принимает своё наибольшее или наименьшее значение в некоторой точке интервала, то есть имеет в этой точке локальный экстремум, и в этой точке производная равна 0.

Геометрическая интерпретация теоремы Ролля - если ординаты обоих концов гладкой кривой равны, то на кривой найдется точка, в которой касательная к кривой параллельна оси абсцисс.

\subsection{Существенность условий теоремы Ролля}

Насколько важны все условия - непрерывность, дифференцируемость и равенство на концах отрезка.

Например, для функции $ y = |x| $ выполняются все условия, кроме дифференцируемости - в нуле нет производной, теорема Ролля не выполняется

\subsection{Теорема Лагранжа о конечных приращениях}

Формула конечных приращений или теорема Лагранжа о среднем значении утверждает, что если функция $ f $ непрерывна на отрезке $[a; b]$ и дифференцируема в интервале $(a;b)$, то найдётся такая точка $ c\in (a;b)$, что

$$
\frac{f(b)-f(a)}{b-a}=f'(c)
$$

Является обобщением теоремы Ролля для $ f(b) = f(a) $

Для \textbf{доказательства} "повернём" график функции, который мы рассматривали в теореме Ролля следующим образом

$ F(x) = f(x) - f(a) - \dfrac{f(b)-f(a)}{b-a} (x-a) $ \\

Значения на концах $ F(a) = F(b) = 0 $

Найдем производную 

$ F'(x) = f'(x) - \dfrac{f(b)-f(a)}{b-a} $ \\

Т.е. $ \exists c: F'(c) = 0 $

$ F'(c) = f'(c) - \dfrac{f(b)-f(a)}{b-a} = 0  $

$ f'(c) = \dfrac{f(b)-f(a)}{b-a} $

Доказано.

Геометрически это можно переформулировать так: на отрезке $[a;b]$ найдётся точка, в которой касательная параллельна хорде, проходящей через точки графика, соответствующие концам отрезка.

\subsection{Формула Лагранжа (формула конечных приращений)}

Видоизменим ф-лу Лагранжа

$ f(b)-f(a) = f'(c)(b-a) $

Т.к. $ c \in (a,b) $, его можно представить в виде

$ c = a + \theta(b-a) $, где $ \theta \in (0,1) $

Подставим в формулу Лагранжа

$ f(b)-f(a) = f'(a + \theta(b-a))(b-a) $

Пусть $ \Delta x $ - расстояние от $ a $ до $ b $

$ a = x $

$b = x + \Delta x $

Тогда 

$ \Delta f(x) = f(x + \Delta x) - f(x) = f(b)-f(a) = f'(x + \theta \Delta x) \Delta x $

Обратите внимание, эта формула - точная версия

$ \Delta f(x) \approx f'(x) \Delta x $

\subsection{Пример применения теоремы Лагранжа}

Доказать

$ | \arctg x_2 - \arctg x_1| \le x_2 - x_1 $

Примем $ f(x) = \arctg x $ и применим к неравенству теорему Лагранжа

$ f(x_2)-f(x_1) = f'(c)(x_2-x_1) $

$ \arctg x_2 - \arctg x_1 =  \dfrac{1}{1 + c^{2}}(x_2 - x_1) $

Т.к. $ \dfrac{1}{1 + c^{2}} < 1 $, неравенство верно.

\subsection{Теорема Коши о среднем значении}

\begin{itemize}
\item 
$f(x)$ и $g(x)$ определены и непрерывны на отрезке $[a,b]$;

\item 
производные $\ f'(x)$ и $\ g'(x)$ конечны на интервале $\ (a,b)$;

\item 
производные $\ f'(x)$ и $\ g'(x)$ не обращаются в нуль одновременно на интервале $\ (a,b)$

\item 
$g(a) \neq g(b)$
\end{itemize}

тогда

$$
\frac{f(b)-f(a)}{g(b)-g(a)}=\frac{f'(c)}{g'(c)}, c \in (a,b)
$$

(Если убрать условие 4, то необходимо усилить условие 3: $ g'(x) $ не должна обращаться в нуль нигде в интервале $(a,b)$)

\textbf{Доказательство} аналогично теореме Лагранжа - "повернём" функцию так, чтобы она удовлетворяла теореме Ролля,
найдем точку $ c $, удовлетворяющую условию теоремы.

$ F(x) = f(x) - f(a) - \dfrac{f(b)-f(a)}{g(b)-g(a)}(g(x)-g(a)) $

$ F'(x) = f'(x) - \dfrac{f(b)-f(a)}{g(b)-g(a)}g'(x) $

$ f'(x) = \dfrac{f(b)-f(a)}{g(b)-g(a)}g'(x) $

$ \exists c: \dfrac{f(b)-f(a)}{g(b)-g(a)}=\dfrac{f'(c)}{g'(c)} $



\subsection{Связь теорем Ролля, Лагранжа, Коши}

Все три теоремы являются обобщение друг друга.

Теорема Коши является обощением теоремы Лагранжа для $ g(x) = x $

Теорема Лагранжа является обощением теоремы Ролля при $ f(b) = f(a) $