\section{Числовая последовательность и ее предел}

В математике пределом последовательности элементов пространства называют элемент того же пространства, который обладает свойством «притягивать», в некотором смысле, элементы данной последовательности. Свойство последовательности, иметь или не иметь предел, называют сходимостью: если у последовательности есть предел, то говорят, что данная последовательность сходится, в противном случае (если у последовательности нет предела) говорят, что последовательность расходится. Часто встречающимся является предел числовой последовательности.

Пределом последовательности точек топологического пространства является такая точка, каждая окрестность которой содержит все элементы последовательности, начиная с некоторого номера. Все открытые, в смысле данной топологии, множества, содержащие данную точку, образуют систему окрестностей этой точки. В метрическом пространстве систему окрестностей образуют, например, все открытые шары с центром в данной точке. Поэтому свойство сходимости последовательности элементов метрического пространства к данной точке формулируется как способность «удерживать» на заданном расстоянии все точки последовательности, начиная с некоторого номера.

Сходящиеся последовательности обладают следующим свойством: каждая подпоследовательность сходящейся последовательности сходится, и её предел совпадает с пределом исходной последовательности. Другими словами, у последовательности не может быть двух различных пределов. Может, однако, оказаться, что у последовательности нет предела, но существует подпоследовательность (данной последовательности), которая предел имеет. Если из последовательности точек пространства можно выделить сходящуюся подпоследовательность, то, говорят, что данное пространство компактно или, точнее, секвенциально компактно.

$$
\forall \varepsilon > 0 \exists N: \forall n,m > N \Rightarrow d(x_n, x_m) < \varepsilon
$$

где $d(x_n, x)$ - расстояние до предельной точки

В случае числовой последовательности

$$
\forall \varepsilon > 0 \exists N: \forall n,m > N \Rightarrow |x_{n} - x_{m}| < \varepsilon
$$