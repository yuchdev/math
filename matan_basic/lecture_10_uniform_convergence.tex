\section{Равномерная непрерывность. Сравнение бесконечно малых функций. Эквивалентность. Символы о и О}

\subsection{Равномерная непрерывность}

Равномерная непрерывность в математическом и функциональном анализе — это свойство функции быть одинаково непрерывной во всех точках области определения.

Числовая функция вещественного переменного $f:M \subset R \to R $ равномерно непрерывна, если
: $ \forall \varepsilon > 0  \exists \delta = \delta(\varepsilon)>0 \; \forall x_1,x_2 \in M\quad \bigl(|x_1-x_2| < \delta \bigr) \Rightarrow \bigl( |f(x_1)-f(x_2)| < \varepsilon\bigr).$

Здесь важно, что выбор $\delta$ зависит только от величины $\varepsilon$.

Геометрический смысл равномерной сходимости - в конечной скорости ее возрастания (или колебания).


\subsection{Сравнение бесконечно малых функций}


\subsubsection{Порядок малости}

Допустим, у нас есть бесконечно малые последовательности при одном и том же $x\to a$ величины $\alpha(x)$ и $\beta(x)$.

\begin{itemize}

\item 
Если ,$\lim\limits_{x \to a}\frac{\beta}{\alpha} = 0$, то $\beta$ — 
бесконечно малая \textbf{высшего порядка малости}, чем $\alpha$

Обозначают $\beta=o(\alpha)$

\item 
Если $\lim\limits_{x\to a}\frac{\beta}{\alpha} = \infty$, то $\beta$ -
бесконечно малая \textbf{низшего порядка малости}, чем $\alpha$.

Соответственно $\alpha=o(\beta)$

\item 
Если $\lim\limits_{x\to a}\frac{\beta}{\alpha} = C$ (предел конечен и не равен 0), то $\alpha$ и $\beta$ являются бесконечно малыми величинами \textbf{одного порядка малости}.

Это обозначается как $\beta=O(\alpha)$ или $\alpha=O(\beta)$ (в силу симметричности данного отношения).

\end{itemize}



\subsubsection{Эквивалентность БМФ}

Частным случаем одного порядка является эквивалентность БМФ

Если $\lim\limits_{x\to a}\frac{\beta}{\alpha} = 1$ (предел конечен равен 1), то $\alpha$ и $\beta$ являются \textbf{эквивалентными} БМФ.

Если предел не существует, то такие функции называются \textbf{несравнимыми} БМФ.

\subsubsection{Порядок малости}

Если $\lim\limits_{x\to a}\frac{\beta}{\alpha^m}=c$ (предел конечен и не равен 0), то бесконечно малая величина $\beta$ имеет $m$-й порядок малости относительно бесконечно малой $\alpha$.

Для вычисления подобных пределов удобно использовать правило Лопиталя.

\subsubsection{Таблица эквивалентностей}

$\sin\alpha(x)\thicksim\alpha(x);$

$\mathrm{tg}\,\alpha(x)\thicksim\alpha(x);$

$\arcsin{\alpha(x)}\thicksim\alpha(x);$

$\mathrm{arctg}\,\alpha(x)\thicksim\alpha(x);$

$\log_a(1+\alpha(x))\thicksim\alpha(x)\cdot\frac{1}{\ln{a}}$, где $a>0$;

$\ln(1+\alpha (x))\thicksim\alpha(x);$

$a^{\alpha(x)}-1\thicksim\alpha(x)\cdot\ln{a}$, где $a>0$;

$e^{\alpha(x)}-1\thicksim\alpha(x);$

$1-\cos{\alpha(x)}\thicksim\frac{\alpha^2(x)}{2};$

$(1+\alpha(x))^\mu-1\thicksim\mu\cdot\alpha(x),\quad\mu\in R$, поэтому используют выражение:

: $\sqrt[n]{1+\alpha(x)}\approx\frac{\alpha(x)}{n}+1$, где $\alpha(x)\xrightarrow[x\to x_0]{}0$.

\subsubsection{Главный степенной член функции}

Если функция $ а(x) \thicksim \alpha x^{m}$, то говорят, что $ \alpha x^{m} $ - 
главный степенной член функции.


\subsubsection{Теорема о замене эквивалентных}

Доказательство - по определению предела. 

Пусть $ \alpha \thicksim \alpha_{1} $, $ \beta \thicksim \beta_{1} $

Домножаем отношение эквивалентных $ \frac{\alpha_{1}}{\beta_{1}} $ на 
$ \frac{\alpha}{\beta}  \frac{\alpha}{\beta} $

После сокращения получим $ \frac{\alpha}{\beta} $

\subsubsection{Теорема об условии эквивалентности}

Немного неочевидно, но:

Функции $ \alpha $ и $ \beta $ эквивалентны, если их разность является БМФ 
более высокого порядка, чем они сами.

Доказательство: разность разделить на любую из функций, получим разность минус 1.
Переносим 1, получаем определение эквивалентности.

\subsection{Символы Ландау}

$ o(f) $, \textbf{о малое от $ f $} обозначает \textbf{бесконечно малое относительно $ f $}, пренебрежимо малую величину при рассмотрении. Смысл термина \textbf{О большо}е зависит от его области применения, но всегда  $ f $ растёт не быстрее, чем $ O(f) $, \textbf{O большое от $ f $} 

\subsubsection{o-малое}

$ f(x) = o( \phi(x) ) $ если $ \lim\lim\limits_{x \to x_{0}} \dfrac{f(x)}{ \phi(x)} = 0 $

Т.е. $ f(x) $ бесконечно малая по сравнению с $ \phi(x) $

Частный случай: $ f(x) = o( 1 ) $. В этом случае $ f(x) $ - сама БМФ.

Примеры:

$ f(x) = o( 1 ) $

$ x = o( x^2 ), x\to \infty $

$ x^2 = o( x ), x\to 0 $

\subsubsection{O-большое}

$ f(x) = O( \phi(x) ) $ существует такое $ M $, что

$ |f(x)| \le M| \phi(x) | $

Т.е. $ f(x) $ ограничена сверху функцией $ \phi(x) $
(с точностью до постоянного множителя) асимптотически.

Частный случай:

$ f(x) = O( 1 ) $ означает, что функция $ f(x) $ ограничена в некоторой окрестности.

Примеры: см. курс алгоритмов.

\subsubsection{Свойства символов Ландау}

$o(f)+o(g) = o(f)$

$o(f) \cdot o(g) = o(f \cdot g)$

$o(o(f)) = o(f)$

$o(f)+O(f) = O(f)$

$o(f) \cdot O(f) = O(f \cdot g)$

$O(f) \cdot O(g) = O(f \cdot g)$

$O(o(f)) = o(f)$ 

$x^{\alpha} = o(x^{\beta}, x\to \infty ), \alpha < \beta $ 

$x^{\beta} = o(x^{\alpha}, x\to 0 ), \alpha < \beta$ 

\subsubsection{Асимптотически равные функции}

От отношения эквивалентности $ \alpha(x) \thicksim \beta(x) $ 
можно перейти к отношению равенства $ \alpha(x) = \beta(x) + o(x) $
по теореме об условии эквивалентности.

Например:

$ e^x - 1 = x + o(x) $

$ \sin{x} = x + o(x) $

$ \ln(1+x) = x + o(x) $