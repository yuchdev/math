\section{Уравнение финансового рычага}

$ ROE $ - Норма доходности акционерного капитала

$ ROA  $ - Норма доходности активов

$ TA  $ - (Total Asset) Полный капитал

$ CR  $ - (Credit)

$ IR  $ - Процентная ставка (Interest rate)

$$ ROE = (1 - T)ROA \dfrac{CR}{TA(ROA-IR)} $$

\section{Уравнение устойчивого роста}

$ RG $ - коэффициент устойчивого роста

$ RL $ - коэффициент удержания прибыли

$$ RG = RL \cdot ROE $$

Очевидно, что без привлечения внешнего финансирования коэфф. устойчивого роста не может быть больше $ ROE $.
Он может быть равен ему в случае, когда вся прибыль идет на развитие бизнеса и не выплачиваются дивиденды акционерам.