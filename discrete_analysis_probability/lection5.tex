\chapter{Лекция 5. 6 октября 2011}


\section{Разбиение чисел на слагаемые}

$n \in N, \quad  n = x_1 + x_2 + \dots + x_t$

Сколькими способами можно разбить $n$ на слагаемые? Чтобы ответить на этот вопрос, нужно как минимум наложить ограничения на $t$ и на $x$.
 

$x_i \in X, x_i \in \naturalset^{+}$

Нам интересны два случая:
\begin{itemize}
\item Порядок слагаемых имеет значение. 
\begin{example}
Например, нам надо выпить. Необходимо набрать 500 г алкоголя, есть пиво, вино, водка и т.п. У нас есть рюмка, в которую мы можем наливать разные напитки и каждый раз в ней может быть разное количество чистого алкоголя. Порядок нам важен по понятным причинам.
\end{example}
 
\item Порядок слагаемых не имеет значения
\begin{example}
Есть капуста, она весит 5 кг. На одной чаше весов --- капуста, на другой --- гири, надо набрать их и взвесить капусту.
Сколькими способами можно выбрать гири так, чтобы уравновесить капусту?
\end{example}
 
\end{itemize}

 \subsection {Порядок слагаемых важен}
Пусть надо разбить число $n = x_1 + x_2 + \dots + x_t$ на слагаемые, порядок важен. При этом все слагаемые являются элементами множества:

$\forall i : x_i \in \{n_1, \dots, n_k\} $

Примем за $f(n; n_1, \dots, n_k)$  число разбиений $n$ в этой ситуации, когда есть ограничение на параметры. 

\begin{thm}
\[ 
f(n; n_1, \dots, n_k) = f(n-n_1; n_1, \dots, n_k) + f(n-n_2; n_1, \dots, n_k) + \dots + f(n-n_k; n_1, \dots, n_k)
\]

Это рекуррентное соотношение. При этом начальное условие:

$n < 0 \Rightarrow f(n; n_1, \dots, n_k) = 0$

$f(0; n_1, \dots, n_k) = 1$
\end{thm}

\begin{cor}
$f(n, 1,2,\dots, n) = 2^{n-1}$
\end{cor}
\begin{proof} По индукции:

$f(n, 1,2,\dots, n) = \phi (n) = \\  = f(n-1, 1,2,\dots, n-1) +  f(n-2, 1,2,\dots, n-2) + \dots + f(1;1) + f(0; 1, \dots n) = \phi(n-1) + \phi(n-2) + \dots + \phi(1) + 1 = \\
= 2^{n-2} + 2^{n-3} + \dots + 2^0 + 1 = 2^{n-1} -1 + 1 = 2^{n-1}$
\end{proof}

\subsection{Порядок слагаемых неважен}

Аналогично, обозначим за  $F(n; n_1, \dots, n_k)$  число разбиений $n$. Нельзя действовать полностью аналогичным образом в этом случае, потому что нет однозначно определенного понятия первого слагаемого.

\begin{thm}
\[F(n; n_1, \dots, n_k ) = F(n-n_1; n1,\dots, n_k) + F(n; n_2 \dots, n_k)\]

Аналогично, обозначим 
$p(n) = F(n; 1,2,\dots, n)$


\end{thm}
Однако оценить $p(n) $ непросто. Полный вывод есть в книге Эндрюса.


\begin{harram}
\[\displaystyle p(n) \sim \frac 1 {4n \sqrt 3} e^{\pi \sqrt{\frac 2 3} \sqrt n}\]
\end{harram}
 
Для работы с неупорядочными разбиениями используется диаграммная техника.

\subsection{Диаграммная техника --- Диаграммы Юнга}

 Пусть есть натуральное число $n = x_1 + \dots + x_t, \quad x_1 \leq x_2 \leq \dots \leq x_t$. Порядок имеет значение. Будем считать, что слагаемые упорядочены в порядке неубывания.
 
$x_1 \leq x_2 \leq \dots \leq x_t$

Нарисуем следующую диаграмму. Рядом с $x_i$ соответствующее его значению количество точек. Всего точек $n$ штук.
  
\begin{picture}(10,60)

\put(0,30){$x_1$}
\multiput(20,32)(10,0){5}{\circle*{4}}

\put(0,20){$x_2$}
\multiput(20,22)(10,0){7}{\circle*{4}}

\put(0,10){$\dots$}

\put(0,0){$x_t$}
\multiput(20,2)(10,0){10}{\circle*{4}}

\end{picture}

Так мы однозначно закодировали одно из разбиений. Такие диаграммы (называемые диаграммы Юнга) удобны для некоторых доказательств. 

\begin{thm}[1]
Число неупорядоченных разбиений числа $n$ на не более $k$ слагаемых равно числу разбиений числа $n+k$ на $k$ слагаемых. 
\end{thm}

\begin{proof}
Рассмотрим диаграмму произвольного разбиения n на $\leq k$ слагаемых. В ней не более $k$ строчек. Пририсуем слева к диаграмме столбец из ровно $k$ точек. Теперь в ней всего $n+k$ точек,  а строк ровно $k$. Так мы получили диаграмму разбиения  $n+k$ на точно $k$ слагаемых.
Таким образом получается биекция между двумя "видами" диаграмм.
\end{proof}

\begin{thm}[2]
Число неупорядоченных разбиений числа n на не более k слагаемых равно числу разбиений числа $n+\frac {k(k+1)} 2 $ на ровно $k$ различных слагаемых. 
\end{thm}

\begin{proof}
Добавим слева треугольничек.  Он гарантированно даёт разные величины слагаемых при прибавке. К общему числу точек он добавит как раз $\frac {k(k+1)} 2$.
\end{proof}

\begin{rem}
Диаграммы можно также транспонировать, получая двойственные им.
\end{rem}


\begin{thm}[3]
Число разбиений числа n на не более k слагаемых,равно числу разбиений n на слагаемые, величина каждого из которых не больше k.
\end{thm}


\section{Бесконечное формальное произведение}

Мы просто делаем преоборазования по привычным правилам.

$ \begin{aligned}
&(1-x) (1-x^2) \dots (1-x^n) \dots  = \\
&(1 - x - x^2 + x^3) (1-x^3) \dots (1-x^n) \dots = \\
&(1 - x - x^2 + x^4 \dots = \\
&1 - x - x^2 + x^5 + x^7 - x^{12} - x^{15} + \dots 
\end{aligned}
$


\begin{thm}

Если $n = \frac {3k^2 \pm k} 2,  k \in \naturalset $ то коэффициент при $ x^n $  равен  $(-1)^k$ иначе  0.

\end{thm}


\begin{proof}
Считаем коэффициент при $ x^n $ 

$
(-x ^{n_1} )(-x ^{n_2} ) \dots  (-x ^{n_t} ) = (-1)^t x^{n_i + \dots + n_t}
$

То есть, $x^n$ возникает только тогда, когда число $n$ можно разбить на слагаемые $n_1 \dots n_t$, при этом знак при нём зависит от того, чётно ли количество этих слагаемых.

Тогда коэффициент при $x_n$ можно записать так:

\[\sum \limits ^x _{n_1,\dots, n_t: \\ n = n_1 + \dots + n_t} (-1)^t = |n_{\text{чёт}} - n_{\text{нечёт}}|
\]
 
 Здесь $n_{\text{чёт}}$ --- количество разбиений числа $n$ на чётное число слагаемых, $n_{\text{нечёт}}$ --- количество разбиений числа $n$ на нечётное число слагаемых.
 

\end{proof}

\begin{cor}
Из доказательства явно следует факт:

$
|n_{\text{чёт}} - n_{\text{нечёт}}|= \begin{cases}
0, n \neq \frac {3k^2 \pm k} 2, k \in \naturalset \\
(-1)^k, \text{в противном случае}
\end{cases}
$
\end{cor}
\section {Формальные степенные ряды}

Назовём формальным степенным рядом следующую картинку :
$a_0 + a_1 x + \dots + a_n x^n + \dots$

\subsection{Операции}

Рассмотрим два формальных степенных ряда:

$A = a_0 + a_1 x + \dots + a_n x^n + \dots $

$B = b_0 + b_1 x + \dots + b_n x^n + \dots $

\begin{enumerate}
\item Сложение

$C = c_0 + c_1 x + \dots + c_n x^n + \dots , \quad \forall i : c_i = a_i + b_i$

\item Произведение

$C = c_0 + c_1 x + \dots + c_n x^n + \dots , \quad \forall i : c_i = a_i b_i + a_1 b_{i-1} + \dots + a_{i-1} b_1 + a_i b_0$

\item Деление

Формируется ряд $C = A / B$ такой, что $ CB = A $.


$\begin{cases}
c_0b_0 = a_0\\
c_0b_1 + c_1b_0 = a_1\\
c_0b_2 + c_1b_1 + c_2b_0 = a_2\\
\dots
\end{cases}
$

Общая формула для $i$-ого члена такая:

$C_i = - \frac 1 {a_0} \sum \limits _{i=1} ^n a_i b_{n-i}, \quad n \geq 1$


Например, $ \frac 1 {(1-x)} = 1 + x + x^2 + \dots $.

Заметим, что мы получили формулу суммы геометрической прогрессии! Конечно, если рассматривать ряд справа аналитически, необходимо ограничение $|x| <1 $ и т.п.

\begin{example}

$
\frac 1 {(1-x^2)^2} = \frac 1 {1 - 2x^2 + x^4} = 
\begin{cases}
\frac 1 {(1-x)^2} \frac 1 {(1+x)^2} \\
\bigg ( \frac 1 {1-x^2} \bigg ) ^2
\end{cases}
$ Одно и то же!

Теперь подробнее о первом случае.

$\frac 1 {1-x} = 1 + x + x^2 + \dots $

$ \frac 1 {1+x} = 1 - x + x^2 - \dots $

$ \bigg ( \frac 1 {1-x} \bigg ) ^2 = 1 + 2x + 3x^2 + \dots + (n+1)x^n + \dots $

$ \bigg ( \frac 1 {1+x} \bigg ) ^2 = 1 - 2x + 3x^2 - \dots + (-1)^n (n+1)x^n + \dots $

$ \bigg ( \frac 1 {1-x} \bigg ) ^2 \bigg ( \frac 1 {1+x} \bigg )^2 = \dots (1 - (-1)^n (n+1) + 2(-1)^{n-1} n + \dots + (n+1) 1) x^n \dots $

Обозначим коэффициент за $C_n$. Теперь покажем, что полученный коэффициент тождественно равен коэффициенту в другом случае.

$\frac 1 {1-x^2} = 1 + x^2 + x^4 + \dots + x^{2k} + \dots $

$\bigg( \frac 1 {1-x^2}\bigg)^2 = 1 + 2x^2 + 3 x^4 + \dots + (k+1) x^{2k} + \dots $

Отсюда имеем коэффициент $C'_n = 
	\begin{cases}
	 0, n = 2k+1 \\
	 (k+1), n = 2k
	 \end{cases} 
	 $

Очевидно, что это то же самое, что и в первом случае.
\end{example}

\end{enumerate}

 
\begin{rem} Напоминание об обычных рядах.

Представим, что формальный ряд стал рядом обычным, аналитическим. Запишем его, как

$\sum \limits _{k=0} ^\infty a_k x^k $

Определим $\rho = \frac 1 {\displaystyle \overline{\lim}_{k \to \infty}  \sqrt[k] {|a_k|} } $


Тогда ряд сходится при всех $ |x| < \rho $

 
Рассмотрим некоторые примеры, чтобы уловить суть формулы.

$\sum \limits _{k=0} ^\infty \underbrace{ 2^k } _{a_k} x^k $

Легко увидеть, что этот ряд сходится, когда $|x |< \frac 1 2$, потому что тогда он становится геометрической прогрессией со знаменателем, меньшим 1. Но зачем  верхний предел?

$\sum \limits _{k=0} ^\infty, a_k = 
\begin{cases}
 3^k, k = 2l \\
 2^k, k = 2l + 1
 \end{cases}
$

Тут обычного предела не существует, верхний же имеется. Этот ряд сходится на $ |x| < \frac 1 3$.

\end{rem} 
