\documentclass[12pt]{book}
 
\usepackage{lineno}
\usepackage[utf8]{inputenc}

\usepackage[all]{xy}

\usepackage{geometry} 
\geometry{a4paper}
\geometry{margin=1in} 


\usepackage{graphicx} 

 
\usepackage{array} 
\usepackage{ulem} % underlines, double underlines, wavy underlines etc
\usepackage{amsmath, amsthm, amssymb}
\usepackage{amsfonts} % Натуральные числа и прочие множества красивыми буковками
\usepackage{paralist} %списки: itemize, enumerate etc
\usepackage{subfig} 
\usepackage[english,russian]{babel} 
 

\usepackage{fancyhdr} 
\pagestyle{fancy}
\renewcommand{\headrulewidth}{1pt}

\renewcommand{\baselinestretch}{1.4}

\lhead{Дискретный анализ и теория вероятностей }\chead{}\rhead{}
\lfoot{}\cfoot{\thepage}\rfoot{}

\usepackage{sectsty}
\allsectionsfont{\sffamily\mdseries\upshape} 
	
\everymath{\displaystyle}
\modulolinenumbers[3]


%%% %%% ToC (table of contents) APPEARANCE
\usepackage[nottoc,notlof,notlot]{tocbibind} % Put the bibliography in the ToC
\usepackage[titles,subfigure]{tocloft} % Alter the style of the Table of Contents
\renewcommand{\cftsecfont}{\rmfamily\mdseries\upshape}
\renewcommand{\cftsecpagefont}{\rmfamily\mdseries\upshape} 


%%%Theorems declarations here%%%
  
\newtheorem{alem}{Лемма} 
\newtheorem{thm}{Теорема}
\newtheorem{example}{Пример}
\newtheorem{task}{Задача}
\newtheorem*{erdesh}{Теорема Эрдеша-Ко-Радо}

\newtheorem*{mainarithm}{Основная теорема арифметики}
 
\newtheorem*{incexc}{Формула включений и исключений}
\newtheorem*{harram}{Теорема Харди-Рамануджана}
\newtheorem*{cyclestask}{Задача о циклических последовательностях}
\newtheorem*{stirling}{Формула Муавра-Стирлинга}
\newtheorem*{keli}{Формула Кэли}
\newtheorem*{pantrkur}{Теорема Пантрягина-Куратовского}


\newtheorem*{rem}{Замечание}
\newtheorem*{solution}{Решение}

\newtheorem*{cor}{Следствие} % Просто следствие

%%%Всякие команды для удобства

\newcommand\naturalset{\mathbb{N}}	%Множество натуральных чисел
\newcommand\poset{\mathfrak{P}}		%Чум
\newcommand\complexset{\mathbb{C}}	%Множество комплексных чисел
\newcommand{\aset}{\mathbb{A}}

%Флаги
\newif\ifdraft  %Черновой вариант - с недоделанными и потерянными кусками конспекта, а также нумерация строчек.
\drafttrue


%Зависимости от флагов
\ifdraft
\linenumbers %номера строк для удобства.
\fi
%%% END Article customizations
 
\title{Дискретный анализ и теория вероятности}
\author{Курс Андрея Михайловича Райгородского}
\date{2010-2011}  
\begin{document} 

\maketitle
 
 
\chapter{ Лекция 1 }
\section{Основы комбинаторики.}
\subsection{Основные правила}

 
Пусть $A, B$ --- множества объектов. В $A$ ровно $n$ объектов, в $B$ ровно $m$ объектов.

 $A =  \{ a_1, \dots, a_n \} $ 
 
 $B =  \{ b_1, \dots, b_m \} $
 
\begin{description}
\item[Правило сложения]~ 

Если элемент из множества $A$ можно выбрать $n$ способами, а элемент из множества $B$ можно выбрать m способами, то выбрать элемент из  $A$ или из $B$ можно $n+m$ способами.
\item[Правило умножения]  ~

Если мы берём сначала элемент из $A$, затем элемент из $B$, то количество способов $m \times n $ .
\item[Принцип Дирихле] ~

Если $n+1$ кроликов сидят в $n$ клетках, в одной клетке точно есть по крайней мере два кролика. 
\end{description}

\subsection{Сочетания, размещения, перестановки}

Пусть $A$ --- множество объектов. 

$A =  \{ a_1, \dots, a_n \}$

Зафиксируем $k \in \naturalset $. Подмножество $A$ мощности $k$ называется \textit{ $k$-сочетанием (без повторений) } . 

Если учитывать порядок элементов, то говорим уже о размещениях.

Есть также сочетания с повторениями, когда мы можем брать совпадающие элементы, но всё-таки сваливать их в кучу. Их порядок по-прежнему может быть любым.

Размещения с повторениями --- такой набор элементов из множества $A$, где могут встречаться повторяющиеся элементы, однако порядок внутри этого набора нам важен. 

$C^k_n$ --- количество $k$-сочетаний без повторений.

$\bar{C}^k_n$ --- количество $k$-сочетаний с повторениями.

$A^k_n$ --- количество $k$-размещений без повторений.

$\bar{A}^k_n$ --- количество $k$-сочетаний без повторений.


По правилу умножения

\[ \bar{A}^k_n  = n^k .\]

Аналогично, 

\[ A^k_n = n(n-1) ... (n-k+1) = \frac {n!} {(n-k)!} .\]

Следствие:

\[ C^k_n = \frac{a^k_n} {k!} = \frac {n!} {(n-k)! k!} .\]

\begin{description}
\item[Перестановка] --- один из способов переставить \ элементы множества в различном порядке. Другими словами, это биекция множества на себя. Вполне очевидно, что количество перестановок выражается, как

$ A^n_n = n!$.

\end{description}

\begin{alem} Имеет место равенство:

$ \bar{C}^k_n = C^k_{n+k-1}$

\end{alem}

\begin{proof} 

Рассмотрим $ A = \{a_1, ..., a_n\} $. Извлекаем "кучки" \  в сумме $k$ объектов. 
Сопоставим $k$-сочетанию битовую строку:

$
\underbrace {11} _{\text{ количество } a_1} 0 
\underbrace {111} _{\text{ количество } a_2} 0...011$ --- в коде (строке) всего $ k+n-1 $ символов.

Если количество $a_i$ равно нулю, то ставим ноль вместо соответствующих ему единиц. По последовательности длины $n+k-1$ c $k$ нулями можно восстановить $k$-сочетание с повторениями. 
\end{proof}
 
\section{Задача о перестановке букв в слове}
Дано слово из $N$ букв. Оно состоит из символов $a_1, \dots, a_n$, каждый из которых встречается $\alpha_i$ раз. Нам необходимо посчитать количество слов, которое можно составить из этих букв. Естественно, что так как буквы могут повторяться, то количество слов может оказаться меньше, чем количество перестановок из букв.
 
Мы можем обобщить задачу. Пусть у нас есть $n$ объектов, из них $n_1$ объектов первого типа, $n_2$ объектов второго типа, и т.л. Объектов $k$-ого типа --- $n_k$, Сколько существует различных последовательностей  из этих объектов?

\begin{thm} Это количество выражается следующей формулой:

$ P(n_1, \dots, n_k) = \frac{n!}{n_1 ! n_2! \dots n_k !} $
\end{thm}

\begin{proof}
Всего есть $n= \sum \limits _i n_i$ позиций, на которые мы можем размещать объекты. Способов выбрать позиции для $a_1$ у нас $C^{n_1}_n$. Как только эти позиции зафиксированны, объекты первого типа уже расставлены, у нас осталось $n-n_1$ позиций. Таким образом, способов разместить $a_2$ --- $C^{n_2} _{n-n_1}$. Мы перемножаем все эти способы, последовательно размещая объекты.

$\displaystyle 
C^{n_1}_n \times C^{n_2}_{n-n_1} \times \dots \times C^{n_k}_{n-n_1-\dots-n_{k-1}}= \\
= \frac {n!} {n_1! (n-n_1)!} \frac {(n-n_1)!} {n_2! (n- n_1 - n_2)!} \times \dots \times \frac {(n-n_1 - \dots - n_{k-1})!} {n_k! 
  \underbrace{(n-n_1 - \dots - n_k)!} _1 } = \\
= \frac {n!} {n_1! n_2! \dots n_k!}
$
\end{proof}





\section{Бином Ньютона и полиномиальная формула}

\subsection{Бином Ньютона}

\[ (x+y)^n=\sum \limits_{k=0}^n C^k_n x^k y^{n-k}  \] 

$C^k_n$ называются биномиальными коэффициентами.

\subsection{Полиномиальная формула} 
\[\displaystyle
(a_1 + a_2 + \dots + a_k)^n = \sum \limits _{n_1+\dots + n_k = n} \frac {n!} {n_1! \dots n_k!} a_1^{n_1} a_2^{n_2} \dots a_k^{n_k} 
\]
\begin{proof}   
TODO Число способов выбрать x с коэффициентом 

Расписываем $ \sum\limits_{i=0}^k x_k ^n $, получаем требуемое. 
\end{proof}

В связи с последней формулой числа $ P(n_1,...,n_k) $ называются полиномиальными коэффициентами .
Иными словами, это количество способов, которыми можно составить слово из букв $a_i$, каждая из которых встречается $n_i$ раз, а всего в слове $n $ букв.

\[ (x_1 + ... + x_k)^n = \sum _{\substack{
   \forall n_i : n_i\geq 0; \\
 n_1 + ... + n_k = n
    } }
 P(n_1, n_2, ...,n_k)x_1^{n_1}x_2^{n_2}...x_k^{n_k}
\]

\section{Тождества}
\begin{enumerate}
 


\item $ C ^k _n = C^{n-k}_n $

\item $ C^ k _n = C^k _{n-1} + C^{k-1} _{n-1}$

\begin{proof} Комбинаторно можно рассуждать так. Пусть у нас есть множество объектов

$A = \{a_1, a_2,  \dots, a_n\}$.

Тогда $ C^k_n $ --- количество k-сочетаний из этого множества объектов, а $ C^k_{n-1} $ --- количество $k$-сочетаний из множества $A'$:

$A' = \{a_2,  \dots, a_n\} = A \backslash \{a_1\} $

$ C^{k-1}_{n-1} $ --- это количество  $k-1$-сочетаний из $A'$. Но любое такое сочетание, будучи дополненным символом $a_1$, даст $k$-сочетание!

Другими словами, любое количество сочетаний $C^k_n$ складывается из числа сочетаний, не содержащих $a_1$ (коих   $ C^k_{n-1} $  ), и из сочетаний, содержащих $a_1$ (количество сочетаний, обязательно содержащих $a_1$ ---  $ C^{k-1}_{n-1} $ ).
\end{proof}

\item 

$C^0_n + C^1_n + \dots + C^n_n = 2^n$
\begin{proof}[Доказательство 1] 
Представим как бином:

$(1+1)^n = 2^n = \sum \limits _{i=0} ^n {C^ i _n} $
\end{proof}
\begin{proof}[Доказательство 2] 
Рассмотрим множество $n$-ок (кортежей с фиксированным числом элементов), состоящих из нулей и единиц.

$ A = \{ (0,\dots, 1,0, \dots, 1 )\} $

Тогда, очевидно, мощность такого множества $2^n$. Посчитаем её другим способом. Найдём количество $n$-ок, в которых ровно $k$ единиц. Очевидно, их будет $C^k_n$. Просуммировав их, получим доказываемое тождество.
\end{proof}

\item $ \sum  _{\substack{
     (n_1, ..., n_k): \\   
      n_i \geq 0, \sum n_i = n }}
    P(n_1, ..., n_k) = k^n$ 
\begin{proof}
Подставим вместо иксов в полиномиальную формулу единицы. 
\end{proof}

\item $\sum \limits _{i=0} ^n {(C^ i _n )^2 } = C^n _{2 n}$
\begin{proof}
Рассмотрим множество :
\[
A = \{a_1, ..., a_n, a_{n+1}, ... ,a_{2n} \} , \quad |A| = 2n .
.\]

Сколько существует сочетаний без повторений?
Взяв $k$ элементов из первой половины, из второй мы возьмём $n-k$.

\[ C^ n _{2 n} = \sum \limits_{k=0}^n C^k _n C^{n-k} _n = \sum \limits _{k=0} ^n (C^k _n) ^2 \]
\end{proof}


\item Выведем формулу для суммы степеней натуральных чисел от 1 до $k$.

Рассмотрим множество мощности $n+1$:

$ \{a_1, \dots, a_{n+1} \}, \quad m \in \naturalset $ 

Всего в нём столько $m$-сочетаний с повторениями: 

\[ C^m _{(n+1)+m-1} = C^m_{n+m} \]

Разобьём эти сочетания на несколько классов:

\begin{enumerate}

\item[1] Все те, которые не содержат объекта $a_1$. Их $C^m_{n+m-1}$

\item[2] Все те, которые содержат ровно один объект $a_1$. Их $C^{m-1}_{n+m-2}$

\dots

\item[m] Все те, которые содержат ровно $m$ объектов $a_1$
\end{enumerate}

Итак, предварительный итог нашей работы:

\[
\begin{aligned}
C^m_{n+m} = &C^m_{n+m-1} + C^{m-1}_{n+m-2} + \dots = \\
&C^{n-1}_{n+m-1} + C^{n-1}_{n+m-2} + \dots 
\end{aligned}
.\]

Теперь рассмотрим разные случаи:

\begin{itemize}
\item 
$n= 1 \Rightarrow C^1_{m+1} = \underbrace{1+1+\dots+1}_{m +1 \text{ штук}}$

\item 
$
n= 2 \Rightarrow 
C^2_{m+2} = C^1_{m+1} + C^1_{m} + C^1_{m-1}+ \dots = 
1 + 2 + 3 + \dots + m + (m+1)
$ 

Это арифметическая прогрессия. Cлева мы имеем

$\displaystyle C^2_{m+2} = \frac {(m+1)!} {2! m!} = \frac {(m+2) (m+1)} {2}$.

Таким образом, мы вывели формулу суммы арифметической прогрессии.

\item 
$
n= 3 \Rightarrow 
C^3_{m+3} = C^2_{m+2} + C^2 _{m+1} + \dots
$

$\displaystyle  \frac {(m+1) (m+2) (m+3) } 6 = \frac {(m+1)(m+2)} 2 + \frac {m(m+1)} 2 + \frac {m(m-1)} 2 + \dots = $

$=\displaystyle  \frac {(m+1)^2} 2 + \frac {(m+1)} 2 + \frac {m^2} 2 +  \frac {(m-1)^2} 2 +  \frac {(m-1)} 2 + \dots = $

 
 $\displaystyle =\frac 1 2 \bigg(1^2 + 2^2 + \dots + (m+1)^2 \bigg) + \frac 1 2 \bigg( \underbrace{ 1 + 2 + \dots + (m+1)} _{\frac {(m+1)(m+2) } 2} \bigg) $


$\displaystyle 1^2 + 2^2 + \dots + (m+1)^2 = 2 \bigg( \frac {(m+1) (m+2) (m+3) } 6 - \frac {(m+1)(m+2)} 4 \bigg) = $

$\displaystyle = (m+1)(m+2) \bigg( \frac {m+3} 3 - \frac 1 2  \bigg) $

Окончательно имеем

\[\displaystyle  1^2 + 2^2 + \dots + k^2  = \frac {k(k+1)(2k+1)} 6 .\]

\item $n \geq 4$ \quad Аналогичные рассуждения помогут нам найти формулу для суммы кубов и т.п.



\end{itemize}


\begin{rem}
Натуральные числа включают 0
\end{rem}

  
\item 
$ C^0_n - C^1_n + C^2_n - \dots + (-1)^n C^n_n = 
\begin{cases} 
0, \quad n \geq 1\\
1, \quad n = 0
\end{cases}
$
\end{enumerate}
 
TODO доказательство $(1-1)^n$

\begin{incexc} 
Даны множества $ S_1, S_2, \dots , S_n $
\[ |S_1 \cup S_2 \cup \dots \cup S_n| = |S_1| + |S_2| + \dots + |S_n| - |S_1 \cap S_2| - \dots + |S_1 \cap S_2 \cap S_3| - \dots + (-1)^{n+1} |S_1 \cap \dots \cap S_n|
\]
--- "интуитивно понятная формула". 
\end{incexc}

Теперь рассмотрим объекты $ a_1, \dots, a_N$ и свойства $ \alpha _1, \alpha _2, \dots, \alpha _M$ 

$ N(\alpha _i , \alpha _j, \alpha _k, \dots) $ --- количество объектов из исходного множества, которые обладают перечисленными в скобках свойствами. Не обязательно только ими!

$ \alpha ' _i $ --- свойство "не обладать свойством  $ \alpha _i $"

Другой вариант формулы включений и исключений имеет вид:

$N(\alpha'_1, \dots, \alpha'_M) = N - N(\alpha_1) - N(\alpha_2) - \dots - N(\alpha_1) + \\
+ N(\alpha_1,\alpha_2) + N(\alpha_1,\alpha_3) + \dots + N(\alpha_{N-1},\alpha_N) + \dots + (-1)^M N(\alpha_1,\dots, \alpha_M)  $

TODO есть множество из m объектов, m-размещения с повторениями. Свойств n штук.

\begin{cyclestask}

Дан алфавит  $ X = \{b_1, \dots, b_n \}$. Будем составлять слова длины $n$ из его символов:

$ a_1 \quad  a_2  \quad \dots  \quad a_n , \quad a_i \in\chi $

\setlength{\unitlength}{1mm}
\begin{picture}(40,10)
\put(0,7){\vector(1,0){5}}
\put(0,7){\circle*{1}}
\put(8,7){\vector(1,0){5}} 
\put(8,7){\circle*{1}}
\put(26,7){\vector(1,0){5}}
\put(26,7){\circle*{1}}
\end{picture}

Если мы хотим подчеркнуть тот факт, что наша последовательность обычная, линейная, незацикленная, будем рисовать точки вместо символов и соединять их направленными стрелками. Количество таких "обычных линейных последовательностей" \ ---  $r^n$.
 
 Для краткости так будем обозначать линейное слово следующим образом:
 
$a_1 a_2 \dots a_n 
$.


Пусть дано линейное слово $a_1 a_2 \dots a_n $. Назовём циклическим сдвигом преобразование:
\[
a_1 a_2 \dots a_n  \mapsto 
a_2 a_3 \dots a_n a_1
\]

\newpage  %TODO тут надо просто сделать неразрывный блок, иначе рисунок переносится на новую страницу.

Что будет, если мы зациклим последовательность? Добавим еще одну стрелку, которая идёт из $a_n$ в $a_1$.

$ a_1 \quad  a_2  \quad \dots  \quad a_n , \quad a_i \in\chi $

\setlength{\unitlength}{1mm}
\begin{picture}(40,20)
\put(0,17){\vector(1,0){5}}
\put(0,17){\circle*{1}}
\put(8,17){\vector(1,0){5}} 
\put(8,17){\circle*{1}}
\put(26,17){\vector(1,0){5}}
\put(26,17){\circle*{1}}
\put(31,17){\circle*{1}}
\qbezier(31,17)(15,2)(4,13)
\put(4,13){\vector(-1,1){4}}
\end{picture}
  
Такая последовательность называется циклической. 

 $ (a_1 a_2 \dots a_n ) $ --- циклическое слово.

Нас интересует $T_r(n)$ --- число всех принципиально различных циклических последовательностей. 

\begin{example}
Возьмём алфавит $\chi = \{C,H,O\}$. Составим слово и зациклим его:

\[
 \xymatrix{ C \ar[r] & O \ar[d] \\
               O \ar[u] & H \ar[l] }
\]

Сколько различных линейных слов отвечает данному циклическому? Очевидно, что четыре. 
\end{example}

\begin{example}
Сколько различных линейных слов отвечает данному циклическому?
\[
 \xymatrix{ C \ar[r] & O \ar[d] \\
               O \ar[u] & C \ar[l] }
\]

Здесь ответ, очевидно, два.

$COCO \mapsto  OCOC  \mapsto  COCO $
\end{example}

Чтобы вывести формулу для $T_r(n)$, нам потребуются некоторые дополнительные выкладки.

\end{cyclestask}

 
\section{Формула обращения Мёбиуса}

 \begin{description}
 \item[Простое число] --- натуральное число, большее единицы, которое делится только на самое себя и на единицу. 
 \end{description}
  
\begin{mainarithm}
Для любого числа $ n \in \naturalset, n > 1$ верно: 

$ n = p_1 ^ {\alpha_1} \dots p_S ^ {\alpha_S} $, где $p_1,\dots, p_S$ --- простые числа, а $\alpha_1,\dots, \alpha_S \in \naturalset$ 
\end{mainarithm}


\subsection{Функция Мёбиуса}
 
 Введём следующую функцию, называемую функцией Мёбиуса:
 
\[ 
\mu (n) = 
\begin{cases} 
	1, 
		\quad n = 1\\
	0, 
		\quad n = p_1^{\alpha_1} p_2^{\alpha_2} \dots p_k^{\alpha_k}, 		
		\quad \text{если } \exists i : \alpha_i \geq 2\\
	(-1)^S,
		 \quad n = p_1 \dots p_S .
\end{cases}
\]

Примеры:

$\mu (7) = -1 $

$\mu (10) = 1 $

$\mu (12) = 0 $
 
\begin{alem}

\[
\displaystyle
\sum \limits _{d|n} \mu(d) = 
\begin{cases} 
1, \quad n = 1 \\
0, \quad n > 1
\end{cases}
\] 
$  \quad d|n $- "$d$ является делителем $n$".


\end{alem}



\begin{proof}
Рассмотрим случаи:
\begin{itemize}

\item 
$
n = 1, \quad \sum\limits_{d|n} \mu(d) = \mu(1) = 1
$
\item
TODO развернуто более
$
n > 1 \Rightarrow    n = p_1 ^ {\alpha_1} \dots p_S ^ {\alpha_S} $

$
 d = p_1^{\beta_1} p_2^{\beta_2} \dots p_s^{\beta_s}, 0 \leq \beta_i \leq \alpha_i
$

Если 
$
\exists i : \quad \beta_i > 1 \Rightarrow \mu(d) = 0
$


$\displaystyle
\sum \limits _{d|n} \mu(d) =  
\sum\limits_{\beta_1 = 0}^1 \dots \sum \limits _{\beta_S = 0} ^1 \mu (p_1^{\beta_1}\dots p_S^{\beta_S} ) = \\
\sum \limits ^S _{k=0} \sum _{\substack {
( \beta_1, \dots, \beta_S) : \\
\beta_1 + \dots + \beta_S = k
}} (-1)^k = \sum \limits _{k=0} ^S C^k_S (-1)^k = 0 
$
\end{itemize}
\end{proof}
\subsection{Формула обращения Мёбиуса}

Пусть $ f,g: \naturalset \mapsto \naturalset,  f(n) = \sum \limits _ {d | n} g(d). $ Тогда 
\[
g(n) = \sum \limits _{d|n} \mu(d) f(\frac n d )
\]

\begin{proof}
$ \displaystyle
\sum \limits _{d|n} \mu(d) \sum \limits _{d'| ^n/_d} g(d') = \\
\sum \limits _{d|n} g(d) \sum \limits _{d'| ^n/_d} \mu(d') = \dots \\
$

Почему можно менять суммы местами? Рассмотрим $ n = 12 $ 

$
\begin{aligned}
&d = 1 :  &d' \in \{1,2,3,4,6,12\} \\
&d = 2 :  &d' = \{1,2,3,6\} \\
&d = 3 :  &d' = \{1,2,4\} \\ 
&d = 4 :  &d' = \{1,2,3\} \\
&d = 6 :  &d' = \{1,2\} \\
&d = 12: &d' = \{1\} \\
\end{aligned}
$

$
\begin{aligned}
&\mu(1)(g(1) + g(2) + g(3) + g(4) + g(6) + g(12)) + \\
+&\mu(2)(g(1) + g(2) + g(3) + g(6)) + \\
+&\mu(3) (g(1) + g(2) + g(4)) \\
+&\mu(4) (g(1) + g(3)) +\\
+&\mu(6)(g(1)+g(2)) \\
+&\mu(12)g(1)
\end{aligned}
$

- перегруппируем - 

$ 
g(1)(\mu(1) + \mu(2) + \mu(3) + \mu(4) + \mu(6) +   \mu(12) ) +\\
g(2)(\mu(1) + \mu(2) + \mu(3) + \mu(6)) + \dots
$  --- уже тут видно, что мы можем поменять суммы местами.


$
\displaystyle
d<n \Rightarrow \frac n d > 1,  \sum \limits _{d' | ^n/ _d} \mu(d') = 0
$

$
\displaystyle 
\begin{aligned}
0 &= g(n)\bigg(\sum\limits_{d'|1}\mu(d') \bigg) + \sum _{\substack {d|n \\ d < n}}g(d)\sum\limits_{d'|^n/_d} \mu(d') = \\
&= g(n) + \underbrace{\sum _{\substack {d|n \\ d < n}}g(d)\sum\limits_{d'|^n/_d} \mu(d') } _{ 0 \text{ по лемме} } = g(n)
\end{aligned}
$, что и требовалось доказать.



\end{proof}
 

Вернёмся к задаче о циклических словах и выведем формулу для $T_r (n)$. 

$X = \{b_1, \dots, b_r\} $ --- алфавит


\begin{description}
\item[Период линейной последовательности] это минимальное число сдвигов, при котором она переходит в себя.
\end{description}

\begin{alem}
Период последовательности обязательно является делителем длины для любого линейного слова. 
\end{alem}
 
 

Пусть $V$ --- множество линейных слов длины $n$.

$d_1, \dots, d_S $ --- все возможные делители числа $n$.

$|V| = r^n $, очевидно.

\begin{rem}
$ \sqcup $ --- символ дизъюнктного (без пересечений) объединения
\end{rem}

Разобьём  $V$ на непересекающиеся множества:


$
V = V_1 \sqcup V_2 \sqcup  \dots \sqcup V_S , \quad \{d: d|n\} = \{d_1, \dots, d_S\}
$

$V_i$ --- множество слов длины n и периода $d_i$.



\begin{alem} Если слово имеет длину $n$ и период $d$, то оно выглядит так:

$ 
a_1 \dots a_d a_1  \dots a_d \dots a_1 \dots a_d
$

и имеет $^n /_d$ "блоков".
\end{alem}
\begin{proof}
Очевидно, так как первые $d$ символов должны "переехать" \ в следующие $d$.
\end{proof}

\begin{cor} 
Между множеством $V_i$ и множеством  $W_i$ линейных последовательностей длины и периода $d_i$ есть взаимно однозначное соответствие (биекция). Значит, $|V_i| = |W_i|$.
\end{cor}


Значит, 
$r^n = \sum \limits _{d|n} |W_i|$
 
   
Обозначим за $U_i$  множество всех различных циклических слов, которые получаются из слов множества $W_i$ добавлением "стрелочки". Тогда

$ d_i |U_i| = |W_i|  $

Положим $|U_i| = M(d_i)$

$ r^n = d_1M(d_1)+\dots + d_S M(d_S) = \sum \limits _{d|n} dM(d) $

Если мы обозначим $r^n = g(n), \quad dM(d) = f(d)$, то по формуле обращения Мёбиуса 

$
f(n) = \sum \limits _{d|n} \mu(d)g\bigg(\frac n d\bigg) $, где $f(n) = nM(n)$/

Здесь $\displaystyle g\bigg(\frac n d \bigg) =  r ^ {^n/_d}$


$\displaystyle
f(n) = n M(n) =  \sum \limits _{d|n} \mu(d) r ^ {^n/_d}
$

$M(n)$ выражает число циклических последовательностей, которые отвечают линейным последовательностям длины $n$ с периодом $n$ 

Число же всех циклических последовательностей выражается, как 

$\displaystyle
T_r(n) = \sum \limits _{d|n} M(d) = \sum \limits _{d|n} \frac 1 d \sum \limits _{d'|d} \mu(d')r^{^d/_{d'}} 
$.

%lection1 ends


\chapter{Лекция 2}
\section{Общая формула обращения Мёбиуса}

\subsection{Частично упорядоченное множество} 

\begin{description}
\item[Чум] - частично упорядоченное множество $ \poset $, на котором выполняется отношение частичного порядка  $ \preceq $ ("предшествует или равен") для некоторых пар элементов. 
\end{description}

Отношение частичного порядка подчиняется следующим аксиомам:

\begin{enumerate}
\item 
$\forall a \in \poset: a
\preceq
a
$
\item
$ a = b \Longleftrightarrow 
\begin{cases}
&a \preceq b \\
&b \preceq a
\end{cases}
$

\item
$
a \preceq b \& b \preceq c \Rightarrow a \preceq c
$

\end{enumerate}


Примеры:

\begin{itemize}
\item Натуральные числа:  $\poset = \naturalset$, причём 
$\preceq$ соответствует $\leq$
 
\item Натуральные числа:  $\poset = \naturalset$, причём 
$\preceq$ соответствует делимость.

$x \preceq y \Leftrightarrow x | y $

\item
$
V = \{ 1, 2, \dots , n\}, P = 2^V
$ (множество всех подмножеств V, называемое также множество-степень) 

$A, B \in P, \quad A \preceq
B \Longleftrightarrow A \subset
 B
$
\end{itemize}

\subsection{Функция Мёбиуса на Чумах}

\begin{rem} Отношение предшествования, строгое и нестрогое.

$ \prec $ эквивалентно одновременному выполнению $\preceq $ и $\neq $.
\end{rem}

Можно ввести функцию Мёбиуса для произвольных ЧУМов рекуррентным путём для двух аргументов:
 


\[
\mu(x,x) = 1 \]

\[
\mu(x,y) = - \sum \limits _{
z: x \preceq
z \prec 
y
} \mu(x,z), \quad x \prec y 
\]
 
Пусть $ p_i $ --- простое число, а отношение частичного порядка --- делимость. Тогда:

$
\mu(1,1) = 1
$

$
\displaystyle \mu(1,2) = -\sum \limits _{1 \preceq z \prec 2} \mu(1,z) = -1
$

$\displaystyle
\mu(1,p) =  -\sum \limits _{1 \preceq z \prec p} \mu(1,z) = -1
$
 
 

$\displaystyle
 \mu(1,p_1, p_2) = -\sum \limits _{z: 1, p_1, p_2} \mu(1,z)  =
- ( 
\underbrace {
\mu (1,1) 
} _{1}
+
\underbrace {
 \mu(1,p_1) 
} _{-1}
+
\underbrace{
 \mu (1, p_2)
} _{-1} 
) 
= 1
$

По индукции для $ p_1 p_2 \dots p_S $ легко показать, что: 

$
\mu(S) = (-1)^S
$

$
\mu(1,p_1^{\alpha_1} \dots p_i^{\alpha_i} ) = 0 , \quad \exists \alpha_i > 1
$

Заметим также следующие случаи:

$
\mu(p^2) \substack {= -} _{z: 1,p} (
\underbrace{
	\mu(1)
} _{1}
 + 
\underbrace{
 \mu(p)
 }
 _{-1
 }
 )  = 0
$

$
\mu(p^k q) = 0
$ 

\begin{rem}
Мы можем также задать функцию для одного аргумента: $\displaystyle \mu(x,y) = \mu\bigg(\frac y x\bigg)$
\end{rem}
 


\subsection{Обобщение формулы обращения Мёбиуса для произвольных чумов}

Пусть $\poset$ --- чум, в котором выполняется условие: каждый главный идеал конечен. В виде логической формулы это выглядит так:

$\forall x \in \poset | \{ y \in \poset : y \preceq x\} | < \infty $ 

Пусть также $f,g : \poset \mapsto \complexset, \quad g(y) = \sum \limits 
_{x
\preceq
y
}
f(x)  $. Тогда:

\[\displaystyle
 f(y) = \sum \limits _{
x \preceq
y
}
\mu(x,y) g(x)
\]


Частный случай, с которым мы имели дело ранее ( для сравнения ):

$
\begin{aligned}
g(n) = \sum \limits _{d|n} f(d) \Rightarrow 
f(n)  &  = \sum \limits _{d | n} \mu(d) g \bigg ( \frac n d \bigg )=\\
& = \sum \limits _{d|n} g(d) \mu \bigg ( \frac n d \bigg )
\end{aligned}
$




\subsection{Вывод формулы включений и исключений из общей формулы обращения Мёбиуса}


Рассмотрим произвольные множества $S_1, S_2,\dots, S_n $.
Чум $\poset$ состоит из всех пересечений элементов $S_1, \dots , S_n$, то есть:

$\displaystyle
\poset = \bigg\{ \bigcap _{i \in I}S_i : I \subseteq \{ 1,2, \dots, n\} \bigg \}
$

Отношение упорядочивания, которое мы будем иметь в виду --- вложение. 
$\preceq \Longleftrightarrow \subseteq$

Заметим также, что $|\poset| \leq 2^n$

\begin{rem}
Объединением подмножеств по элементам пустого множества будем считать пересечение всех подмножеств $S_i$.

$\displaystyle \bigcup \limits _{i \in \emptyset}S_i = S_1 \cap S_2 \cap \dots \cap S_n $
\end{rem}

Рассмотрим $P_1, P_2 \in \poset $. Как посчитать $\mu ( P_1, P_2 )$ ?

$\displaystyle
P_1 = \bigcup \limits _{i \in I_1} S_i, \quad P_2 = \bigcup \limits _{i \in I_2} S_i$

Правильный ответ на этот вопрос --- $\mu ( P_1, P_2 ) = (-1) ^ {|I_1| - |I_2|} $. Обоснуем это.


Введем функции $f(P)$ и $ g(P) $.

$f(P) = |P|$

$g(P) = |\{x: x \in P \& x \notin P'\prec P \}|$

То есть, $g(P)$ показывает, сколько элементов из $P$ не являются элементами одного из строгих предшественников $P$.

Очевидно, что эти две функции связаны между собой следующим образом:

$\displaystyle
f(P) = \sum \limits _{P' \preceq P} g(p) $


Если мы применим формулу Мёбиуса, то получаем:

$\displaystyle
g(P) = \sum \limits _{P' \preceq P} \mu(P', P) g(P')$

Возьмём $P = S_1 \cap \dots \cap S_n$. 
Очевидно, что $g(P) = 0$. Тогда:

$\displaystyle
0 = \sum \limits _{P' \preceq P} (-1)^{|I(P')|}|P'|$

$|I(P')|$ --- количество элементов пересечений, составляющих $P'$


Рассмотрим отдельно слагаемое, где $P' = P$. $I(P) = \emptyset, |I| = 0$. Вынесем один член из суммы

$\displaystyle
0 = \sum \limits _{P' \preceq P} (-1)^{|I(P')|}|P'| = |S_1 \cup \dots \cup S_n| + \sum \limits _{P' \prec P} (-1)^{|I(P')|}|P'| $

$\displaystyle 
\begin{aligned}
|S_1 \cup \dots \cup S_n| &= \sum \limits _{P' \prec P} (-1)^{|I(P')|+1}|P'| \\
 & = \sum \limits _{i=1} ^n \sum \limits _{I: |I| = i} (-1)^{|I(P')|+1}|\bigcap \limits _{j \in I} S_j|
 \end{aligned} $
 

Мы получили формулу включений-исключений в явном виде.


\section{Асимптотики для биномиальных коэффициентов}
\begin{thm}
$
C_n^0 < C_n^1 < \dots < C_n^{\lfloor  n/2\rfloor} \geq C_n^{\lfloor n/2\rfloor + 1} $
\end{thm}
\begin{rem}
А если экстраполировать график из значений C, получится график нормального распределения.
\end{rem}

\begin{thm}
\begin{itemize}
\item
$C^n_{2n} < 4^n$, то есть любой из биномиальных коэффициентов $C^i_{2n}$ меньше $4^n$, ведь это их сумма!
\item
$\displaystyle
C^n_{2n} > \frac {4^n} {2n+1}$, так как всех биномиальных коэффициентов $2n+1$ штук.
\end{itemize}
\end{thm}

Эти неравенства легко получить, но они малоинформативны.

\begin{stirling}
\[\displaystyle 
n! \sim \sqrt{2 \pi n} \bigg (\frac n e \bigg) ^n  
\]
\end{stirling}


Проведём различные оценки.



$ C^n_{2n} \sim \frac 
	{\sqrt {4 \pi n} \bigg(\displaystyle\frac {2n} e \bigg)^{2n}} 
	{\bigg( \sqrt {2 \pi n} \bigg (\displaystyle \frac n e \bigg )^n \bigg ) ^2 } = \frac {4^n} {\sqrt{\pi n}}
$

Человек, знающий формулу Стирлинга, впадает в искушение применять её чаще, чем она способна дать хорошие результаты. Попробуем сделать другие оценки.

\begin{enumerate}

\item
$\displaystyle C^k_n 
 =\frac {n!} {k! (n-k)!} = \frac {n(n-1) \dots (n-k+1)} {k!}= \\
	=\frac {n^k} k! \bigg(1-\frac 1 n \bigg) \bigg(1 - \frac 2 n \bigg) \dots\bigg(1 - \frac {k-1} n\bigg) =\\
 =\frac {n^k} {k!} \exp \bigg ( ln  \bigg((1-\frac 1 n) (1 - \frac 2 n ) \dots (1 - \frac {k-1} n) \bigg) \bigg) \leq\\
 \leq \frac {n^k} {k!} \exp \bigg( - \frac 1 n - \frac 2 n - \dots - \frac {k-1} n \bigg) = \frac {n^k} {k!} e^{- \frac {k(k-1)} {2n} }
  $
  
\begin{rem}
Заметим, что в показателе экспоненты появилось $\displaystyle \frac {k^2} 2$, как и в функции плотности вероятности для нормального распределения!
\end{rem}

\item $\displaystyle  
C^k_n = \frac {n^k} {k!} \exp \bigg( - \frac 1 n - \frac 2 n - \dots - \frac {k-1} n + O\bigg ( 
\underbrace{
\frac 1 {n^2} + \frac 4 {n^2} + \dots + \frac {(k-1)^2} {n^2}
  \bigg )}
  _{\displaystyle \frac {k^3} {n^2}}
   \bigg) 
  $

\begin{cor}
Если $k = o(\sqrt{n}) $, то $\displaystyle C^k_n \sim \frac {n^k} {k!}$
\end{cor}
\item
$\displaystyle
\frac {C^{^n /_2-x}_n} {C^{^n /_2}_n}= \frac {n!} {\displaystyle\bigg(\frac n 2 - x\bigg)!\bigg(\frac n 2+x\bigg)!} \frac {\displaystyle \bigg ( \frac n 2 \bigg)! \bigg( \frac n 2 \bigg)!} {n!} = \frac {\displaystyle \frac n 2 \bigg ( \frac n 2 - 1\bigg ) \dots \bigg( \frac n 2 - x +1 \bigg)} {\displaystyle \bigg ( \frac n 2 + 1\bigg) \bigg(\frac n 2 + 1\bigg) \bigg( \frac n 2 + 2 \bigg) \dots \bigg ( \frac n 2 + x \bigg )} $

Сократим на $\frac n 2 $:

$
\frac {\displaystyle\bigg ( 1- \frac 2 n \bigg ) \bigg ( 1 - \frac 4 n \bigg) \dots \bigg ( 1 - \frac {2(x-1)} n \bigg)} {\displaystyle \bigg (1 + \frac 2 n \bigg ) \bigg (1 + \frac 4 n\bigg) \dots \bigg ( 1 + \frac {2x} n \bigg )} = \exp \bigg(  -\frac {2x (x-1)} {2n} - \frac {2x(x+1)} {2n} + O\bigg(\frac {x^3} {n^2}\bigg) \bigg) = \\
= e^ {\displaystyle - \frac {2x^2} n +  O\bigg(\frac {x^3} {n^2}\bigg)}
$

\begin{cor}
Если $x = o(\sqrt{n}) $, то $\displaystyle C^{\displaystyle^n /_2-x}\sim C^{\displaystyle^n /_2}$
\end{cor}
\end{enumerate}
\newcommand\realset{\mathbb{R}}
\begin{thm}
Пусть $a > 1, a \in \realset$. Тогда $ C ^n _ {[an]} = \bigg ( \frac {a ^a} {(a-1)^{a-1}} + o(1)\bigg)^n$
\end{thm}

\begin{rem}
$\bigg(C+ o(1) \bigg) ^n, c > 1$ --- подобно ли это выражение $C^n $? Очевидно, нет:

$\bigg (2 + \frac 1 {\sqrt n} \bigg ) ^ n  = 2^n \underbrace {\bigg(1+\frac 1 {2 \sqrt n } \bigg )} _{\displaystyle \approx e^{^{\sqrt n}} / _2}$

\end{rem}


\begin{rem} Пусть $P(n)$ --- произвольная функция, не слишком быстро растущая или убывающая (медленнее любой экспоненты).

\[ P \sim \pm e ^ {o(n)} \Rightarrow  P(n)(C + o(1) )^n = (c + o(1) ) ^n \]

Обратите внимание, $o(1) $ в левой и правой частях формулы внутри себя содержат различные функции!
\end{rem}

\begin{proof}
$C^n_{[an]} = \frac {[an]!} {n!([an]-n)!} \sim P_1(n) \frac {\displaystyle \bigg ( \frac {[an]} e \bigg ) ^ {[an]}} {\displaystyle\bigg( \frac n e \bigg) ^n \bigg ( \frac {[an]-n} e \bigg ) ^ {[an]-n}}  $

Нам хотелось бы избавиться от взятия целой части. Покажем, что мы можем заменить $P_1$ на какую-то другую функцию, тоже изменяющуюся не быстрее экспоненты, и при этом избавиться от взятия целой части. Для этого упростим выражение по частям:

$[an]^{[an]} = (an - \epsilon)^{an-\epsilon}  = (an)^{an-\epsilon} \underbrace {\bigg(1- \frac \epsilon {an} \bigg) ^ {an - \epsilon} }
_{\text{возьмём за } P_2(n)} = P_3(n) (an)^{an}, \quad \epsilon \in [0,1] $

Это выражение уже годится на роль $P_2(n)$, поскольку убывает или возрастает точно не экспоненциально, а медленнее. 

$(an)^{-\epsilon} $ тоже изменяется медленнее экспоненты, поэтому мы можем составить из этого выражения и $P_2 $ новую функцию $P_3$

Вернёмся к исходному выражению. Со знаменателем можно проделать аналогичные действия. Так мы получим:

$C^n_{[an]} = P_4(n) \frac {(an)^{an}} { n^n (an - n)^{an-n}} = P_4(n) \bigg ( \frac {a^a} {(a-1)^{a-1}} \bigg ) ^n = \bigg( \frac {a^a} {(a-1)^{a-1}} + o(1) \bigg )^n$
\end{proof}

\begin{cor}
Пусть $a \in \realset, b \leq \frac a 2$. Тогда:

$\sum \limits _{k=0} ^{[b_n]} C^k_{[an]} = \bigg( \frac  {a^a} {b^b(a-b)^{a-b}} + o(1)\bigg ) ^n $
\end{cor}
\begin{proof}
Действительно, рассмотрим максимальное слагаемое из суммы. В ходе аналогичных доказательству выше рассуждений мы приходим к:

$ C^{[b_n]}_{[an]} = \bigg( \frac  {a^a} {b^b(a-b)^{a-b}} + o(1)\bigg )^n $

Отличие всей суммы от максимального слагаемого --- не более чем в число слагаемых раз. 

$C^{[b_n]}_{[an]} \leq \Sigma \leq [b_n] \times  C^{[b_n]}_{[an]} $
Эта разность вполне вбирается в себя $o(1)$.
\end{proof}

\begin{thm}
Пусть $n = n_1 + \dots + n_k, \forall i: n_i \sim a_i n, a_i \in (0,1)$
Тогда:

$ P(n_1, \dots, n_k) = \bigg ( \frac 1 {a_1^ {a_1} a_2 ^ {a_2} \dots a_k^{a_k} } + o(1) \bigg) $
\end{thm}

\begin{rem}
Пусть $p_1 , \dots, p_k, \quad \sum p_i = 1$ --- набор вероятностей.

$H(p_1, \dots, p_k) = - \sum \limits _{i=1} ^k p_i \ln p_i$ --- выражение для энтропии.

Как мы можем заметить:

$ \frac 1 {a_1^ {a_1} a_2 ^ {a_2} \dots a_k^{a_k} }  = e ^ {H(a_1, \dots, a_k) }$
\end{rem}

\begin{task}
Рассмотрим систему неравенств:

$\begin{cases}
p^3 \leq x(1-x) ^{3n} (1-y)^{C^t_n} \\
(1-p)^{C^2_t} \leq y(1-x) ^{\frac  {nt^2 } 2 } (1-y)^{C^t_n}
\end{cases}
$

Здесь:

$p= p(t) \in (0,1) $

$x = x(t) \in (0,1)$

$y = y(t) \in (0,1)$

$n = n(t) \in \naturalset, t\in \naturalset$

Найдите для каждого $t$ максимальное $n$ такое, что существуют  $p$,$x$,$y$ такие, что система выполняется. Иными словами, надо найти $n(t)$ максимально быстрорастущую с точностью до константы, такую, для которой можно подобрать $p$,$ x$ и $y$, удовлетворяющие системе.
\end{task}

%lection 2 ends

 

\chapter{Лекция 4}

\section{Основы теории графов}

\begin{description}
\item[Граф] это пара из двух множеств --- вершин и рёбер. $ G = (V,E)$. Для нас $ |V| < \infty $.

$E $  --- любой набор различных пар несовпадающих элементов $V$, то есть отсутствие кратных рёбер и петель. Также граф изначально подразумевается неориентированным. 

\item[Мультиграф] разрешает кратные рёбра.
\item[Псевдограф] разрешает петли.
\item[Орграф] задаёт ориентацию рёбер, т.е. $(x,y) \neq (y,x) $

\item[Маршрут] --- последовательность вида вершина --- ребро --- вершина --- ребро и т.п. 
\item[Простая цепь]  ---  маршрут, где все вершины различны (и все рёбра, автоматически, тоже).
\item[Простой цикл] --- маршрут, где все вершины, кроме первой и последней, различны.
\item[Эквивалентность вершин.] Две вершины называются эквивалентными, если между ними в графе существует простая цепь.
\item[Компоненты связности] --- в графах это классы эквивалентности.
\item[Степень вершины $\deg v$] --- количество рёбер, в которые она входит как элемент пары.

В случае с орграфом мы говорим уже о двух характеристиках:

\begin{description}

\item[indeg $v$] --- количество рёбер, входящих в $v$;

\item[outdeg $v$] --- количество рёбер, выходящих из $v$.

\end{description}

Очевидно, что сумма всех степеней вершин равна удвоенному числу рёбер.

$\sum \limits _{v \in V} \deg v = 2 |E|  $

Интересным и важным вопросом явлется распределение степеней у вершин различных графов.

\item[Изоморфизм графов] Два графа $G(V_1, E_1) $ и $H(V_2, E_2$ изоморфны, если

 $\exists \phi : V_1 \leftrightarrow V_2 \quad \& \quad \bigg ( (x,y) \in E_1 \Leftrightarrow  (\phi(x), \phi(y)) \in E_2 \bigg ) $

Имеется проблема существования полиномиального алгоритма, проверяющего два графа на изоморфизм.


\item[Дерево] --- связный граф, у которого нет циклов (ациклический).

У понятия дерева есть еще три равносильных определения:

\begin{itemize}

\item Граф, у которого между любыми двумя вершинами есть ровно один путь по рёбрам (простая цепь).
\item Связный граф, у которого рёбер на единицу меньше, чем вершин.
\item Ациклический граф, у которого рёбер на 1 меньше вершин. 

\end{itemize}

\item[Полный граф] --- граф, в котором между любыми двумя вершинами есть ребро. Обозначается как $K_n$, где $n$ --- число вершин.
\item[Планарность графа.] Говорят, что граф планарен, если его можно изобразить на плоскости без пересечения рёбер. Два полных графа внизу изоморфны и планарны, видно, что граф справа --- это копия левого, но изображённая без пересечений рёбер.  

\setlength{\unitlength}{0.5mm}
\begin{picture}(60,60)

\put(0,0){\line(0,1){50}}
\put(0,0){\line(1,0){50}}
\put(0,0){\line(1,1){50}}

\put(50,50){\line(0,-1){50}}
\put(50,50){\line(-1,0){50}}
\put(0,50){\line(1,-1){50}}
 
\put(0,0){\circle*{5}}
\put(0,50){\circle*{5}}
\put(50,0){\circle*{5}}
\put(50,50){\circle*{5}}

\put(100,0) {
\begin{picture}(400,60)

\put(25,25){\line(1,1){25}} 
\put(25,25){\line(1,-1){25}} 
\put(25,25){\line(-1,0){25}} 
\put(0,25){\line(2,1){50}} 
\put(0,25){\line(2,-1){50}} 
\put(50,50){\line(0,-1){50}} 

\put(25,25){\circle*{5}}
\put(0,25){\circle*{5}}
\put(50,0){\circle*{5}}
\put(50,50){\circle*{5}}
\end{picture}
}
\end{picture}

Полный же граф на 5 вершинах не планарен.

\begin{picture}(400,60)
%\multiput(0,0)(10,0){7}{\line(0,1){60}}\multiput(0,0)(0,10){7}{\line(1,0){60}} 

\put(0,20){\circle*{5}}
\put(10,0){\circle*{5}}
\put(30,0){\circle*{5}}
\put(40,20){\circle*{5}}
\put(20, 40){\circle*{5}}

\put(0,20){\line(1,1){20}}
\put(10,0){\line(-1,2){10}}
\put(10,0){\line(1,0){20}}
\put(0,20){\line(1,1){20}}
\put(30,0){\line(1,2){10}}
\put(40,20){\line(-1,1){20}}

\put(0,20){\line(1,0){40}} 
\put(0,20){\line(3,-2){30}} 
\put(40,20){\line(-3,-2){30}} 

\put(20,40){\line(1,-4){10}} 
\put(20,40){\line(-1,-4){10}} 
\end{picture}


\item[Двудольный граф] --- граф, множество вершин которого можно разбить на две части таким образом, что каждое ребро графа соединяет какую-то вершину из одной части с какой-то вершиной другой части, то есть не существует ребра, соединяющего две вершины из одной и той же части. Полный двудольный граф обозначается как $K_{a,b}$, где $a,b$ --- мощности соответствующих множеств вершин, на которые граф можно разбить.

\item[Подграф] $ G'(V',E')$ является подграфом для $G(V,E)$ если $ V' \subseteq V \& E' \subseteq E$. 
Обратите внимание, например, здесь на правой картинке представлен подграф, включающий вершину, которая в нём не соединена ни с какой другой!!


\begin{picture}(400,60)
\put(0,0){\circle*{5}}
\put(50,0){\circle*{5}}
\put(0,50){\circle*{5}}

\put(0,0){\line(1,0){50}}
\put(0,0){\line(0,1){50}}
\put(50,0){\line(-1,1){50}}

\put(100,0) {


\begin{picture}(400,60)
\put(0,0){\circle*{5}}
\put(50,0){\circle*{5}}
\put(0,50){\circle*{5}}
 
\put(50,0){\line(-1,1){50}}

\end{picture}

}
\end{picture}

\item[Индуцированный подграф] --- тот , в котором сохранены все рёбра на данном множестве вершин $E'$. Подграф справа --- не индуцированный. 

\item[Остовный подграф] --- подграф, где $E' = E$.

\item[Гомеоморфизм графов.] Графы гомеоморфны когда можно свести один к другому путём удаления вершин и "стягивания" рёбер, например:



\begin{picture}(400,60)

\put(0,0){\line(0,1){50}}
\put(0,0){\line(1,0){50}} 

\put(50,50){\line(0,-1){50}}
\put(50,50){\line(-1,0){50}} 
 
\put(0,0){\circle*{5}}
\put(0,50){\circle*{5}}
\put(50,0){\circle*{5}}
\put(50,50){\circle*{5}}

\put(100,0) {

\begin{picture}(60,60)

\put(0,0){\line(0,1){50}}
\put(0,0){\line(1,0){50}} 

\put(50,50){\line(1,-1){25}}
\put(50,0){\line(1,1){25}}
\put(50,50){\line(-1,0){50}} 
 
\put(0,0){\circle*{5}}
\put(0,50){\circle*{5}}
\put(50,0){\circle*{5}}
\put(50,50){\circle*{5}}
\put(75,25){\circle*{5}}

\end{picture}
}

\end{picture}

\item[Эйлеровость.] Граф называется Эйлеров если существует вершина, от которой можно выстроить цикл Эйлера, который покрывает все рёбра и проходит по каждому из них ровно один раз.
\end{description}

\begin{pantrkur}
Граф планарен тогда и только тогда, когда он не содержит подграфов, гомеоморфных $K_5 $ и $K_{3,3}$.
\end{pantrkur}

\begin{thm}
Связный граф (мультиграф) Эйлеров тогда и только тогда, когда степени всех его вершин чётны.
\end{thm}


\subsection{Подсчёт числа связных графов с данным количеством рёбер}

Пусть $C^k_n$ --- число связных графов с $n$ вершинами и $k$ рёбрами.

$C(n,k) = 0, k < n-2$ --- очевидный факт.

\begin{keli}
Число связных графов с $n$ вершинами и $k$ рёбрами:

\[T_n = n^{n-2}\]

\end{keli}
\begin{proof} (с помощью кодов Прюфера)

Мы докажем теорему если установим биекцию между множеством всех деревьев на $n$ вершинах и всеми последовательностями из $n-2$ элементов, где все элементы от $1$ до $n$ . Таких последовательностей $n^{n-2}$. Опишем процесс составления биекции.
 
Возьмём дерево $D$ на $n$ вершинах. У него есть висячие вершины (степени 1). 

Возьмём минимальную из них по номеру, $b_1$.  Тогда $a_1$ --- другой конец висячего ребра, а ребро $e_1 = ( a_1, b_1 )$. 
Добавляем в код $a_1$  и отрываем от дерева вершину. У нас получается опять дерево, так как связность мы утратить не могли (отрезанная вершина --- висячая). 

$D \backslash e_1 = D_1$

Проделаем аналогичную процедуру. 

$e_2 = ( a_2, b_2 )$. 


Повторяем $n-2$ раз пока не останется одно ребро. Так мы получим последовательность:


 
$a_1 a_2 .. a_{n-2}, \quad a_i \in \{1,2, \dots , n\}$

Мы показали, что разным деревьям соответствуют разные коды. Теперь докажем, что по последовательности можно восстановить дерево.

Дана последовательность:

$ a_1, a_2, \dots , a_{n-2} $

$ 1,2,\dots, n$

Очевидно, что не все натуральные числа от $1$ до $n$ найдут себе соответствие среди элементов последовательности, так как их больше. Возьмём самое малое из чисел, которое не нашло себе пары среди элементов последовательности. Назовём его $b_1$. Составим пару $e_1 = (a_1, b_1)$ и выкинем из последовательности $a_1$, а из чисел --- $b_1$. 
Действуя таким образом мы вычленим $n-2$ "рёбер", а потом из оставшихся двух чисел сформируем последнее "ребро". 



Чтобы "добить" \ биективность, надо доказать, что любому коду соответствует дерево. 
В графе, полученном из любого кода, всегда $n$ вершин и $n-1$ ребро. Осталось доказать по индукции, что он ацикличен, что достаточно просто. 

\end{proof} 

\begin{thm}
Количество лесов на $n$ вершинах с $r$ компонентами, в которых выделены $1, \dots, r$ принадлежности различным компонентам, равно $r n^{n-1-r}$ 
\end{thm}
  

Пусть $C(n,k) $ --- количество связных графов с n вершинами и k рёбрами. На данный момент мы знаем, что:

$C(n,k) = 0, k < n-1$

$C(n,n-1) = n^{n-2}$
 
 Чем будет $C(n,n)$? Мы добавим в дерево одно ребро и, таким образом, получим цикл, причем только один! Соответственно, $U_n = C(n,n)$ --- количество унициклических связных графов на $n$ вершинах.




\begin{thm}  
$ U_n \sim \sqrt{\frac \pi 8} n^{n-^1/_2} $
\end{thm}

\begin{proof}
Рассмотрим унициклический граф $ G $. Он, очевидно, выглядит как цикл, к вершинам которого могут быть прицеплены деревья. Если удалить рёбра цикла, то останется лес (некоторые деревья могут состоять только из одной вершины), причем число компонент связности в нём будет равно величине цикла. 

Мы хотим перебрать все унициклические графы на $n$ вершинах. Пусть $ 3 \leq r \leq n$ --- длина цикла. Тогда 

$C(n,n) = \sum \limits _{r=3} ^n \bigg (C^r_n \frac {(r-1)!} 2  \times r n ^{n-1-r} \bigg ) = \sum \limits _{r=3} ^n \frac {n (n-1) \dots (n-r+1)} {r!} \frac {(r-1)!} 2  r n ^{n-1-r}  $

$C(n,n) = \sum \limits _{r=3} ^n \frac {n (n-1) \dots (n-r+1)}  2   n ^{n-1-r}  $

$C(n,n) = \sum \limits _{r=3} ^n \frac 1 2    n ^{n-1-r} n^r \bigg( 1 - \frac 1 n \bigg)   \bigg( 1 - \frac 2 n \bigg) \dots \bigg( 1 - \frac {r-1} n \bigg) $

\[
U_n  = \frac 1 2 n ^ {n-1} \sum \limits _{r=3} ^n \prod \limits _{j=1} ^{r-1} \bigg ( 1 - \frac j n \bigg)
\] 

Теперь чтобы вывести асимптотику всего выражения нам надо вывести асимптотику суммы. Разобьём сумму на две части:

$\sum \limits _{r=3} ^n \prod \limits _{j=1} ^{r-1} \bigg ( 1 - \frac j n \bigg) =
 \sum \limits _{r=3} ^{\lfloor n^{^5/_9}\rfloor } \prod \limits _{j=1} ^{r-1} \bigg ( 1 - \frac j n \bigg) + 
 \sum \limits _{r = \lfloor n^{^5/_9}\rfloor + 1 } ^ n \prod \limits _{j=1} ^{r-1} \bigg ( 1 - \frac j n \bigg)
$

Отдельно упростим произведение:

$ \prod \limits _{j=1} ^{r-1} \bigg ( 1 - \frac j n \bigg) = \exp \bigg (\sum \limits _{j=1} ^{r-1} \ln \bigg ( 1 - \frac j n \bigg) \bigg ) =  $

Теперь можно представить каждый логарифм в более удобном виде с помощью ряда Тейлора:

$ =  \exp \bigg (\sum \limits _{j=1} ^{r-1} \bigg ( \frac j n + o \bigg(\frac {j^2} {n^2} \bigg) \bigg) \bigg ) 
= \exp \bigg ( - \frac {r (r-1)} {2n} + O\bigg ( \frac {r^3} {n^2} \bigg )\bigg )$

Рассмотрим первую из двух сумм.

$\sum \limits _{r=3} ^{\lfloor n^{^5/_9}\rfloor } \exp \bigg ( - \frac {r^2} {2n} - \frac r {2n} + o\bigg ( \frac {r^3} {n^2} \bigg )\bigg ) $

$r \leq n^{^5/_9} \Rightarrow \frac {r^3} {n^2}  \leq n ^ {\frac {15} {9} - 2}$, что на бесконечности стремится к нулю, как и $ \frac r {2n} $

$\sum \limits _{r=3} ^{\lfloor n^{^5/_9}\rfloor } \exp \bigg ( - \frac {r^2} {2n}  + o(1)\bigg )\bigg ) \sim
\sum \limits _{r=3} ^{\lfloor n^{^5/_9}\rfloor } e^ {- \frac {r^2} {2n} } \sim \int_0^{+\infty} e^{-^{r^2} / _{2n}}\,\mathrm{d}r   $


Интеграл можно взять до бесконечности, так как он сходится. 
Воспользуемся следующим фактом:

$\frac 1 {\sqrt{2 \pi}} \int _{-\infty} ^{+\infty} e^{-\frac {x^2} 2} \, \mathrm{d}x = 1$

Наш интеграл сводится к нему путём замены $x = \frac r {\sqrt{n}}$

$  \int_0^{+\infty} e^{-{x^2} / _2}\sqrt{n}\,\mathrm{d}x = \frac 1 2 \sqrt{2\pi} \sqrt{n} $

Мы получили асимптотику для первой суммы. Проведя  аналогичные рассуждения, оценим вторую, она асимптотически стремится к нулю.

Таким образом ответ:

 $ U_n \sim \sqrt {\frac \pi 8 } n^{n-^1/_2}$
\end{proof} 

 

\section{Обобщение формулы Кэли}

 У формулы Кэли есть естественное обобщение:

Пусть $C(n,n+k)$ - число всех связных графов с $n$ вершинами и $n+k$ рёбрами.

$\displaystyle 
k = -1 \Rightarrow   C(n,n-1) = T_n = n^{n-2} \\
k = 0  \Rightarrow   C(n,n) = U_n \sim \sqrt{\frac \pi 8} n^{n-0.5} 
$

Далее:

$\displaystyle C(n,n+1) \sim \frac 5 {24} n ^{n+1}$


Общий результат датируется 1993 г.  Полное реккурентное соотношение достаточно сложно:

$
C(n,n+k) \sim \gamma _k n^{n+ \frac {3k-1} 2} \bigg(1 + O\bigg(\frac {k^{^3/_2}} {\sqrt n} \bigg )\bigg)
$

$
\gamma_k = \frac {\sqrt \pi 3^k (k-1)}{2^ {\frac {5k-1} 2} \Gamma(k/2)} \delta_k
$

$
\delta_1 = \delta_2 = \frac 5 {36} 
$

$\delta_{k+1} = \delta_k + \sum \limits _{h=1} ^{k-1} \frac {\delta_h \delta_{k-h}} {(k+1)} ( C^h _k)^{-1}, \quad k \geq 2$
\subsection{Число независимости, кликовое и хроматические числа}

Для графа $ G = (V, E) $  :

\begin{description}
\item[Число независимости] $\alpha(G) = \max \{|W| : W \subseteq V, \forall x, y \in W : (x,y) \notin E\}$

\item[Кликовое число] $\omega(G) = \max \{|W| : W \subseteq V, \forall x, y \in W : (x,y) \in E\}$, мощность максимального полного подграфа (клики). 

\item[Хроматическое число] $\chi(G) = \min \{\chi : V = V_1  \sqcup \dots \sqcup V_{\chi}, \forall i \forall x,y \in V_i: (x,y) \notin E  \}$ --- минимальное количество цветов, в которое можно покрасить вершины графа, чтобы концы любого ребра были разного цвета.

Всегда выполняется :
 $\chi(G) \geq \max \{ \omega(G), \frac {|V|} {\alpha(G)} \} $
 
Ведь для любой клики нужно как минимум столько цветов, сколько в ней есть вершин; кроме того в каждой раскраске, дающей хроматическое число, каждый цвет представляет собой независимое множество, и даже если мы раскрасим все независимые множества в разные цвета, потребуется не меньше $  \frac {|V|} {\alpha(G)} \} $ цветов. 
 

Предположим, что $\chi(G) < \frac {|V|} {\alpha(G)}$ . Тогда есть цвет, в который покрашено более, чем $ \frac {|V|} { \frac {|V|} {\alpha(G)}} = \alpha(G) $ вершин. Тогда в этот цвет покрашено как минимум одно ребро полностью. 
\end{description}

\subsection{Гиперграфы}

\begin{description}
\item[Гиперграф] ---обобщённый вид графа, в котором каждым ребром могут соединяться не только две вершины, но и любые подмножества вершин. Формально говоря, это пара из двух множеств $H = (V, E), E \subseteq 2^V$. $H$ называется k-однородным, если $\forall e \in E : |e| = k$. Для $k=2$ получаем обычный граф.

К каждому гиперграфу можно привязать граф пересечений. 

$H=(V,E) \mapsto G(E,F)$, то есть рёбрам гиперграфа сопоставляем вершины графа. Что касается рёбер, $e_1 ,e_2 \in E$ образуют ребро в $F$ если их пересечение непусто.

\end{description}
 
\begin{example}

Рассмотрим гиперграф:

$H = ( \{ 1,2,3,4,5\}, \{ (1,2,3), (2,3), (3,4,5), (1,4,5), (2,5)\}$

Пересечения:

$G(\{v_1, v_2, v_3, v_4, v_5\}, \{(v_1,v_2), (v_1,v_3), (v_1,v_4), (v_1,v_5), (v_2, v_3),(v_2,v_5),(v_3,v_4), \\
(v_3,v_5), (v_4,v_5)\})$

\end{example}


Рассмотрим гиперграф $H = (V, C^k_v) $ --- k-однородный гиперграф, где присутствуют все возможные рёбра, т.е. полный k-однородный гиперграф. Построим граф пересечений $G$ для него. 

Число независимости $\alpha(G) $ --- максимальное количество k-элементных подмножеств множества $V$, которые попарно не пересекаются.
 
 \[ \alpha(G) = \bigg[ \frac n k \bigg] \]


Кликовое число $\omega (G)$ --- максимальное количество k-элементных подмножеств $V$, которые попарно пересекаются.

$\forall k : \omega(G) \geq C^{k-1} _{n-1}$
 
\begin{proof}

\end{proof}

\begin{rem} Перестановки.

Дано множество и операция на нём: $V = \{1,2, \dots, n\} , \sigma : V \mapsto V$. Операция $\sigma$ называется перестановкой, если это биекция множества на себя.

Всего существует $n!$ различных перестановок: $\{\sigma_1, \dots, \sigma_{n!} \} \ni e$
 
 \end{rem}
 
 
\begin{erdesh} 
 
\[ \omega(G) = \begin{cases}
C^k_n, 2k > n\\
C^{k-1}_{n-1}, k \leq \frac n 2 
\end{cases}\]

 
\end{erdesh}
\newcommand{\fset}{\mathbb{F}}
\begin{proof}

Фиксируем один элемент  в множестве V. В нём осталось $n-1$ элементов. Рассматриваем только те k-элементные подмножества, содержащие зафиксированный элемент. Помимо зафиксированного, в них $k-1$, и, естественно, они все пересекаются.
Первая часть теоремы доказана, остался случай 
Докажем второй.

Пусть $\fset = \{ F_1, \dots , F_s\}$ --- совокупность k-элементных подмножеств множества $V=\{1,2,\dots, n\}, 2k \leq n$. Любые два множества из неё пересекаются:

$\forall i,j: F_i \cap F_j \neq \emptyset$

Фактически, это клика в графе пересечений $G$. Нам нужно доказать:
\begin{enumerate}
\item $\exists \fset : s = C_{n-1}^{k-1}$

\item $\forall \fset: s \leq C^{k-1} _{n-1}$
\end{enumerate}

\begin{enumerate}
\item

Возьмём $F = \{ F \subset   V: |F| = k, 1 \in F \} $. Это все подмножества, которые содержат первый элемент , размера k. Разумеется, количество таких подмножеств это количество способов выбрать $k-1$ элементов из $n-1$ оставшихся элементов множества. Все эти подмножества пересекаются хотя бы в первом элементе множества. 

$|F| = C^{k-1} _{n-1}$

\item Доказательство этого пункта за авторством Katona.

Рассмотрим множество $\aset = \{A_1, \dots , A_n\}$, состоящее из множеств $A_i$.

$A_1 = \{1,2, \dots, k\}$

$A_2 = \{2, \dots, k, k+1\}$

$\dots$ , если мы "выскакиваем " \ за границу справа, то " вылезаем"  \ слева.

$A_n = \{n,1,2,\dots, k-1\}$

\begin{alem}[1]
$|F \cap \aset | \leq k$
\end{alem}

\begin{proof}~
\begin{itemize}  

\item 
$\fset \cap \aset = \emptyset \Rightarrow$ очевидно.

\item

$\fset \cap \aset \neq \emptyset \Rightarrow \exists s: A_s \in \fset  $

$\Rightarrow \forall F \in \fset : F \cap A_s \neq \emptyset$

$\Rightarrow \forall A_i \in (\fset \cap \aset) : A_i \cap A_s \neq \emptyset$

\end{itemize}
\end{proof}
 
Пусть $A_s = A_1$.

$
\begin{matrix}
1 & 2 & ~ & k & k+1 & ~ & n \\ 
 * & * & \dots & * & * & \dots & * 
\end{matrix}
$ \quad Здесь первые k элементов $\in A_s$

Перечислим все $A_i \in \aset$, которые пересекаются с $A_s$:

$A_1, A_2, A_3, \dots, A_k; A_n, A_{n-1},\dots, A_{n-k+2}$

Так как $k \leq \frac n 2$, это корректно. Крайние элементы:

$(A_2, A_{n-k+2})$

$(A_3, A_{n-k+3})$

$(A_4, A_{n-k+4})$

\dots

$(A_k, A_{n-k})$

Мы видим, что какое  бы   множество $A_s$ мы ни взяли из $\aset$, мы не сможем взять больше $k-1$ других множеств. Значит, мощность пересечения действительно не больше $k$:

$|F \cap \aset | \leq k$  , что и требовалось доказать.

Применим перестановку $\sigma_i $ к $V$. Поменяются и множества $A_1, \dots, A_n$, т.е. совокупность $\aset \mapsto \aset _{\sigma_i}$. В частности, $\aset_e = \aset$. 

\begin{alem}[2]
$\forall i: |\fset \cap A_{\sigma_i}| \leq k$
\end{alem}
 
Введём функцию-индикатор:

$I(F, A_{\sigma_i} )=
\begin{cases}
1, F \in A_{\sigma_i}\\
0
\end{cases}
$

Вычислим следующую сумму двумя различными способами:

$\sum \limits _{i=1} ^{n!} \sum \limits _{F \in \fset} I(F, A_{\sigma_i} )\leq k n!$ (внутренняя сумма это мощность пересечения из условия леммы)

С другой стороны, фиксируем произвольное $F \in \fset$ и считаем, сколько есть различных перестановок $\sigma_i$, для которых $F \in A_{\sigma_i}$. Найденные количества суммируем по всем $F \in \fset$

Сколько различных перестановок отправляют $F$ в какое-то фиксированное множество?

$\forall i : F \xrightarrow{\sigma} A_i \{1,2, \dots, k\}$

Всего таких $ \sigma $ --- $k! (n-k)!$. Следовательно, существует

$n k! (n-k)!$ перестановок, которые делают из $F$ в $\aset = A_e$.

Обозначим эти перестановки:

$\sigma_{i_1}, \dots, \sigma_{i_{nk!(n-k)!}}$

Возьмём обратные перестановки:

$\sigma_{i_1}^{-1}, \dots, \sigma_{i_{nk!(n-k)!}} ^{-1}$

Тогда:

$\forall \nu = \overline{1,2,\dots, nk!(n-k)!} :F \in A_{\sigma_{i_\nu}^{-1}} $

$\sum \limits _{i=1} ^{n!} \sum \limits _{F \in \fset} I(F, A_{\sigma_i} )= |\fset| nk! (n-k)! $

$|\fset| \leq \frac {kn!} {nk!(n-k)!} = C_{n-1}^{k-1}$
 
\end{enumerate}

\end{proof}


\chapter{Лекция 5. 6 октября 2011}


\section{Разбиение чисел на слагаемые}

$n \in N, \quad  n = x_1 + x_2 + \dots + x_t$

Сколькими способами можно разбить $n$ на слагаемые? Чтобы ответить на этот вопрос, нужно как минимум наложить ограничения на $t$ и на $x$.
 

$x_i \in X, x_i \in \naturalset^{+}$

Нам интересны два случая:
\begin{itemize}
\item Порядок слагаемых имеет значение. 
\begin{example}
Например, нам надо выпить. Необходимо набрать 500 г алкоголя, есть пиво, вино, водка и т.п. У нас есть рюмка, в которую мы можем наливать разные напитки и каждый раз в ней может быть разное количество чистого алкоголя. Порядок нам важен по понятным причинам.
\end{example}
 
\item Порядок слагаемых не имеет значения
\begin{example}
Есть капуста, она весит 5 кг. На одной чаше весов --- капуста, на другой --- гири, надо набрать их и взвесить капусту.
Сколькими способами можно выбрать гири так, чтобы уравновесить капусту?
\end{example}
 
\end{itemize}

 \subsection {Порядок слагаемых важен}
Пусть надо разбить число $n = x_1 + x_2 + \dots + x_t$ на слагаемые, порядок важен. При этом все слагаемые являются элементами множества:

$\forall i : x_i \in \{n_1, \dots, n_k\} $

Примем за $f(n; n_1, \dots, n_k)$  число разбиений $n$ в этой ситуации, когда есть ограничение на параметры. 

\begin{thm}
\[ 
f(n; n_1, \dots, n_k) = f(n-n_1; n_1, \dots, n_k) + f(n-n_2; n_1, \dots, n_k) + \dots + f(n-n_k; n_1, \dots, n_k)
\]

Это рекуррентное соотношение. При этом начальное условие:

$n < 0 \Rightarrow f(n; n_1, \dots, n_k) = 0$

$f(0; n_1, \dots, n_k) = 1$
\end{thm}

\begin{cor}
$f(n, 1,2,\dots, n) = 2^{n-1}$
\end{cor}
\begin{proof} По индукции:

$f(n, 1,2,\dots, n) = \phi (n) = \\  = f(n-1, 1,2,\dots, n-1) +  f(n-2, 1,2,\dots, n-2) + \dots + f(1;1) + f(0; 1, \dots n) = \phi(n-1) + \phi(n-2) + \dots + \phi(1) + 1 = \\
= 2^{n-2} + 2^{n-3} + \dots + 2^0 + 1 = 2^{n-1} -1 + 1 = 2^{n-1}$
\end{proof}

\subsection{Порядок слагаемых неважен}

Аналогично, обозначим за  $F(n; n_1, \dots, n_k)$  число разбиений $n$. Нельзя действовать полностью аналогичным образом в этом случае, потому что нет однозначно определенного понятия первого слагаемого.

\begin{thm}
\[F(n; n_1, \dots, n_k ) = F(n-n_1; n1,\dots, n_k) + F(n; n_2 \dots, n_k)\]

Аналогично, обозначим 
$p(n) = F(n; 1,2,\dots, n)$


\end{thm}
Однако оценить $p(n) $ непросто. Полный вывод есть в книге Эндрюса.


\begin{harram}
\[\displaystyle p(n) \sim \frac 1 {4n \sqrt 3} e^{\pi \sqrt{\frac 2 3} \sqrt n}\]
\end{harram}
 
Для работы с неупорядочными разбиениями используется диаграммная техника.

\subsection{Диаграммная техника --- Диаграммы Юнга}

 Пусть есть натуральное число $n = x_1 + \dots + x_t, \quad x_1 \leq x_2 \leq \dots \leq x_t$. Порядок имеет значение. Будем считать, что слагаемые упорядочены в порядке неубывания.
 
$x_1 \leq x_2 \leq \dots \leq x_t$

Нарисуем следующую диаграмму. Рядом с $x_i$ соответствующее его значению количество точек. Всего точек $n$ штук.
  
\begin{picture}(10,60)

\put(0,30){$x_1$}
\multiput(20,32)(10,0){5}{\circle*{4}}

\put(0,20){$x_2$}
\multiput(20,22)(10,0){7}{\circle*{4}}

\put(0,10){$\dots$}

\put(0,0){$x_t$}
\multiput(20,2)(10,0){10}{\circle*{4}}

\end{picture}

Так мы однозначно закодировали одно из разбиений. Такие диаграммы (называемые диаграммы Юнга) удобны для некоторых доказательств. 

\begin{thm}[1]
Число неупорядоченных разбиений числа $n$ на не более $k$ слагаемых равно числу разбиений числа $n+k$ на $k$ слагаемых. 
\end{thm}

\begin{proof}
Рассмотрим диаграмму произвольного разбиения n на $\leq k$ слагаемых. В ней не более $k$ строчек. Пририсуем слева к диаграмме столбец из ровно $k$ точек. Теперь в ней всего $n+k$ точек,  а строк ровно $k$. Так мы получили диаграмму разбиения  $n+k$ на точно $k$ слагаемых.
Таким образом получается биекция между двумя "видами" диаграмм.
\end{proof}

\begin{thm}[2]
Число неупорядоченных разбиений числа n на не более k слагаемых равно числу разбиений числа $n+\frac {k(k+1)} 2 $ на ровно $k$ различных слагаемых. 
\end{thm}

\begin{proof}
Добавим слева треугольничек.  Он гарантированно даёт разные величины слагаемых при прибавке. К общему числу точек он добавит как раз $\frac {k(k+1)} 2$.
\end{proof}

\begin{rem}
Диаграммы можно также транспонировать, получая двойственные им.
\end{rem}


\begin{thm}[3]
Число разбиений числа n на не более k слагаемых,равно числу разбиений n на слагаемые, величина каждого из которых не больше k.
\end{thm}


\section{Бесконечное формальное произведение}

Мы просто делаем преоборазования по привычным правилам.

$ \begin{aligned}
&(1-x) (1-x^2) \dots (1-x^n) \dots  = \\
&(1 - x - x^2 + x^3) (1-x^3) \dots (1-x^n) \dots = \\
&(1 - x - x^2 + x^4 \dots = \\
&1 - x - x^2 + x^5 + x^7 - x^{12} - x^{15} + \dots 
\end{aligned}
$


\begin{thm}

Если $n = \frac {3k^2 \pm k} 2,  k \in \naturalset $ то коэффициент при $ x^n $  равен  $(-1)^k$ иначе  0.

\end{thm}


\begin{proof}
Считаем коэффициент при $ x^n $ 

$
(-x ^{n_1} )(-x ^{n_2} ) \dots  (-x ^{n_t} ) = (-1)^t x^{n_i + \dots + n_t}
$

То есть, $x^n$ возникает только тогда, когда число $n$ можно разбить на слагаемые $n_1 \dots n_t$, при этом знак при нём зависит от того, чётно ли количество этих слагаемых.

Тогда коэффициент при $x_n$ можно записать так:

\[\sum \limits ^x _{n_1,\dots, n_t: \\ n = n_1 + \dots + n_t} (-1)^t = |n_{\text{чёт}} - n_{\text{нечёт}}|
\]
 
 Здесь $n_{\text{чёт}}$ --- количество разбиений числа $n$ на чётное число слагаемых, $n_{\text{нечёт}}$ --- количество разбиений числа $n$ на нечётное число слагаемых.
 

\end{proof}

\begin{cor}
Из доказательства явно следует факт:

$
|n_{\text{чёт}} - n_{\text{нечёт}}|= \begin{cases}
0, n \neq \frac {3k^2 \pm k} 2, k \in \naturalset \\
(-1)^k, \text{в противном случае}
\end{cases}
$
\end{cor}
\section {Формальные степенные ряды}

Назовём формальным степенным рядом следующую картинку :
$a_0 + a_1 x + \dots + a_n x^n + \dots$

\subsection{Операции}

Рассмотрим два формальных степенных ряда:

$A = a_0 + a_1 x + \dots + a_n x^n + \dots $

$B = b_0 + b_1 x + \dots + b_n x^n + \dots $

\begin{enumerate}
\item Сложение

$C = c_0 + c_1 x + \dots + c_n x^n + \dots , \quad \forall i : c_i = a_i + b_i$

\item Произведение

$C = c_0 + c_1 x + \dots + c_n x^n + \dots , \quad \forall i : c_i = a_i b_i + a_1 b_{i-1} + \dots + a_{i-1} b_1 + a_i b_0$

\item Деление

Формируется ряд $C = A / B$ такой, что $ CB = A $.


$\begin{cases}
c_0b_0 = a_0\\
c_0b_1 + c_1b_0 = a_1\\
c_0b_2 + c_1b_1 + c_2b_0 = a_2\\
\dots
\end{cases}
$

Общая формула для $i$-ого члена такая:

$C_i = - \frac 1 {a_0} \sum \limits _{i=1} ^n a_i b_{n-i}, \quad n \geq 1$


Например, $ \frac 1 {(1-x)} = 1 + x + x^2 + \dots $.

Заметим, что мы получили формулу суммы геометрической прогрессии! Конечно, если рассматривать ряд справа аналитически, необходимо ограничение $|x| <1 $ и т.п.

\begin{example}

$
\frac 1 {(1-x^2)^2} = \frac 1 {1 - 2x^2 + x^4} = 
\begin{cases}
\frac 1 {(1-x)^2} \frac 1 {(1+x)^2} \\
\bigg ( \frac 1 {1-x^2} \bigg ) ^2
\end{cases}
$ Одно и то же!

Теперь подробнее о первом случае.

$\frac 1 {1-x} = 1 + x + x^2 + \dots $

$ \frac 1 {1+x} = 1 - x + x^2 - \dots $

$ \bigg ( \frac 1 {1-x} \bigg ) ^2 = 1 + 2x + 3x^2 + \dots + (n+1)x^n + \dots $

$ \bigg ( \frac 1 {1+x} \bigg ) ^2 = 1 - 2x + 3x^2 - \dots + (-1)^n (n+1)x^n + \dots $

$ \bigg ( \frac 1 {1-x} \bigg ) ^2 \bigg ( \frac 1 {1+x} \bigg )^2 = \dots (1 - (-1)^n (n+1) + 2(-1)^{n-1} n + \dots + (n+1) 1) x^n \dots $

Обозначим коэффициент за $C_n$. Теперь покажем, что полученный коэффициент тождественно равен коэффициенту в другом случае.

$\frac 1 {1-x^2} = 1 + x^2 + x^4 + \dots + x^{2k} + \dots $

$\bigg( \frac 1 {1-x^2}\bigg)^2 = 1 + 2x^2 + 3 x^4 + \dots + (k+1) x^{2k} + \dots $

Отсюда имеем коэффициент $C'_n = 
	\begin{cases}
	 0, n = 2k+1 \\
	 (k+1), n = 2k
	 \end{cases} 
	 $

Очевидно, что это то же самое, что и в первом случае.
\end{example}

\end{enumerate}

 
\begin{rem} Напоминание об обычных рядах.

Представим, что формальный ряд стал рядом обычным, аналитическим. Запишем его, как

$\sum \limits _{k=0} ^\infty a_k x^k $

Определим $\rho = \frac 1 {\displaystyle \overline{\lim}_{k \to \infty}  \sqrt[k] {|a_k|} } $


Тогда ряд сходится при всех $ |x| < \rho $

 
Рассмотрим некоторые примеры, чтобы уловить суть формулы.

$\sum \limits _{k=0} ^\infty \underbrace{ 2^k } _{a_k} x^k $

Легко увидеть, что этот ряд сходится, когда $|x |< \frac 1 2$, потому что тогда он становится геометрической прогрессией со знаменателем, меньшим 1. Но зачем  верхний предел?

$\sum \limits _{k=0} ^\infty, a_k = 
\begin{cases}
 3^k, k = 2l \\
 2^k, k = 2l + 1
 \end{cases}
$

Тут обычного предела не существует, верхний же имеется. Этот ряд сходится на $ |x| < \frac 1 3$.

\end{rem} 


\newpage

\begin{itemize}
\item 
\( (x+y)^n=\sum \limits_{k=0}^n C^k_n x^k y^{n-k}  \)

\item 
$ C ^k _n = C^{n-k}_n $

\item 
$ C^ k _n = C^k _{n-1} + C^{k-1} _{n-1}$

\item 
$C^0_n + C^1_n + \dots + C^n_n + \sum \limits _{i=0} ^n {C^ i _n} = 2^n$

\item $ \sum  _{\substack{
     (n_1, ..., n_k): \\   
      n_i \geq 0, \sum n_i = n }}
    P(n_1, ..., n_k) = k^n$ , $P(n)$ --- полиномиальные коэффициенты
	
	
\item $\sum \limits _{i=0} ^n {(C^ i _n )^2 } = C^n _{2 n}$

\item Степени сумм натуральных чисел

\(
\begin{aligned}
C^m_{n+m} = &C^{n-1}_{n+m-1} + C^{n-1}_{n+m-2} + \dots 
\end{aligned}
\) 

\item
$C_n^0 < C_n^1 < \dots < C_n^{\lfloor  n/2\rfloor} \geq C_n^{\lfloor n/2\rfloor + 1} $

\item
$C^n_{2n} < 4^n$

\item
\( n! \sim \sqrt{2 \pi n} \bigg (\frac n e \bigg) ^n  \)

\item

$  C^n_{2n} \sim  \frac {4^n} {\sqrt{\pi n}} $


\item
$\displaystyle C^k_n  \leq \frac {n^k} {k!} e^{- \frac {k(k-1)} {2n} } $


\item
Если $k = o(\sqrt{n}) $, то $\displaystyle C^k_n \sim \frac {n^k} {k!}$

\item 
$\frac {C^{^n /_2-x}_n} {C^{^n /_2}_n} =  e^ {\displaystyle - \frac {2x^2} n +  O\bigg(\frac {x^3} {n^2}\bigg)} $




\item 
Если $x = o(\sqrt{n}) $, то $\displaystyle C^{\displaystyle^n /_2-x}\sim C^{\displaystyle^n /_2}$


\item 
$P \sim \pm e ^ {o(n)} \Rightarrow  P(n)(C + o(1) )^n = (c + o(1) ) ^n $

\item

Пусть $a > 1, a \in \realset$. Тогда $ C ^n _ {[an]} = \bigg ( \frac {a ^a} {(a-1)^{a-1}} + o(1)\bigg)^n$

\item 
Пусть $a \in \realset, b \leq \frac a 2$. Тогда:

$\sum \limits _{k=0} ^{[b_n]} C^k_{[an]} = \bigg( \frac  {a^a} {b^b(a-b)^{a-b}} + o(1)\bigg ) ^n $
\end{itemize}


\end{document}
