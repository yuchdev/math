\section{Основы комбинаторики.}

\begin{description}
\item[Рекомендуемая литература]~ 

1. Виленкин. "Комбинаторика" (переиздание 2006 г.) \\
2. Риордан. "Введение в комбинаторный анализ" \\
3. Грэхем, Кнут. Паташник. "Конкретная математика" \\
\end{description}

\subsection{Основные правила}

\begin{description}
\item[Задача комбинаторики]~	

Задача комбинаторики - пересчет количества элементов (сочетаний, перестановок, размещений и перечислений элементов) и отношений на них.

\item[Правило суммы]  ~

Если элемент из множества $A$ можно выбрать $n$ способами, а элемент из множества $B$ можно выбрать m способами, то выбрать элемент либо из $A$, либо или из $B$ можно сделать $n+m$ способами.

\item[Правило умножения]  ~

Если мы берём сначала элемент из $A$, затем элемент из $B$, то количество способов выбрать упорядоченную пару $m \times n $ способов.

TODO Картинка декартова произведения

\item[Правило биекции]  ~

Если есть два конечных множества $\mathcal{A}$ и $\mathcal{B}$, и имеется биекция, то в множествах одинаковое количество элементов. 
\end{description}

\subsection{Примеры выборок}

Пусть в чемпионате по футболу участвует 16 команд, и первые три места могут быть заняты любыми 3 командами. \\
Очевидно, первое место может занять одна из 16 команд, второе - одна из оставшияхся 15, третье - одна из оставшихся 14. \\
По правилу произведения количество различных выборок $16 \cdot 15 \cdot 14 = 3360$ \\

Модифицируем конструкцию для демострации правила суммы. Пусть сначала 16 команд играют групповые турниры в группах по 4, и из каждой группы выходит 8 команд, и из них уже выбираются первые три места.

TODO картинка

Здесь будет меньше вариантов, т.к. все три команды победителей не могут оказаться из одной группы.
Поэтому чтобы посчитать общее количество, необходимо из уже полученного количества выборок вычесть все запрещенные варианты.

Посчитать их можно следующим образом: из первой группы не могут выйти в победители три команды $4 \cdot 3 \cdot 2$, также из второй, третьей и четвертой. Т.е. ответ:

$16 \cdot 15 \cdot 14 - 4 \cdot 4 \cdot 3 \cdot 2 = 3264$

\subsection{Комбинаторные определения}

\begin{description}
\item[Перестановки]~	

Перестановка --- один из способов переставить элементы множества в различном порядке. Другими словами, это биекция множества на себя, или же кортеж (упорядоченное множество из различных чисел от 1 до n).

Первый элемент может быть любым (n вариантов). на второй очевидно остается (n-1 вариантов) и т.д. до последнего.

Из вышесказанного очевидно, что количество перестановок выражается, как

$$
n(n-1)(n-2) \ldots 3 \cdot 2 \cdot 1 = n!
$$

\end{description}

\begin{description}
\item[Количество всех подмножеств данного множества]~	

TODO картинка

Т.е. на каждый элемент имеется 2 варианта - он либо есть в подмножестве, либо нет.
Про второй, третий и т.д. изветсно то же самое.

Перемножим все варианты
$2 \cdot 2 \cdot 2 \cdot 2 \cdot 2 \cdot 2 = 2^n$

Кстати, это объясняет, почему множество всех подмножеств принято обозначать $2^X$

\end{description}


\begin{description}
\item[Размещения]~	

Пусть у нас имеется n элементов, 
${1,2, \ldots , n}$
и мы выбираем из них кортеж, состоящий только из k различных элементов этого множества.
Такая выборка называется размещением. Посчитаем их число. Очевидно, таким же образом, как и с перестановками, просто эту запись нужно в какой-то момент оборвать:
$$
n(n-1)(n-2) \ldots (n - k + 1) = \frac{n!}{(n-k)!}
$$

Обозначается размещение следующим образом:
$$
A_n^k = \frac{n!}{(n-k)!}
$$
\end{description}

\begin{description}
\item[Сочетания]~	

Аналогично, пусть у нас имеется n элементов, и мы выбираем из них некое k-элементное неупорядоченное подмножество.

$${1,2, \ldots , n}$$

Такая выборка называется сочетанием. Посчитать их число также несложно. 
Возьмем k-элементное подмножество с различными элементами.
Тогда мы можем сделать из него кортеж $k!$ способами, просто переставляя элементы.


$\{a_1, a_2, \cdots ,a_k\} \leadsto$  кортеж $k!$ способами 

Это количество не зависит от того, какое конкретно мы взяли подмножествою
Поэтому чтобы узнать количество неупорядоченных подмножеств, 
количество соответствующих k-элементных кортежей необходимо поделить на k!


Обозначается размещение следующим образом:
$$
C_n^k = \frac{n!}{(n-k)!k!}
$$
Также этот объект также называется биномиальным коэффициентом.

\end{description}

\begin{description}
\item[Размещения с повторениями]~	

Во всех прошлых примерах мы предполагали, что количество элементов различно.
Но это не всегда так.
Возьмем некоторое n-элементное множество и составим k-элементый кортеж из его элементов, причем любой элемент используется любое число раз.\\
${a_1, a_2, \cdots ,a_n}$\\
Очевидно, первый элемент выбирается n способами, второй - n способами, и так вплоть до k-го.
Поэтому число размещений с повторениями определяется так:
$$
\bar{A}_n^k = n^k
$$

\end{description}


\begin{description}
\item[Сочетания с повторениями]~	

Сочетания с повторениями является гораздо более интересным комбинаторным объектом.

Возьмем некоторое n-элементное множество и составим k-элементное неупорядоченное подмножество из его элементов, причем любой элемент используется любое число раз.\\
${a_1, a_2, \ldots ,a_n}$\\

Обратите внимание, что k может быть больше n, т.к. любой элемент мы можем брать любое число раз.
Подсчитывается исходя из комбинаторных соображений

%TODO полоска
TODO полоска

В множестве $n+k-1$ клетка. Покрасим в этой полоске $n-1$ клетку (что также равно k клеток).
Количество клеток от начала множества - это количество единиц, количество от первой закрашенной до второй - количество двоек, троек в множестве нет (нет промежутка между 2 и 3 закрашенными клетками), соответсвенно, 2 четверки и 1 пятерка.
Между выбором элементов и подобной картинкой существует взаимно однозначное соответствие.
Нам нужно просто выбрать $k$ клеток из $n+k-1$ (количество незакрашенных элементов)
Объект обозначается следующим образом:
$$
\bar{C}_n^k = C_{n+k-1}^{n-1} = C_{n+k-1}^k
$$
\end{description}

\section{Свойства биномиальных коэффициентов.}

\begin{description}
\item[Свойство 1]~	
$$
C_n^k = C_n^{n-k}
$$
Доказательство тождества очевидно из чисто комбинаторных соображений - если мы выбираем из множества некое подмножества $ n $ способами, то мы таким же количеством способов выкидываем оставшееся.
\end{description}


\begin{description}
\item[Свойство 2]~	
$$
C_n^k = C_{n-1}^{k-1} + C_{n-1}^k
$$


TODO картинки

Берем множество из n элементов. В левой части тождества написано количество способов выбрать k-элементное подмножество. Посчитаем его по-другому.

Выделим одну клетку. Она либо попала в это множество, либо не попала. \\
Посчитаем количество способов, когда эта клетка в подмножество не попала. \\
Т.е. имеем $ n-1 $ элемент. Из них нужно выбрать $ k $. \\
Это количество $C_{n-1}^k$ \\
\\
Теперь посчитаем количество способов выбрать k-элементное подмножество с учетом этой клетки. \\
Тогда мы должны выбрать ее, а из оставшихся клеток выбрать еще $ k-1 $.
Это количество $C_{n-1}^{k-1}$ - кол-во способов выбрать k-элементное подмножество, содержащее выделенную клетку. \\
Значит общее количество способов - сумма двух приведенных количеств сочетаний.

Вообще, комбинаторные тождества часто доказываются путем подсчетов одних и тех же сочетаний разными способами.

\end{description}


\begin{description}
\item[Свойство 3]~	

Сумма всех биномиальных коэффициентов равна 2 в степени n
$$
C_n^0 + C_n^1 + C_n^2 + \ldots + C_n^n = 2^n  
$$
Или в сокращенной форме: 
$$
\sum\limits_{i = 0}^{n} C_n^i = 2^n
$$

В правой части тождества - количество всех подмножеств множества из $ n $ элементов.
Мы его уже посчитали, оно равно $2^n$.

 \end{description}


\begin{description}
\item[Свойство 4]~	

$$
k C_n^k  = n C_{n-1}^{k-1}
$$

Для доказательства подобных тождеств лучше всего придумать комбинаторную трактовку одной из частей.

TODO массив

Рассмотрим левую часть тождества.
Пусть $C_n^k$ - выбор $k$-элементного подмножества.
Тогда умножение на $ k $ - выбор еще одного элемента из этого подмножества 
(для наглядности закрасим этот элемент)

Теперь посчитаем это выражение другим способом.
Сначала выберем закрашенный элемент. Это можно сделать n способами.
Теперь у нас остался $ n-1 $ элемент, и из них нужно выбрать $ k-1 $ элемент.

Это можно сделать $C_{n-1}^{k-1}$ способами.
Перемножив их, получае в точности правую часть равенства.

\end{description}


\begin{description}
\item[Свойство 5]~	

Предыдущее свойство является частным случаем этого.

$$
C_n^k C_{n-k}^{m-k} = C_m^k C_n^m
$$

Доказательство аналогично предыдущему.
%TODO массив
TODO массив

Рассмотрим правую часть тождества и правый множитель.
%TODO выделить
Выбираем из n-элементного множества m-элементное подмножество.
А потом берем выбранное m-элементное подмножество и закрашиваем в нем k элементов.
Количество способов сделать это $C_m^k$.
Тогда общее количество выборок равно правой части тождества, в котором покрашено $ k $ элементов.

Теперь посчитаем это выражение другим способом (левая часть тождества).
Сначала выберем в $n$-элементном множестве $ k $ элементов и их закрасим ($C_n^k$). \\
У нас осталось $n-k$-элементное множество, 
в котором нужно выбрать $ n-k $ элементов ($C_{m-k}^{n-k}$).
Перемножив их, получае в точности левую часть равенства.

\end{description}


\begin{description}
\item[Свойство 6]~	

Пусть имеется число сочетаний  с повторениями с повторениями, составленные из элементов $n+1$ вида: 
$$
\bar{C}_{n+1}^{k} = C_{n+k}^{k}
$$

Само множество:
$$
{1,2,\ldots,n,n + 1}
$$
Берем первый элемент $ i $ раз.
Осталось выбрать $ k-i $ элементов из множества ${2,\ldots,n,n+1}$

Количество способов сделать это $\bar{C}_{n}^{k-i} $
Поскольку $i$ может быть любым числом от 0 до $k$ получаем соотношение:
$$
\sum\limits_{i = 0}^{k} \bar{C}_{n}^{k-i} = \bar{C}_{n+1}^{k}
$$

Запишем обе части доказанного равенства через биномиальные коэффициенты.
$$
\sum\limits_{i = 0}^{k} C_{n+k-i-1}^{k-i} = C_{n+k}^{k} 
$$

Распишем в виде обычной суммы:
$$
C_{n+k-1}^{k} + C_{n+k-2}^{k-1} + C_{n+k-3}^{k-2} + \ldots + C_n^1 + C_{n-1}^0 = C_{n+k}^k
$$

Для удобства записи можно заменить $n$ на $n+1$
$$
C_{n+k}^{k} + C_{n+k-1}^{k-1} + \ldots + C_{n+1}^1 + C_{n}^0 = C_{n+k+1}^k
$$

А если еще воспользоваться Свойством 1 ( $C_n^k = C_n^{n-k}$ ),
то формула приобретает вид:
$$
C_{n}^{n} + C_{n+1}^{n} + \ldots + C_{n+k}^{n} = C_{n+k+1}^{n+1}
$$

Рассмотрим частные случаи:
$n=1$
$$
1 + 2 + \ldots + (k+1) = C_{k+2}^2 = \frac{(k+2)(k+1)}{2}
$$

$n=2$
$$
\frac{1\cdot2}{2} + \frac{2\cdot3}{2} + \ldots + \frac{(k+2)(k+1)}{2} = C_{k+3}^3 = \frac{(k+3)(k+2)(k+1)}{6}
$$

Если скомбинировать эти формулы, можно получить формулу для суммы квадратов (надо из второй умноженной на 2 вычесть первую).



 \end{description}

\begin{description}
\item[Свойство 7. Свертка Вандермонда]~	

Возьмем все то же количество сочетаний с повторениями.

$$
\bar{C}_n^k = C_{n+k-1}^k
$$
Очевидно, в данное сочетание входит не более k различных элементов
Разобьем это сочетание на i классов таким образом, что в i-й класс входит 
ровно i различных элементов.

Теперь посчитаем способов выбрать из одного конкретного i-го мнжества 
k-элементное множество Cс повторениями. 
Сначала положим в множество все эти $i$ элементов по одному разу, а оставшиеся элементы (их $k-i$ штук) могут быть любыми элементами из этих $i$, т.е. их в точности 
$$
\bar{C}_i^{k-i} = C_{k-1}^{k-i} = C_{k-1}^{i-1}
$$

% TODO непонятный момент
$$
\bar{C}_i^{k-i} = C_{i+k-1}^k
$$
А количество способов выбрать это i-е подмножество из n элементов - просто количество сочетаний. Таким образом, общее количество сочетаний с повторениями, в которые входит 
ровно $ i $ различных элементов:
$$
C_n^i C_{k-1}^{i-1}
$$

Теперь суммируем по всем классам:
$$
\sum\limits_{i=1}^k C_n^i C_{k-1}^{i-1} = \bar{C}_n^k = C_{n+k-1}^k
$$
$$
\sum\limits_{i = 1}^{k}  C_{k-1}^{k-i} C_n^i = C_{n+k-1}^k 
$$

В результате получаем
$$
C_{n+k-1}^k = \sum_{i = 1}^k C_{k-1}^{i-1} C_n^i = C_{k-1}^0 C_n^1 + C_{k-1}^1 C_n^2 + C_{k-1}^k C_n^3 + \ldots + C_{k-1}^{k-1} C_n^k
$$

\end{description}

\section{Треугольник Паскаля.}

TODO 

\section{Бином Ньютона.}

TODO 