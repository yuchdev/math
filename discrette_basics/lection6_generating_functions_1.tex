\section{Производящие функции}

\begin{description}
\item[Рекомендуемая литература]~ 

1. Ландо "Лекции о производящих функциях" \\
2. Грэхем, Кнут. Паташник. "Конкретная математика" \\
\end{description}

\subsection{Основные понятия}

Пусть имеется некая последовательность чисел:
$$
a_0, a_1, a_2, 3 a_3, \ldots
$$

Каждой такой последовательности формально можно сопоставить степенной ряд
$$
A(t) = a_0 + a_1 t + a_2 t^2 + a_3 t^3 + \ldots = \sum\limits_{n=0}^{\infty} a_{n} t^{n}
$$

Ряд не обладает никаким набором специфических свойств, он не обязан сходится и т.п.
Единственное обязательное свойство:
$$
A(0) = a_{0}
$$

Тогда объект $ A(t) $ называется \textbf{производящей функцией}.

Если заданы несклько производящих функицй, можно определить операции сложения и умножения
$$
A(t) = a_0 + a_2 t^2 + a_3 t^3 + \ldots
$$
$$
B(t) = b_0 + b_2 t^2 + b_3 t^3 + \ldots
$$

Сумма:
$$
A(t) + B(t) = (a_0+b_0) + (a_1+b_1) t + (a_2+b_2) t^2 + (a_3+b_3) t^3 + \ldots
$$
Произведение производящих функций является производящей функцией свёртки:
$$
A(t) B(t) = c_0 + c_2 t^2 + c_3 t^3 + \ldots = \sum_{n=0}^\infty c_n t^n
$$

где
$c_0 = a_0 b_0$	\\
$c_1 = a_0 b_1 + a_1 b_0 $ \\
$c_2 = a_0 b_2 + a_1 b_1 + a_2 b_0  $\\
$c_3 = a_0 b_3 + a_1 b_2 + a_2 b_1 + a_0 b_3$\\
$\ldots$\\

В целом:
$$
c_n = \sum_{k=0}^n a_k b_{n-k}
$$

Если мы раскроем скобки, то получится группировка степеней при соотетствующих t.

В некоторых случаях можно говорить о композиции производящих функций:
$$
A(B(t)) = a_0 + a_1 B(t) + a_2 B(t)^{2} + a_3 B(t)^{3} + \ldots =
$$
группируем по степеням $ t $
$= a_{0} + a_{1}b_{1}t $ \\
$+ (a_{1}b_{2} + a_{2}b_{1}^{2} ) t^{2} + $
$+ (a_{1}b_{3} + a_{2}2 b_{1} b_{2} + a_{3}b_{1}^{3}) t^{3} + $ \\
$\ldots$
Необходимое условие композиции: $B(0) = 0$

\textbf{Пример:}
$$
B(t) = -t; A(B(t)) = a_{0} - a_{1}t + a_{2} t^{2} - a_{3} t^{3} + \ldots
$$
$$
B(t) = t^{2}; A(B(t)) = a_{0} + a_{1} t^{2} + a_{2} t^{4} + \ldots
$$

\subsection{Теорема}
$\exists! B(t) : b_{0} = 0, b_{1} \neq 0$

Тогда $\nexists A(t), C(t) :  A(B(t)) = t, B(C(t)) = t$

Т.е. найдуться такие $ A(t), C(t) $, что их композиции равны $ t $

В некоторых случаях $ A(t) $ и $ C(t) $ могут совпадать.

Исходим из соображений, что B(t) известно, ищем $ A $ и $ C $ таким образом, чтобы 
их композиция с $ B $ оказалась равна $ t $.
Для этого достаточно, чтобы ненулевой коэффициент при $ t \neq 0$, а при всех остальных коэффициентах он должен обращаться в 0.

Попробуем подобрать прямо из формулы композиции:

$A(B(t))= a_{0} + a_{1}b_{1}t $ 
$+ (a_{1}b_{2} + a_{2}b_{1}^{2} ) t^{2} $
$+ (a_{1}b_{3} + a_{2}2 b_{1} b_{2} + a_{3}b_{1}^{3}) t^{3} + $ \\
$\ldots$

Сразу подбираем $ a_{0} = 0 $ 

Если $ b_{1} \neq 0 $, то $ a_{1} $ найдется: $a_{1} = \frac{1}{b_{1}}$

Второе выражение: $a_{1}b_{2} + a_{2}b_{1}^{2} = 0$. 

$ a_{1}, b_{1}, b_{2} $ известны,
$ a_{2} $ найдется как решение линейного уравнения и т.д. - последнее слагаемое
всегда будет иметь вид $ a_{n} b_{1}^{n} $, а предыдущие коэффициенты будут 
к этому моменту известны.


\subsection{Теорема}

$\exists! A(t) : a_{0} \neq 0$
Тогда $ \exists! B(t): A(t)B(t) = 1 $

Доказательство:

Мы знаем соотношения коэффициентов $ c $ в произведении функций:
$c_0 = a_0 b_0$	\\
$c_1 = a_0 b_1 + a_1 b_0 $ \\
$c_2 = a_0 b_2 + a_1 b_1 + a_2 b_0  $\\
$c_3 = a_0 b_3 + a_1 b_2 + a_2 b_1 + a_0 b_3$\\
$\ldots$\\

Для того, чтобы $ A(t)B(t) = 1 $ достаточно чтобы $ a_0 b_0 = 1 $ 
(находим $ b_{0} = \dfrac{1}{a_{0}} $)

Остальные же коэффициенты должны равняться 0, 
для этого последовательно находим коэффиициенты при $ b_{i} $ решением линейным уравнений.

Доказано

\subsection{Дифференцирование производящих функций}

Если есть производящая функция
$$
A(t) = a_0 + a_1 t + a_2 t^2 + a_3 t^3 + \ldots
$$

То формально ее можно продифференцировать:
$$
A(t) = a_1 + 2 a_2 t + 3 a_3 t^2 + 4 a_4 t^3 + \ldots
$$
А можно проинтегрировать:
$$
\int A(t) dt  = a_0 t + \frac{a_1 t^2}{2} + \frac{a_2 t^3}{3} + \frac{a_3 t^4}{4} + \ldots
$$

Несложно понять, что производная интеграла дает исходный ряд:
$$
(\int A(t) dt )' = A(t)
$$

Если ряды производящих функций сходятся, можно использовать соотвествующий аппарат матанализа.

\textbf{Пример:}

$ e^t = 1 + \frac{t}{1!} + \frac{t^{2}}{2!} + \frac{t^{3}}{3!} +\ldots$ \\

Этот ряд хорош тем что он всегда (и довольно быстро) сходится

Можно вспомнить тригонометрические функции (тоже всега сходятся)
$$
\sin t = \frac{t}{1!} - \frac{t^{3}}{3!} + \frac{t^{5}}{5!} - \ldots
$$
$$
\cos t = 1 - \frac{t^{2}}{2!} + \frac{t^{4}}{4!} - \ldots
$$

Еще несколько примеров рядов:

Биномиальный ряд (сходится не всегда)
В случае p натурально, он становится биномом, в обратном случае получается ряд

$$
(1+t)^p = 1 + pt + \frac{p(p-1)}{2!}t^{2} + \frac{p(p-1)(p-2)}{3!}t^{3} + \ldots
$$

$$
ln( \dfrac{1}{1-t}) = t + \dfrac{t^{2}}{2}+ \dfrac{t^{3}}{3} + \dfrac{t^{4}}{4} + \ldots
$$

Оперируя рядами, можно проверять различные выкладки, например 

$ e^{t} e^{-t} = 1 $ \\
$(1+t)^p(1+t)^q = (1+t)^{p+q}$ \\
$\sin{t}^2 + \cos{t}^2 = 1$ \\

Т.е., доказать эти равенства не для функции, а для формальных степенных рядов.

\subsection{Пример}

Простейшая последовательность: 
$1,1,1,\ldots$

$A(t) = 1+t+t^{2}+t^{3}+ \ldots = \dfrac{1}{1-t}$

\subsection{Пример. Геометрическая прогрессия}

$b,bq,bq^{2},bq^{3},\ldots$

$b + bqt + bq^2t^2 + bq^3t^3 + \ldots (1 + qt + (qt)^2 + (qt)^3 + \ldots) + \ldots 
= \dfrac{b}{1 - qt}$

\section{Использование производящих функций}

\subsection{Числа Фибоначчи}

$F_{n+2} = F_{n+1} + F_{n}$ \\
$0,1,1,2,3,5,8,13,21,34,55,89,144, \ldots$ \\

Производящая функция для чисел фибоначчи

\begin{equation}
\label{fib1}
Fib(t) = F_0 +  F_1 t + F_2 t^2 + \ldots
\end{equation}

Воспользуемся приведенным рекуррентным соотношнием для выяснения 
каких-либо свойств производящей функции.

Например, умножим на $ t^{2} $
\begin{equation}
\label{fib2}
t^2 \cdot Fib(t) = F_0 t^2 +  F_1 t^3 + F_2 t^4 + F_3 t^5 + \ldots
\end{equation}

Ту же самую производящую функию умножим на $ t $
\begin{equation}
\label{fib3}
t \cdot Fib(t) = F_0 t +  F_1 t^2 + F_2 t^3+ F_3 t^4 + \ldots
\end{equation}

В выражениях одно под другим приведены коэффициенты при определенных степенях $ (2,3,\ldots) $

Т.е., в формуле (\ref{fib1}) при коэффициентах с одинаковым показателем при $ t $ приведена сумма 
(\ref{fib2}) и (\ref{fib3})

Также получили формулу
$$
(t + t^2) Fib(t) = Fib(t) - t
$$
Разрешив относительно $ Fib(t) $, получаем
$$
Fib(t) = \dfrac{-t}{1-t-t^2}
$$

Т.е., некая рациональная функция, которую можно расписать на простейшие.

Решим квадратное уравнение

$t^2 + t + 1 = 0$

Его корни:

$\phi_{1,2} = \dfrac{-1 \pm \sqrt{5}}{2}$

Таким образом, разложение производящей функции на простейшие выглядит так:

$ Fib(t) = \dfrac{-t}{(t-\phi_1)(t-\phi_2)} =$

$ = 
\dfrac{\alpha}{(1-\phi_2)}  
+
\dfrac{\beta}{(1-\phi_1)}$ \\

$ -t = \alpha(t - \phi_1) + \beta (t - \phi_2)$ \\

Числа $ \alpha, \beta $, стоящие в числителе, можно найти, воспользовавшись 
методом неопределённых коэффициентов 
(представление  функции в виде в виде суммы элементарных дробей).

Должны совпасть коэффициенты при $ t^{1} $ и $ t^{0} $, поскольку многочлены равны


$\begin{cases}
- 1 = \alpha + \beta \\
0 = - \alpha \phi_2 - \beta \phi_1
\end{cases}$

% найдем бета
$\beta = 
\dfrac{-\alpha \phi_2}
{\phi_1} $ 

% подставим его
$ -1 = \alpha - \dfrac{\alpha  \phi_2}{\phi_1} = $ 

$ = \alpha (1 - \frac{\phi_1}{\phi_1})  $

$ = \alpha \left( \dfrac{\phi_1 - \phi_2}{\phi_1}  \right)$ \\

Вспомним, что $ \phi_1, \phi_2 $ - корни квадратного уравнения, и их разность
$ \phi_2 - \phi_1 = \sqrt{5}$


$\alpha = - \dfrac{\phi_1}
{\phi_1 - \phi_2} 
=
- \dfrac{\phi_1}
{\sqrt{5}}
$

$\beta = 
\dfrac{\phi_2}
{\phi_1 - \phi_2} 
=
\dfrac{\phi_2}
{ \sqrt{5} } 
$ 


Подаставляя в формулу для чисел Фибоначчи, получаем:


$$
Fib(t) =
\dfrac{\alpha}{(1-\phi_1)}
+
\dfrac{\beta}{(1-\phi_2)} 
=
\dfrac{ - \phi_1}{\sqrt{5} (t-\phi_1)}  
+
\dfrac{\phi_2}{\sqrt{5} (t-\phi_2)} 
=
\dfrac{1}{\sqrt{5}} 
\left( 
\dfrac{-1}{1-\frac{t}{\phi_2} }
+
\dfrac{1}{1-\frac{t}{\phi_1} }
\right)=
$$

$$
 = \dfrac{1}{\sqrt{5}} \sum\limits_{n=0}^{\infty}
( \frac{-1}{\phi_2^n} + \frac{1}{\phi_1^n} ) t^{n}  =
$$

Рассмотрим подробнее $ \phi_1 $ и $ \phi_2 $

По теореме Виета:
$ \phi_1 \phi_2  = 1$

Поэтому 

$$
= \frac{1}{\sqrt{5}} \sum\limits_{n = 0}^{\infty} 
( (- \phi_1)^{n}  + ( -\phi_1)^{n}) t^n 
= 
\sum\limits_{i = 0}^{\infty}
\left(
(\frac{1 + \sqrt{5}}{2})^n - (\frac{1 - \sqrt{5}}{2})^n 
\right) t^n
$$

В итоге, найдена формула для чисел Фибоначчи выглядит следующим образом 
(носит название формулы Бинэ)

\begin{equation}
\label{bine}
F_n = \sum\limits_{i = 0}^{\infty}
\left(
(\frac{1 + \sqrt{5}}{2})^n - (\frac{1 - \sqrt{5}}{2})^n
\right) t^n
\end{equation}

Проведем небольшую оценку.

Рассмотрим сторое слагаемое, стоящее под знаком суммы.

Заметим, что оно довольно быстро стремится к нулю (начиная с $ n = 2 $ оно меньше $ 1/2 $).

$\lim\limits_{n \to \infty} (\frac{1 - \sqrt{5}}{2}) = 0$

Отсюда мы сразу можем написать асимптотику - 
полученная формула будет эквивалентна более простой:
$$
F_n = \sum\limits_{i = 0}^{\infty}
\left(
(\frac{1 + \sqrt{5}}{2})^n - (\frac{1 - \sqrt{5}}{2})^n
\right) t^n
\sim
\left(
\frac{1 + \sqrt{5}}{2}
\right) ^ n
$$

Т.е., это рациональное число, очень близкое к какому-то целому. Это целое и будет n-м числом Фибоначчи.

Формулу можно использовать для различных приложений, например суммы $ n $ чисел Фибоначчи (сводится к геометрической прогрессии):

$$
F_1 + F_2 + F_3 + \ldots 
$$

\subsection{Теорема}

$$
\begin{pmatrix}
0 & 1 \\
1 & 1
\end{pmatrix}^{n}
=
\begin{pmatrix}
F_{n-1} & F_{n} \\
F_{n} & F_{n+1}
\end{pmatrix}
$$
Доказательство (по индукции):

Верно для $ n=0, n=1 $

Проверим для n = n + 1

$$
\begin{pmatrix}
0 & 1 \\
1 & 1
\end{pmatrix}^{n}
\cdot
\begin{pmatrix}
F_{n-1} & F_{n} \\
F_{n} & F_{n+1}
\end{pmatrix}
=
\begin{pmatrix}
F_{n} & F_{n+1} \\
F_{n-1} + F_{n} & F_{n} + F_{n+1}
\end{pmatrix}
=
\begin{pmatrix}
F_{n} & F_{n+1} \\
F_{n+1} & F_{n+2}
\end{pmatrix}
$$

\textbf{Пример:}

$$
\begin{pmatrix}
0 & 1 \\
1 & 1
\end{pmatrix}^{2n}
=
\left(
\begin{pmatrix}
0 & 1 \\
1 & 1
\end{pmatrix}^{n}
\right)^{2}
=
\begin{pmatrix}
F_{2n-1} & F_{2n} \\
F_{2n} & F_{2n+1}
\end{pmatrix}^{2}
= 
\begin{pmatrix}
F_{2n-1}^{2} + F_{n}^{2} & F_{n-1} F_{n} + F_{n} F_{n+1} \\
F_{n-1} F_{n} + F_{n} F_{n+1} & F_{n}^{2} + F_{n+1}^{2}
\end{pmatrix}
$$

Из данной матричной формулы можно получить довольно много полезных соотношений.
$$
F_{n}^{2} + F_{n+1}^{2}  = F_{2n+1}
$$
$$
F_{n}( F_{n-1} + F_{n+1} )= F_{2n}
$$

\section{Линейные рекуррентные соотношения}

Исследуем последовательности вида:

$$
a_{n+k} = c_1 a_{n+k-1} + c_2 a_{n+k-2} + \ldots + c_k a_n
$$
т.е., в которых значение $ n+k $ элемента последовательности 
зависит от $ k $ предыдущих занчений.

Соответственно, необходимо задать хотя бы $ k-1 $ 
начальных значений для рекурсии глубины $ k $

Для ЛРС мы тоже можем рассмотреть производящую функцию:

$$
A(t) = a_0 + a_1 t + a_2 t^2 + \ldots
$$

и проделать следующие действия:

домножить ее на $c_1 t$: $ c_1 t A(t) $ \\
домножить ее на $c_2 t^2$: $ c_2 t^{2} A(t) $ \\
домножить ее на $c_3 t^3$: $ c_3 t^{3} A(t) $ \\
$\ldots$ \\
домножить ее на $c_{k} t^{k}$: $ c_k t^{k} A(t) $ \\

И потом сложить все полученные соотношения.

Посмотрим, как будет устроен коэфициент при $ t^{n+k} $

В выражении $ c_1 t A(t) $ такой показатель можно получить, 
если взять из A(t) $ n+k-1$-й член ряда: \\
($ c_{1} a_{n_k-1} $).

В выражении $ c_2 t^{2} A(t)$ такой показатель можно получить, 
если взять из A(t) $ n+k-2$-й член ряда: \\
($ с_{2} a_{n_k-2} $).

И т.д. до $ n$-го члена: \\
($ c_{k} a_{n} $).

Полученные значения сложим, и по рекуррентному соотношению 
получим $ a_{n+k}$-й коэффициент, т.е. тот самый, который должен быть 
в производящей функции $ A(t) $:

$(c_0 + c_1 t + c_2 t^2 + \ldots + c_k t^k) A(t) = A(t) - P(t)$

$ P(t) $ - какой-то полином, необходимый для того, чтобы при $ t^{n+k} $
по крайней мере выполнялось условие $ n =0 $.

Т.е. степень полинома $ P(t) $ меньше $ k $.

Выведем саму производящую функцию:

$0 = A(t) - P(t) - A(t)(c_0 + c_1 t + c_2 t^2 + \ldots + c_k t^k)$

$A(t)(1 - (c_0 + c_1 t + c_2 t^2 + \ldots + c_k t^k) ) = P(t)$

$$
A(t) = \dfrac{P(t)}{Q(t)}
$$

Где $Q(t) = 1 - c_1 t - c_2 t^2 - \ldots - c_k t^k$

В данном случае производящая функция будет рациональной, причем
степень числителя меньше степени знаменателя.

Далее следует разложить функцию на простейшие дроби.

Пусть функция в общем виде:
$Q(t) = (1 - \alpha_1 t)^{m_1} (1 - \alpha_2 t)^{m_2} (1 - \alpha_n t)^{m_n}$

$1/\alpha_1, 1/\alpha_2, \ldots, 1/\alpha_n - $ корни данного многочлена 
(в общем случае могут быть и комплексными)

Тогда разложение производящей функции в общем виде:

$$
\frac{P(t)}{Q(t)} = \sum\limits_{j=1}^{n} 
\left(
\frac{b_{j_1}}{1-a_j t} 
+ \frac{b_{j_2}}{(1-a_j t)^{2}} 
+ \frac{b_{j_3}}{(1-a_j t)^{3}}
+ \ldots 
+ \frac{b_{j_m}}{(1-a_j t)^{m_j}}
\right)
$$
где $ b_{j_i} $ - какие-то коэффициенты, 
которые находятся решением соответсвующего линейного уравнения.

Рассмотрим слагаемые от 1 до $ m $

1. $ \frac{1}{1-a t} = \sum\limits_{n=0}^{\infty} {a}^{n} {t}^{n}$ - геометрическая прогрессия 

2. $ \frac{b_{j_2}}{(1-a_j t)^{2}} = \frac{1}{\alpha} \left( \frac{1}{(1-a t)} \right)' $ - очевидно, несколько преобразованная производная предыдущего слагаемого\\

3. То же самое со второй производной, и т.д.

\subsection{Пример}

Пусть имеется неограниченный запас монет достоинством 1,2,5,10 рублей.

Мы хотим узнать, каким количеством способов $ a_n $ можно сумму $ N $
представить в виде суммы доступных монет?

Найдем производящую функцию для этой последоватльности:

Начнем со случая, когда имеется только одна разновидность монет - 1 рубль.

Тогда количество способов заплатить сумму $ N $ рублей ровно одно:

Поэтому производящая функция для этой последоватльности $1,1,1,1,\ldots$:

$A_1(t) = 1 + t + t^{2} + t^{3} + \ldots = 1/(1-t)$ 
- т.е. геометрическая прогрессая со степенью 1\\

Допустим, имеется одна разновидность монет - только 2 рубля.

Теперь, если сумма нечетна, мы можем заплатить 0 способами, 
если четна - единственным способом $2,2,2,2,\ldots$

Тогда
$A_2(t) + = 1/(1-t^2)$  \\

Перемножив две получившиеся функции, заметим, что коэффициент при $ t^{n} $ равен количеству способов представить $n = 2a + b: a,b \ge 0$

Т.е., если имеется 2 вида монет - 1 и 2 рубля.

$A_{1,2} = A_1 A_2 = \frac{1}{1 - t} \frac{1}{1 - t^{2} }$ \\

По аналогии:

$A_{1,2,5} = A_1 A_2 A_5 = \frac{1}{1 - t} \frac{1}{1 - t^{2}} \frac{1}{1 - t^{5}}$ \\

$A_{1,2,5,10} = A_1 A_2 A_5 A_{10} = \frac{1}{1 - t} \frac{1}{1 - t^{2}} \frac{1}{1 - t^{5}} \frac{1}{1 - t^{10}}$ \\

\subsection{Общий пример}

Пусть имеется некое множество, состоящее из натуральных чисел
$ H \subset \mathbb{N}\setminus\{0\} $ \\

Обозначим через 
$p(H,n)$ количество способов представить $ n $ в виде суммы слагаемых, содержащихся в множестве $ H $\\

В приведенном выше случае  $ H = \{1,2,5,10\}$ \\

Можем найти производящую функцию для подобной последовательности 
(т.е., оббобщение приведенного примера)

$P(t) = \sum\limits_{k \subset H}^{} p(H,n)t^n $

В частности, производящая функция просто числа разбиений - это следующее произведение

$ P(t) = \prod\limits_{k=1}^{\infty} \frac{1}{1-t^k}$ \\

Если же мы вводим ограничение, что каждый элемент мы можем использовать не более $ d $ раз, их производящие функции

Только для 1:\\
$P(t) = \prod\limits_{k=1}^{\infty} ( 1 + t + t^{2} + ... + t^{d} )$ \\

Только для 2:\\
$P(t) = \prod\limits_{k=1}^{\infty} ( 1 + t^2 + t^{4} + ... + t^{2d} )$ \\

И в общем случае:

$P(t) = \prod\limits_{k=1}^{\infty} ( 1 + t^k + t^{2k} + ... + t^{dk} )$ \\

\subsection{Теорема Эйлера. Доказательство с помощью производящих функций}

Формулировка: количество способов представить $ N $ в виде суммы различных слагаемых равно количеству способов представить в виде нечетных.

(В предыдущей главе теорема была доказана комбинаторно).

Если каждое слагаемое используется неболее 1 раза, производящая функция разбиения:

$P(t) = \prod\limits_{k=1}^{\infty}  (1+t^k)$ \\

Если используется нечетное число слагаемых, производящая функция разбиения:

$ P(t) = \prod\limits_{k=1}^{\infty}  \frac{1}{1-t^{2k-1}} $

Доказать:
$$
\prod\limits_{k=1}^{\infty}  (1+t^k) = \prod\limits_{k=1}^{\infty}  \frac{1}{1-t^{2k-1}}
$$

Заметим, что в выражении

$ \prod\limits_{k=1}^{\infty}  (1+t^k) =
\dfrac
{\prod\limits_{k=1}^{\infty} (1-t^{2k}) }
{\prod\limits_{k=1}^{\infty} (1-t^k) }$ 

Т.е., в числителе только четные показатели степени, в знаменателе - любые.

Если их последовательно разделить, остануться только нечетные, потому что все четные сократятся, т.е. каждый множитель сокражается как $ \frac{(1-t^{k})(1+t^{k})}{(1-t^{k})} $

А это и есть исходное равенство.

\subsection{Метод перевала. Формула Харди-Рамануджана}

Если ряд производящей функции сходится, то работа с ней сильно упрощается.
Например, если есть сходящийся ряд:

$$
\sum\limits_{n = 0}^{\infty} a_n t^n
$$

Коэффициент при n-й степени находится при помощи интегральной функции.

$$
a_n  = \frac{1}{2 \pi i} \int \frac{F(t)}{t^{n+1}} dt
$$

Интеграл, охватывающий контур точки 0.

Если есть интеграл, асимптотику последовательности можно найти гораздо проще.
Существуют общие методы. позволяющие оценивать асимптотику подобных последовательностей.
Например, \textbf{формула Харди-Рамануджана}

$$
P(n) \sim \dfrac{1}{4 n \sqrt{3}} e^{\pi \sqrt{\frac{2n}{3}}}
$$
где $ P(n) $ - общее число разбиений $ n $, для которого 
$ F(t) = \prod\limits_{k = 1}^{\infty} \dfrac{1}{1 - t^{k}} $

Все примем без доказательства.


\subsection{Производящая функция бинома}

Производящая функция бинома:
$$
(1+t)^n = \sum_{k=0}^n C_n^k t^k 
$$

$$
\frac{1}{(1-t)^{n+1}} = \sum\limits_{n+k}^{n} t^k 
$$
