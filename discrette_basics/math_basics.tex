\documentclass[12pt]{book}
 
\usepackage{lineno}
\usepackage[utf8]{inputenc}

\usepackage[all]{xy}

\usepackage{geometry} 
\geometry{a4paper}
\geometry{margin=1in} 


\usepackage{graphicx} 


\usepackage{array} 
\usepackage{ulem} % underlines, double underlines, wavy underlines etc
\usepackage{amsmath, amsthm, amssymb}
\usepackage{amsfonts} % Натуральные числа и прочие множества красивыми буковками
\usepackage{paralist} %списки: itemize, enumerate etc
\usepackage{subfig}
\usepackage[english,russian]{babel} 
 

\usepackage{fancyhdr} 
\pagestyle{fancy}
\renewcommand{\headrulewidth}{1pt}

\renewcommand{\baselinestretch}{1.4}

\lhead{Введение в математику }\chead{}\rhead{}
\lfoot{}\cfoot{\thepage}\rfoot{}

\usepackage{sectsty}
\allsectionsfont{\sffamily\mdseries\upshape} 
	
\everymath{\displaystyle}
\modulolinenumbers[3]


%%% %%% ToC (table of contents) APPEARANCE
\usepackage[nottoc,notlof,notlot]{tocbibind} % Put the bibliography in the ToC
\usepackage[titles,subfigure]{tocloft} % Alter the style of the Table of Contents
\renewcommand{\cftsecfont}{\rmfamily\mdseries\upshape}
\renewcommand{\cftsecpagefont}{\rmfamily\mdseries\upshape} 



\title{Введение в математику}
\author{Храбров А.И.}
\date{2010-2011}
\begin{document} 

\maketitle

\tableofcontents

\chapter{ Лекция 3 }
\section{Основы комбинаторики.}

\begin{description}
\item[Рекомендуемая литература]~ 

1. Виленкин. "Комбинаторика" (переиздание 2006 г.) \\
2. Риордан. "Введение в комбинаторный анализ" \\
3. Грэхем, Кнут. Паташник. "Конкретная математика" \\
\end{description}

\subsection{Основные правила}

\begin{description}
\item[Задача комбинаторики]~	

Задача комбинаторики - пересчет количества элементов (сочетаний, перестановок, размещений и перечислений элементов) и отношений на них.

\item[Правило суммы]  ~

Если элемент из множества $A$ можно выбрать $n$ способами, а элемент из множества $B$ можно выбрать m способами, то выбрать элемент либо из $A$, либо или из $B$ можно сделать $n+m$ способами.

\item[Правило умножения]  ~

Если мы берём сначала элемент из $A$, затем элемент из $B$, то количество способов выбрать упорядоченную пару $m \times n $ способов.

TODO Картинка декартова произведения

\item[Правило биекции]  ~

Если есть два конечных множества $\mathcal{A}$ и $\mathcal{B}$, и имеется биекция, то в множествах одинаковое количество элементов. 
\end{description}

\subsection{Примеры выборок}

Пусть в чемпионате по футболу участвует 16 команд, и первые три места могут быть заняты любыми 3 командами. \\
Очевидно, первое место может занять одна из 16 команд, второе - одна из оставшияхся 15, третье - одна из оставшихся 14. \\
По правилу произведения количество различных выборок $16 \cdot 15 \cdot 14 = 3360$ \\

Модифицируем конструкцию для демострации правила суммы. Пусть сначала 16 команд играют групповые турниры в группах по 4, и из каждой группы выходит 8 команд, и из них уже выбираются первые три места.

TODO картинка

Здесь будет меньше вариантов, т.к. все три команды победителей не могут оказаться из одной группы.
Поэтому чтобы посчитать общее количество, необходимо из уже полученного количества выборок вычесть все запрещенные варианты.

Посчитать их можно следующим образом: из первой группы не могут выйти в победители три команды $4 \cdot 3 \cdot 2$, также из второй, третьей и четвертой. Т.е. ответ:

$16 \cdot 15 \cdot 14 - 4 \cdot 4 \cdot 3 \cdot 2 = 3264$

\subsection{Комбинаторные определения}

\begin{description}
\item[Перестановки]~	

Перестановка --- один из способов переставить элементы множества в различном порядке. Другими словами, это биекция множества на себя, или же кортеж (упорядоченное множество из различных чисел от 1 до n).

Первый элемент может быть любым (n вариантов). на второй очевидно остается (n-1 вариантов) и т.д. до последнего.

Из вышесказанного очевидно, что количество перестановок выражается, как

$$
n(n-1)(n-2) \ldots 3 \cdot 2 \cdot 1 = n!
$$

\end{description}

\begin{description}
\item[Количество всех подмножеств данного множества]~	

TODO картинка

Т.е. на каждый элемент имеется 2 варианта - он либо есть в подмножестве, либо нет.
Про второй, третий и т.д. изветсно то же самое.

Перемножим все варианты
$2 \cdot 2 \cdot 2 \cdot 2 \cdot 2 \cdot 2 = 2^n$

Кстати, это объясняет, почему множество всех подмножеств принято обозначать $2^X$

\end{description}


\begin{description}
\item[Размещения]~	

Пусть у нас имеется n элементов, 
${1,2, \ldots , n}$
и мы выбираем из них кортеж, состоящий только из k различных элементов этого множества.
Такая выборка называется размещением. Посчитаем их число. Очевидно, таким же образом, как и с перестановками, просто эту запись нужно в какой-то момент оборвать:
$$
n(n-1)(n-2) \ldots (n - k + 1) = \frac{n!}{(n-k)!}
$$

Обозначается размещение следующим образом:
$$
A_n^k = \frac{n!}{(n-k)!}
$$
\end{description}

\begin{description}
\item[Сочетания]~	

Аналогично, пусть у нас имеется n элементов, и мы выбираем из них некое k-элементное неупорядоченное подмножество.

$${1,2, \ldots , n}$$

Такая выборка называется сочетанием. Посчитать их число также несложно. 
Возьмем k-элементное подмножество с различными элементами.
Тогда мы можем сделать из него кортеж $k!$ способами, просто переставляя элементы.


$\{a_1, a_2, \cdots ,a_k\} \leadsto$  кортеж $k!$ способами 

Это количество не зависит от того, какое конкретно мы взяли подмножествою
Поэтому чтобы узнать количество неупорядоченных подмножеств, 
количество соответствующих k-элементных кортежей необходимо поделить на k!


Обозначается размещение следующим образом:
$$
C_n^k = \frac{n!}{(n-k)!k!}
$$
Также этот объект также называется биномиальным коэффициентом.

\end{description}

\begin{description}
\item[Размещения с повторениями]~	

Во всех прошлых примерах мы предполагали, что количество элементов различно.
Но это не всегда так.
Возьмем некоторое n-элементное множество и составим k-элементый кортеж из его элементов, причем любой элемент используется любое число раз.\\
${a_1, a_2, \cdots ,a_n}$\\
Очевидно, первый элемент выбирается n способами, второй - n способами, и так вплоть до k-го.
Поэтому число размещений с повторениями определяется так:
$$
\bar{A}_n^k = n^k
$$

\end{description}


\begin{description}
\item[Сочетания с повторениями]~	

Сочетания с повторениями является гораздо более интересным комбинаторным объектом.

Возьмем некоторое n-элементное множество и составим k-элементное неупорядоченное подмножество из его элементов, причем любой элемент используется любое число раз.\\
${a_1, a_2, \ldots ,a_n}$\\

Обратите внимание, что k может быть больше n, т.к. любой элемент мы можем брать любое число раз.
Подсчитывается исходя из комбинаторных соображений

%TODO полоска
TODO полоска

В множестве $n+k-1$ клетка. Покрасим в этой полоске $n-1$ клетку (что также равно k клеток).
Количество клеток от начала множества - это количество единиц, количество от первой закрашенной до второй - количество двоек, троек в множестве нет (нет промежутка между 2 и 3 закрашенными клетками), соответсвенно, 2 четверки и 1 пятерка.
Между выбором элементов и подобной картинкой существует взаимно однозначное соответствие.
Нам нужно просто выбрать $k$ клеток из $n+k-1$ (количество незакрашенных элементов)
Объект обозначается следующим образом:
$$
\bar{C}_n^k = C_{n+k-1}^{n-1} = C_{n+k-1}^k
$$
\end{description}

\section{Свойства биномиальных коэффициентов.}

\begin{description}
\item[Свойство 1]~	
$$
C_n^k = C_n^{n-k}
$$
Доказательство тождества очевидно из чисто комбинаторных соображений - если мы выбираем из множества некое подмножества $ n $ способами, то мы таким же количеством способов выкидываем оставшееся.
\end{description}


\begin{description}
\item[Свойство 2]~	
$$
C_n^k = C_{n-1}^{k-1} + C_{n-1}^k
$$


TODO картинки

Берем множество из n элементов. В левой части тождества написано количество способов выбрать k-элементное подмножество. Посчитаем его по-другому.

Выделим одну клетку. Она либо попала в это множество, либо не попала. \\
Посчитаем количество способов, когда эта клетка в подмножество не попала. \\
Т.е. имеем $ n-1 $ элемент. Из них нужно выбрать $ k $. \\
Это количество $C_{n-1}^k$ \\
\\
Теперь посчитаем количество способов выбрать k-элементное подмножество с учетом этой клетки. \\
Тогда мы должны выбрать ее, а из оставшихся клеток выбрать еще $ k-1 $.
Это количество $C_{n-1}^{k-1}$ - кол-во способов выбрать k-элементное подмножество, содержащее выделенную клетку. \\
Значит общее количество способов - сумма двух приведенных количеств сочетаний.

Вообще, комбинаторные тождества часто доказываются путем подсчетов одних и тех же сочетаний разными способами.

\end{description}


\begin{description}
\item[Свойство 3]~	

Сумма всех биномиальных коэффициентов равна 2 в степени n
$$
C_n^0 + C_n^1 + C_n^2 + \ldots + C_n^n = 2^n  
$$
Или в сокращенной форме: 
$$
\sum\limits_{i = 0}^{n} C_n^i = 2^n
$$

В правой части тождества - количество всех подмножеств множества из $ n $ элементов.
Мы его уже посчитали, оно равно $2^n$.

 \end{description}


\begin{description}
\item[Свойство 4]~	

$$
k C_n^k  = n C_{n-1}^{k-1}
$$

Для доказательства подобных тождеств лучше всего придумать комбинаторную трактовку одной из частей.

TODO массив

Рассмотрим левую часть тождества.
Пусть $C_n^k$ - выбор $k$-элементного подмножества.
Тогда умножение на $ k $ - выбор еще одного элемента из этого подмножества 
(для наглядности закрасим этот элемент)

Теперь посчитаем это выражение другим способом.
Сначала выберем закрашенный элемент. Это можно сделать n способами.
Теперь у нас остался $ n-1 $ элемент, и из них нужно выбрать $ k-1 $ элемент.

Это можно сделать $C_{n-1}^{k-1}$ способами.
Перемножив их, получае в точности правую часть равенства.

\end{description}


\begin{description}
\item[Свойство 5]~	

Предыдущее свойство является частным случаем этого.

$$
C_n^k C_{n-k}^{m-k} = C_m^k C_n^m
$$

Доказательство аналогично предыдущему.
%TODO массив
TODO массив

Рассмотрим правую часть тождества и правый множитель.
%TODO выделить
Выбираем из n-элементного множества m-элементное подмножество.
А потом берем выбранное m-элементное подмножество и закрашиваем в нем k элементов.
Количество способов сделать это $C_m^k$.
Тогда общее количество выборок равно правой части тождества, в котором покрашено $ k $ элементов.

Теперь посчитаем это выражение другим способом (левая часть тождества).
Сначала выберем в $n$-элементном множестве $ k $ элементов и их закрасим ($C_n^k$). \\
У нас осталось $n-k$-элементное множество, 
в котором нужно выбрать $ n-k $ элементов ($C_{m-k}^{n-k}$).
Перемножив их, получае в точности левую часть равенства.

\end{description}


\begin{description}
\item[Свойство 6]~	

Пусть имеется число сочетаний  с повторениями с повторениями, составленные из элементов $n+1$ вида: 
$$
\bar{C}_{n+1}^{k} = C_{n+k}^{k}
$$

Само множество:
$$
{1,2,\ldots,n,n + 1}
$$
Берем первый элемент $ i $ раз.
Осталось выбрать $ k-i $ элементов из множества ${2,\ldots,n,n+1}$

Количество способов сделать это $\bar{C}_{n}^{k-i} $
Поскольку $i$ может быть любым числом от 0 до $k$ получаем соотношение:
$$
\sum\limits_{i = 0}^{k} \bar{C}_{n}^{k-i} = \bar{C}_{n+1}^{k}
$$

Запишем обе части доказанного равенства через биномиальные коэффициенты.
$$
\sum\limits_{i = 0}^{k} C_{n+k-i-1}^{k-i} = C_{n+k}^{k} 
$$

Распишем в виде обычной суммы:
$$
C_{n+k-1}^{k} + C_{n+k-2}^{k-1} + C_{n+k-3}^{k-2} + \ldots + C_n^1 + C_{n-1}^0 = C_{n+k}^k
$$

Для удобства записи можно заменить $n$ на $n+1$
$$
C_{n+k}^{k} + C_{n+k-1}^{k-1} + \ldots + C_{n+1}^1 + C_{n}^0 = C_{n+k+1}^k
$$

А если еще воспользоваться Свойством 1 ( $C_n^k = C_n^{n-k}$ ),
то формула приобретает вид:
$$
C_{n}^{n} + C_{n+1}^{n} + \ldots + C_{n+k}^{n} = C_{n+k+1}^{n+1}
$$

Рассмотрим частные случаи:
$n=1$
$$
1 + 2 + \ldots + (k+1) = C_{k+2}^2 = \frac{(k+2)(k+1)}{2}
$$

$n=2$
$$
\frac{1\cdot2}{2} + \frac{2\cdot3}{2} + \ldots + \frac{(k+2)(k+1)}{2} = C_{k+3}^3 = \frac{(k+3)(k+2)(k+1)}{6}
$$

Если скомбинировать эти формулы, можно получить формулу для суммы квадратов (надо из второй умноженной на 2 вычесть первую).



 \end{description}

\begin{description}
\item[Свойство 7. Свертка Вандермонда]~	

Возьмем все то же количество сочетаний с повторениями.

$$
\bar{C}_n^k = C_{n+k-1}^k
$$
Очевидно, в данное сочетание входит не более k различных элементов
Разобьем это сочетание на i классов таким образом, что в i-й класс входит 
ровно i различных элементов.

Теперь посчитаем способов выбрать из одного конкретного i-го мнжества 
k-элементное множество Cс повторениями. 
Сначала положим в множество все эти $i$ элементов по одному разу, а оставшиеся элементы (их $k-i$ штук) могут быть любыми элементами из этих $i$, т.е. их в точности 
$$
\bar{C}_i^{k-i} = C_{k-1}^{k-i} = C_{k-1}^{i-1}
$$

% TODO непонятный момент
$$
\bar{C}_i^{k-i} = C_{i+k-1}^k
$$
А количество способов выбрать это i-е подмножество из n элементов - просто количество сочетаний. Таким образом, общее количество сочетаний с повторениями, в которые входит 
ровно $ i $ различных элементов:
$$
C_n^i C_{k-1}^{i-1}
$$

Теперь суммируем по всем классам:
$$
\sum\limits_{i=1}^k C_n^i C_{k-1}^{i-1} = \bar{C}_n^k = C_{n+k-1}^k
$$
$$
\sum\limits_{i = 1}^{k}  C_{k-1}^{k-i} C_n^i = C_{n+k-1}^k 
$$

В результате получаем
$$
C_{n+k-1}^k = \sum_{i = 1}^k C_{k-1}^{i-1} C_n^i = C_{k-1}^0 C_n^1 + C_{k-1}^1 C_n^2 + C_{k-1}^k C_n^3 + \ldots + C_{k-1}^{k-1} C_n^k
$$

\end{description}

\section{Треугольник Паскаля.}

TODO 

\section{Бином Ньютона.}

TODO 

\chapter{ Лекция 4 }
\section{Формула включений-исключений.}

\subsection{Теорема}

Путсь имеется N объектов $a_1, a_2, a_3, \ldots, a_n$, и у них имеются свойства $P(1), P(2), \ldots ,P(k)$
Обозначим через $N_i$ количество объектов, обладающих свойством $P(i)$
Вообще, индексы $N_{i_1, i_2, \ldots, i_k}$ - обзначим кол-во объектов, 
обладающих свойствами $P(i_1), P(i_2), \ldots , P_{i_k}$
Тогда кол-во объектов, $N(0)$ не обладающими ни одним из свойств. можно посчитать по формуле
$$
N(0) =  N 
- \sum\limits_{i}^{} N_{i} 
+ \sum\limits_{i_1, i_2}^{} N_{i_1, i_2} 
- \sum\limits_{i_1, i_2, i_3}^{} N_{i_1, i_2, i_3} 
+ \ldots 
+ (-1)^k \sum\limits_{i_1, \ldots, i_k}^{} N_{i_1, \ldots i_k} 
+ \ldots
+ (-1)^n N_{i_1, \ldots, i_n}^{k}
$$
Т.е. из общего кол-ва объектов вычтем кол-во объектов, обладающих одним свовом, прибавить кол-во объектов, облад. 2 св-вами, прибавит кол-во объектов, вычесть кол-во объектов, 3 св-вами и т.д., 1до количества объектов, обладающих всеми свойствами (там будет уже не сумма, а просто количество объектов, обладащих всеми свойствами, если такой конечно есть)

Док-во:

Объект, не обладающий ни одним из свойств, посчитан только один раз в $N$. Объект со свойством $P(i)$ посчитан один раз в $N$ и один раз в $N_i$, следовательно, он учтен $1-1=0$ раз. В общем случае объект, обладающий $r$ свойствами: $(j_1, j_2, \dots, j_r)$, мы учли в $N_{i_1, i_2, \dots, i_k}$, причем только в том
случае, если $(i_1, i_2, \dots, i_k) \subset (j_1, j_2, \dots, j_r)$. Таким образом, в $k$-й сумме он учтен $C_r^k$ раз.

% cut from here
% % % % % % % % % % % % %
%Если есть объект, не обладающий ни одним из свойств, мы его посчитали как разность всех объектов, и объектов, обладающим хотя бы одном свойством (посчитали ровно один раз).

%Теперь посмотрим, сколько раз мы посчитали объект, обладающий ровно одним свой свойством - один раз как N и один раз как сумму объектов, обладающих одним свойством. 
%// TODO выделить цветом
%$$
%N(0) =  N - \sum\limits_{i}^{k} N_{i} +  \sum\limits_{i_1, i_2}^{k} N_{i_1, i_2} - \sum\limits_{i_1, i_2, i_3}^{k} N_{i_1, i_2, i_3} + \ldots + (-1)^k \sum\limits_{i_1, \ldots, i_k}^{k} N_{i_1, \ldots i_k} + (-1)^n N_{i_1, \ldots, i_n}^{k}
%$$


%И в общем случае - объект, обладающий r свойствами: $(j_1, j_2, ,j_r)$, мы учли его в общем числе объектов N и в подмножестве индексов $N_{i_1, i_2, , i_k}$, причем только в том случае, если $(i_1, i_2, , i_k) \subset (j_1, j_2, ,j_m)$, по определению в k-й сумме $C_t^k$.

%В k-й сумме это будет количество выбрать k-элементное подмножество из множества r элементов, т.е. 
% % % % % % % % % % %
Т.е. объект мы посчитали всего 
$$
C_{r}^{0} - C_{r}^{1} + C_{r}^{2} -
 \ldots + (-1)^k C_{r}^{k} + (-1)^r C_{r}^{r} = (1 - 1)^r = 0
$$ 
Четные с плюсом, нечетные с минусом, т.е. по $(a-b)^r$ по биному Ньютона.
чтд

\subsection{Пример. Задача о беспорядках}

Пусть имеется N упорядоченных мест мест в театре, и пришедшие N человек расселись в случайном порядке.
Требуется установить количество перестановок, в которых ни один человек не занял свое место.

Т.е., формально

$\{1,2,\ldots, n \}$ 

$\{a_1,a_2,\ldots, a_n \}$ - перестановка n чисел

Сколько есть перестановок, таких что $a_i \neq i \forall i$?

Общее кол-во перестановок n!

$P_i$ означает, что $i$-й элемент попал на свое место.

Остальные можем переставить $(n-1)!$ способами, значит, $N_i=(n-1)!$.



TODO В лекции равно $N_{i_1, i_2, , i_k} = (n-k)!$
Если фиксированы элементы $i_1$, $i_2$, \dots, $i_k$, то остается $ (n-k)!$ вариантов,
значит по формуле включения-исключения имеем

$$
n! - C_{n}^{1} (n-1)! + C_{n}^{2} (n-2)! - C_{n}^{3} (n-3)! + \ldots + (-1)^n C_{n}^{n} (0!)
$$

Обратим внимание на то, что написано на k-м месте. Это не что иное как
$$
C_n^k(n-k)!= \frac{n!(n-k)!}{k!(n-k)!} = \frac{n!}{k!}
$$
Т.е. в каждой скобке есть $n!$, который можно вынести

$$
n! (1 - 1 + 1/2! -1/3! + 1/4! - ... + (-1)^n(1/n!)) \approx \frac{n!}{e}
$$
Это количество таких перестановок, что никакой элемент не оказался на своем месте.
Приближенно равно $e^{-1}$ по формуле Тейлора, обрезанная по n-му члену(причем сходится довольно быстро), т.е. отличается не больше чем на 1/(n+1)! Т.е. точность данной формулы достаточно высокая.

Т.е. ответ "задачи о беспорядках" - $1/e \approx 0,368$

\subsection{Пример. Задача о встречах}

Продолжение предыдущей задачи.

Мы хоим узнать, сколько таких перестановок, что ровно r элементов остались на своих местах.

Задача сводится к предыдущей:

Выберем r индексов из n элементов  - $C_n^r$. они зафиксированы, а на оставшихся местах - элементы в беспорядке.

Мы знаем количество беспорядкаов на остальных местах - нужно просто оборвать формулу из предыдущей задачи в определенный момент:
$$
C_{n}^{r}(n-r)!(1-1+1/2!-1/3! + \ldots + (-1)^{n-r} \frac{1}{(n-r)!} ) = 
\frac{n!}{r!} (1-1+1/2!-1/3! + \ldots + (-1)^{n-r} \frac{1}{(n-r)!} ) \approx \frac{n!}{r!e}
$$
Из этой формулы, например, можно посчитать вероятность того, что ровно один элемент наодится на своем месте. Она практически не отличаются от вероятности, когда ни один элемент не находтся на своем месте, т.е. разница исчезающе мала для больших $n$.

\subsection{Функция Эйлера}

Функция Эйлера --- количество чисел от 1 до $n$ взаимно простых с $n$.
Формула этой функции также находится с помощью формулы включений-исключений.

Пусть $n$ раскладывается на множители следующим способом:
$$
n = p_1^{{\alpha}_1} p_2^{{\alpha}_2} \ldots p_k^{{\alpha}_k}
$$
где $p_i$ - некие простые числа

Свойство, которое мы будем считать - это делимость числа m на $p_i$

Соответственно, мы будем интересоваться чисел, которые не делятся ни на одну из $p_i$

Т.е., нужно сначала найти, сколько чисел от 1 до n делатся на $p_{i_1}, p_{i_2},  \ldots ,p_{i_j}$

Числа, которые просто кратны p и не превосходящие $n$:
$$
p, 2p, \ldots, np/p 
$$

Их $n/p$ штук.

Аналогично количество чисел, не превосходящих $n$ и кратных 
$p_{i_1}$, $p_{i_2}$, \dots, $p_{i_j}$, равно
$$
\frac{n}{p_{i_1} p_{i_2} \ldots p_{i_j}}
$$

Поэтому формула ВИ

$$
\phi(n) = n - \sum\limits_{i_1} \frac{n}{p_i} + 
\sum\limits_{i_1 < i_2} \frac{n}{p_{i_1} p_{i_2}} 
- \sum\limits_{i_1 < i_2 < i_3}^{} \frac{n}{p_{i_1} p_{i_2} p_{i_2}} 
+ \ldots 
+ (-1)^{k} \frac{n}{p_1 p_2 \ldots p_k} 
= n(1 - 1/p_1)(1 - 1/p_2) \ldots (1 - 1/p_k)
$$

Выкладки по сокращению опущены.

Вкратце, выражение разворачивается в приведенное, т.к. все одиночные произведения  - со знаком минус, 
все попарные произведения  - со знаком минус, все по три - со знаком плюс и т.д.


\section{Разбиения.}

\subsection{Разбиение множеств}

Есть множество из n элементов (не обязательно чисел)
$$
\{a_1, a_2, \ldots, a_n\}
$$

необходимо узнать количество способов разбить его на множество из 
$k_1, k_2, k_3, \ldots, k_r$ элементов, так что $k_1+k_2+k_3+ \ldots+k_r = n$

Честный случай, когда у нас разбиение на 2 подмножества:

$k_1 + k_2 = n$

Тогда это биномиальный коэффициент $C_n^{k_1}$

Теперь посчитаем в общем случае. Если мы хотим разбить на произвольное количество элементов, 
последователбно выберем $k_1, k_2, k_3, \ldots, k_r$ элементов.

Сначала выберем $k_1$ элементов, количество способов сделать это $C_n^{k_1}$

Затем выберем $k_2$ элементов, количество способов сделать это $C_{n-k_1}^{k_2}$

И т.д. до r

Общее количество способов
$$
C_{n}^{k_1} C_{n-k_1}^{k_2} C_{n-k_1-k_2}^{k_3} \ldots C_{n-k_1-k_2- \ldots - k_{r-1}}^{k_r}
$$

Распишем по факториалам
$$
= \frac{n!}{(n-k_1)!k_1!} \frac{(n-k_1)!}{(n-k_1-k_2)!k_2!} \frac{(n-k_1-k_2)!}{(n-k_1-k_2-k_3)!k_3!} 
\ldots  \frac{(n-k_1-k_2- \ldots - k_r)!}{(n-k_1-k_2-k_3- \ldots - k_r)!k_r! 0!} 
$$

Числители и знаменатели попарно сокращаются, остается

$$
\frac{n!}{k_1! k_2! \ldots k_r!}
$$

Обратите внимание, что это обобщение биномиального коэффициента при $(r=2)$ .

Поэтому этот объект обычно называется \textbf{мультиномиальным коэффициентом}. Обозначается он как
$$
\binom{n}{k_1, k_2, \ldots, k_r}
$$


\begin{description}
\item[Пример. Раскладки в преферансе]~	

В преферансе раздается 32 карты, трем игрокам по 10 карт и 2 в прикупе. Порядок получения карт не имеет значения, порядок игроков разумеется существенен. Общее количесво раскладок будет $ \frac{32!}{10!10!10!2!} \approx 2,75 \cdot 10^{15} $ \\
Т.е. шансов повторения игры - практически никаких.

\end{description}

\begin{description}
\item[Пример]~	

Возьмем следующее выражение:
$$
(x_1 + x_2 + \ldots + x_r)^n
$$

Раскрыв скобки, получим сумму некоторых мономов:
$$
\sum x_1^{k_1} x_2^{k_2} \ldots x_r^{k_r}
$$

Для того, чтобы возникло $x_1^{k_1}$ мы должны в $k_1$ скобок выбрать первый элемент (т.е. отнести $k_1$ скобок в первое множество). \\
Для того, чтобы возникло $x_2^{k_2}$ мы должны в $k_2$ скобок выбрать второй элемент. \\
И т.д. \\
$\ldots$
Т.е. это то же самое, что взять и разбить множество на ${k_1! k_2! \ldots k_r!}$ элементов.
Т.е. перед каждый элемент суммы определяется мультиномиальным коэффициентом.
$$
\sum \binom{n}{k_1, k_2, \ldots, k_r} x_1^{k_1} x_2^{k_2} \ldots x_r^{k_r}
$$

Например при $k=2, r=2$ получим обычный бином.
$$
(x_1 + x_2)^2 = \sum \binom{2}{k_1=2, k_2=0} x_1^{k_1} x_2^{k_2} 
+ \sum \binom{2}{k_1=1, k_2=1} x_1^{k_1} x_2^{k_2}
+ \sum \binom{2}{k_1=0, k_2=2} x_1^{k_1} x_2^{k_2}
$$
Т.е., хорошо известная формула квадрата суммы.
\end{description}

\subsection{Задача о разбиении чисел}

В задаче о разбиении чисел имеется 2 варианта, и первый намного проще с точки зрения выяснения его свойств.

Пусть имеется число n, и мы выясняем количество способов представить его в виде суммы k слагаемых.
$$
n = x_1 + x_2 + \ldots + x_k
$$

\subsection{Задача об упордоченных разбиениях}

В первом варианте порядок разбиения имеет значение (мы считаем упорядоченные разбиения)

Т.е.
$$
3 = 2+1 \\
3 = 1+2
$$
- разные разбиения

Если мы интересуемся упорядоченными разбиениями, их можно представить следующим образом:

Обозначим число n как n точек. В промежутки между n точками поместим k-1 перегородок.
Всего перегородок может быть от 0 до n-1.
Т.к. нулей нет, по краям перегородка ставить нельзя (несколько в ряд тоже нельзя)

TODO картинка

Каждой расстановке перегородок соответсвует разбиение.
Таким оразом мы получили биекция между количесовом разбиений n чисел на k слагаемых и количеством расстановки перегородок.

Это количество нам известно, по определению оно равно 
$$
C_{n-1}^{k-1}
$$

TODO А что если количество перегородок = 0 \\

Если же мы хотим узнать количество всех возможных разбиений от 0 до n-1, это количество тоже изветсно
$$
2^{n-1}
$$
Получить его можно, просто сложив все биномиальные коэффициенты, или вспомнив, что это количество всех подмножеств данного множества (т.е., возможностей расставить хоть сколько-нибудь перегородок)

\subsection{Задача о неупордоченных разбиениях}

Если мы представляем в виде суммы 
$$
n = x_1 + x_2 + \ldots + x_k
$$

Такое количество разбиений обозначим $p_k(n)$

В данном случае считаем, что
$$
3 = 2+1
3 = 1+2
$$
- это одно и то же разбиение

Для того чтобы зафиксировать различные $x_n$, будем считать, что они не возрастают
$$
x_1 \ge x_2 \ge \ldots \ge x_k
$$

Теперь у нас есть только одно разбиение.

Заведем рекуррентную формулу:
$$
n = x_1 + x_2 + \ldots + x_k
$$

Вычтем из каждого x по 1
$$
n-k = (x_1 - 1) + (x_2 = 1) + \ldots + (x_k - 1)
$$

И обозначим каждую скобку как $y_i$
$$
n = y_1 + y_2 + \ldots + y_k
$$

Для всех y также выполняется условние монотонности
$$
y_1 \ge y_2 \ge \ldots \ge y_k
$$

и возможна ситуация, когда для какого-либо $y_s \ge 0$, а $y_{s+1} = 0$, тогда все последующие тоже равны 0:
$$
y_{s+1} = y_{s+1} = \ldots = y_k 0
$$

Соответственно, все 
$$
x_1 \ge x_2 \ge \ldots \ge x_s \ge 1
$$

Таким образом, получаем разбиение числа n на в точности s слагаемых, 
а т.к. $0 \ge s \ge k$, получим разбиение не более чем на k слагаемых (потому что какие-то их y могут обратиться в 0).

Такое количество несложно посчитать, просто сложив количество разбиений на 1 слагаемых, на 2 и т.д.

$$
p_k(n) = p_{k}(n-k) + p_{k-1}(n-k) + \ldots + p_{1}(n-k)
$$
Здесь $p_k(n)$ - количество способов разбить число n ровно на k слагаемых

Теперь внимательно посмотрим на сумму начиная со второго слагаемого
$$
p_k(n) = p_{k}(n-k) \underbrace{+ p_{k-1}(n-k) + \ldots + p_{1}(n-k)}_{p_{k-1}(n-1)}
$$

по той же формуле.

Таким образом, получим рекуррентное соотношение:
$$
p_k(n) = p_{k}(n-k) + p_{k-1}(n-1)
$$

TODO Спросить про таблицу.

Очевидно, при $n=k$ количество разбиений равно 1 (и даже при $n=k-1$ количество разбиений равно 1)

TODO соотношения для k=2 k=3 \\

Можно показать, что $p_k(n)$ является многочленом степени $k-1$ со старшим членом равным 
$$
\frac{n^{k-1}}{(k-1)!k!}
$$
, а коэффициенты $p_2(n), p_2(n), \ldots$ зависят от остатка $n$ по $ \mod k!$ (примем без доказательства)

По этой формуле удобно считать количество разбиений только для небольших номеров и n больше k.

\subsection{Соотношения между упорядоченными и неупордоченными разбиениями}

Посчитаем примерно, во сколько раз \textit{примерно} упорядоченных разбиений больше чем неупорядоченных.

Неупорядоченные разбиения (берем только первый член):
$$
p_k(n) = \frac{n^{k-1}}{(k-1)!k!} + \ldots
$$

Упорядоченные разбиения:
$$
C_{n-1}^{k-1} = \frac{(n-1)!}{(n-k)!(k-1)!}
$$

Составим отношение:
$$
\frac{C_{n-1}^{k-1}}{p_k(n)} = \frac{n^{k-1}}{(k-1)!k!} \frac{(k-1)!(n-k)!}{(n-1)!} =
$$
сокращаем $(k-1)!$

$$
= \frac{n^{k-1}}{(n-1)(n-2) \ldots (n-k+1) k!} = 
$$

Поделим каждую скобку почленно на n. \\
Теперь заметим, если n очень большое, то числитель стремится к 1.
$$
= \underbrace{\frac{1}{(1-1/n)(1-2/n) \ldots (1-(k+1)/n)}}_{ \lim_{n \to \infty} =1 }  \frac{1}{k!} = \frac{1}{k!}
$$

Вывод отсюда следует такой - неупорядоченному разбиению соответствет примерно $1/k!$ упорядоченных, а это просто число перестановок, а это означает, что в подавляющем большинстве разбиений слагаемые \textbf{различны}. 
Справедливо, если $n \gg k$

\subsection{Разбиение на нефиксированное число слагаемых}

Обозначается $p(n)$
$$
n = x_1 + x_2 + \ldots + x_k
$$
также будем считать, что $x$ не возрастают
$$
x_1 \ge x_2 \ge \ldots \ge x_k
$$

Получаем просто сумму 
$$
p(n) = p_1(n) + p_2(n) + \ldots + p_k(n)
$$

TODO спросить таблицу

Как видно из таблицы, скорость роста довольно большая.
Приведем еще больше значений, чтобы убедиться в этом.

TODO спросить еще таблицу

При первичной оценке $p(n)$ приближенно ведет себя как $e^{ \sqrt{n} }$

Для поиска p(n) сложно использовать биекцию (скорее сюрьекцию или иньекцию), т.е. получается не точное равенство, а некоторая оценка.
Также для их поиска используются производящие функции.

\section{Диаграммы Юнга}

TODO Нарисовать диаграммы (заметить, что пропущено разбиение 5=5) \\

$$
p(1) = 1
$$

$$
p(2) = 2
$$

$$
p(3) = 3
$$

$$
p(4) = 5
$$

$$
p(5) = 7
$$

$$
p(6) = 11
$$

Многие доказательства, связанные с разбиениями, основаны на подобных графических представлениях.

Также для представления разбиений используются \textbf{графы (диаграммы) Ферре} (просто используются точки вместо квадратов)

TODO рисунок

\subsection{Теоремы, доказываемые с помощью диаграмм}

\begin{description}
\item[Теорема]~	

Число разбиений $ n $ на $ k $ частей равно числу разбиений на части, наибольшая из которых равна $ k $.

Доказательство проводится чисто из визуальных соображений.

TODO рисунок

Рисуем диаграмму Юнга.
$$
k = 5, n = 10
$$

Из количества разбиений на $ k $ частей получить кол-во разбиений, наибольшая из которых равна $ k $, диаграмму можно взять и симметрично отразить относительно диагонали (такая диаграмма называется \textit{сопряженной}).

Доказательство очевидно из рисунка

TODO Рисунок
TODO

\end{description}


\begin{description}
\item[Теорема]~	

Число разбиений $ a-c $ в точности на $ b-1 $  слагаемое (такое что $ b \le c $), равно количеству 
разбиений числа $a-b$ в точности на $c-1$ слагаемое, каждое из которых не превосходит $b$.

TODO диаграммы Юнга (или Ферре)

$$
b = 5, c = 7, a - c = 13
$$

Пририсуем сверху (снизу) строчку из $c$ элементов.

Получим разбиение $c$ на $b$ элементов (т.к. перед добавлением там был $b-1$ элемент).

Вычеркнем этот столбец, и возьмем сопряженную диаграмму.

Т.е. мы получили разбиение на $c-1$ слагаемое $a_i$, каждое из которых $ a_i \le b$

\end{description}


\begin{description}
\item[Теорема Эйлера]~	

Рассматриваем разбиения только на слагаемые, принадлежащие какому-либо классу, например, только на нечетные, или только на различные.

TODO таблица

Интересно, что по крайней мере для 7 (а на самом деле и дальше) количество разбиений только на нечетные или только на различные, равно.

Между ними можно построить биекцию.
Возьмем разбиения на \textbf{различные} слагаемые. Выбыраем из них все четные, и разложим на одинаковые нечетные слагаемые, являющиеся максимальной степень двойки.
Т.е.
$$
2^a b = \underbrace{b + b + \ldots +b}_{2^a}
$$

Например, $12 = 3+3+3+3$

Таким образом, мы получили биекцию между разбиениями.

Но оказывается, существует и обратное преобразование (нечетные в различные).
Если есть нечетное число $ b $, встречающееся $ k $ раз, то из него можно восстановить различные
Запишем k в двоичной системе счисления.
$$
k = 2^0 + 2^{k_1} + 2^{k_2} + \ldots + 2^{k_n}
b + 2^{k_1} b + 2^{k_2} b +\ldots + 2^{k_n} b
$$
Например:
$$
k = 2^0 + 2^3 + 2^{10} + 2^{11}
b + 8 b + 1024 b + 2024 b
$$
Очевидно, все слагаемые будут различны, т.к. никакая степень двойки в двоичной записи 2 раза не встречается (не может дважды встретится $ 2^3 $, вместо этого мы записали бы  $ 2^6 $).


\end{description}







\chapter{ Лекция 5 }
\section{Асимптотики биномиальных коэффициентов}

\subsection{Задача оценки асимптотики. О-нотация}

Хотя задача о росте биномиального коэффициента не является 
принципиально сложной, иногда возникает необходимость эту скорость
с чем-либо сравнить.

Например, что растет быстрее:

$C_n^k$ или  $4^n $ \\

$ C_n^k $ или $ \frac{4^n}{n} $

и т.п.

На практике весто точных формул обычно используются асимптотические (приближенные).

\begin{description}
\item[Напомним некоторые определения из анализа:] ~

Пусть на множстве натуральных чисел заданы отображения $ f, g $ на множество действительных чисел.

$ \exists f, g \colon \mathbb{N} \to \mathbb{R} $ 

Фуекции называются эквивалентными, если удовлетворяется следующее условие:
$$
f ~ g \Leftrightarrow \lim_{n to 0} \frac{f(x)}{g(x)} = 1 
$$

Функция $ f $ называется "о-малым" от $ g $ если выполняется следующее условие:
$$
f = o(g) \Leftrightarrow \lim_{n to 0} \frac{f(x)}{g(x)} = 0
$$
Т.е. $ f $ доминирует над $ g $ асимптотически

Функция $ f $ называется "O-большим" от $ g $ если выполняется следующее условие:
$$
f = O(g) \Leftrightarrow |f(n)| \le C|g(n)|q 
$$
Т.е. $ f $ ограничена сверху функцией $ g $ (с точностью до постоянного множителя) асимптотически.

Вышесказанное может быть справедливо не для $ n_0 $, а для некоторого $ n_j (j > 0) $.

\end{description}

\begin{description}
\item[Основные формулы О-нотации:] ~

$o(f)+o(g) = o(f)$

$o(f) \cdot o(g) = o(f \cdot g)$

$o(o(f)) = o(f)$

$o(f)+O(f) = O(f)$

$o(f) \cdot O(f) = O(f \cdot g)$

$O(f) \cdot O(g) = O(f \cdot g)$

$O(o(f)) = o(f)$
\end{description}

\begin{description}
\item[Примеры:] ~
$$
n^2 + n + 1 \sim n^2 - 1
$$
$$
n^2 + n + 1 = o(n^3)
$$
$$
n^2 + n + 1 = O(n^2)
$$
$$
e^x = 1 + x + x^2/2 + O(x^3),  x \to 0
$$

\end{description}


\subsection{Простейшие оценки}

Начнем с простейших оценок - как растет биномиальный коэффициент $ C_{n}^{k} $ при росте $ k $?
Для четных коэффициентов:

$$
C_{2n}^{0} < C_{2n}^{1} < C_{2n}^{2} < \cdots < \underbrace{C_{2n}^{n+1}}_{\max}  > C_{2n}^{n+2} > \cdots C_{2n}^{2n}
$$
Очевидно, биномиальные коэффициенты растут от 0-го до среднего коэффициента, а дальше они убывают.
Это можно понять как из комбинаторных соображений, так и используя свойство $ C_{n}^{k} = C_{n}^{n-k} $

Для нечетных коэффициентов:
$$
C_{2+1}^{0} < C_{2n+1}^{1} < C_{2n}^{2} < \cdots < \underbrace{C_{2n}^{n} = C_{2n}^{n+1}}_{max}  > C_{2n}^{n+2} > \cdots C_{2n}^{2n}
$$
В данном случае 2 самых больших биномиальных коэффициента

\subsection{Теорема}
\begin{equation}
\label{binomial1}
\frac{4^n}{2n+1} < C_{2n}^{n} < 4^n = 2^n
\end{equation}

Докажем правую часть равенства:
$4^n = 2^n$ - сумма всех коэффициентов, а $ _{2n}^{n} $ - всего лишь наибольший из них


Докажем левую часть равенства:
$4^n = 2^n$ - сумма всех коэффициентов, а $ 2n + 1 $ - их количество, т.е. получившееся выражение меньше наибольшего коэффициента.


\begin{description}
\item[Замечание 1] ~
$$
\frac{4^n}{2\sqrt{n}} < \frac{4^n}{2\sqrt{3n+1}} < \frac{4^n}{\sqrt{3n}}
$$
Левая часть доказывается по индукции, правая приводится к индукции следующим образом:
$$ 
\frac{4^n}{\sqrt{3n+1}} < \frac{4^n}{\sqrt{3n}}
$$
\end{description}



\begin{description}
\item[Замечание 2] ~
На самом дделе, более точная оценка выглядит так:
$$
\frac{4^n}{\pi \sqrt{n+1/2}} < C_{2n}^{n} < \frac{4^n}{\pi \sqrt{n+1/4}}
$$

Без доказательства (доказывается из оценки следующего интеграла, см. Фихтенгольц):
$$
\int_0^{\pi} \sin^n{x} dx
$$

\end{description}


\section{Асимптотики факториала}

\subsection{Теорема}

Также полезно оценивать $ n! $

Докажем следующее отношение:

\begin{equation}
\label{factor1}
\left( \frac{n}{e} \right)^n < n! < e (n/2)^n 
\end{equation}



Главная часть оценки в выражении уже присутствует: $ \left( \frac{n}{e} \right)^n $

Доказательство левой части: по индукции

Доказательство правой части:

Заметим, что $ n! = 1 \cdot 2 \cdot 3 \cdot \ldots \cdot n $

Перемножим их попарно, и заметим что каждое произведение меньше полусуммы в квадрате:

$ 1 \cdot n  \le \left(  \frac{n+1}{2} \right)^2 $	\\
$ 2 \cdot (n-1) \le \left(  \frac{n+1}{2} \right)^2 $ 	\\
$ 3 \cdot (n-2) \le\left(  \frac{n+1}{2} \right)^2 $	\\
$ \ldots $\\
$ n \cdot 1 \le \left(  \frac{n+1}{2} \right)^2$	\\

Перемножив почленно, получим:
$$
n!^{2}  \le \left(  \frac{n+1}{2} \right)^{2n} 
$$

$$
n!  \le \left(  \frac{n+1}{2} \right)^{n} 
$$

Осталось заметить, что $ \left(  \frac{n+1}{2} \right)^{n} <  e (n/2)^n $

Это следует из следующих соображений:
$$
\left( \frac{\frac{n+1}{2}} {e (n/2)} \right)^n = \frac{(1 + 1/n)^n}{e^n}
$$

$$ \left( \frac{n+1}{n} \right)^n = (1 + 1/n)^n < e$$

%TODO прояснить момент,это из-за того, как растет n?
\textbf{TODO прояснить момент,это из-за того, как растет n? Связано ли с 1 зам. пределом?}

\subsection{Формула Стирлинга}

Такие оценки можно достаточно качественно уточнять, но
обычно при асимптотических оценках используются не равенства, а эквивалетности, а именно используется
\textbf{формула Стирлинга}

\begin{equation}
\label{stirling}
n! \sim \sqrt{2 \pi n} \left(\frac{n}{e}\right)^n
\end{equation}

Примем без доказательства.
Факториал - это произведение, т.е. логарифм этого произведения - некая сумма. Надо сумму заменить на интеграл и оценить.

Исходя из формулы Стирлинга найдем основные эквивалетности.

$$
C_{n}^{2n} = \frac{(2n)!}{n!n!} \sim \frac{(2^n)^{2n}c^{-2n} \sqrt{2 \pi 2 n}}{ (n^n e^{-n} \sqrt{2 \pi n})^2 } =
$$
сокращаем
$$
 = \frac{ 2^{2n} \sqrt{2 \pi 2 n}}{2 \pi n} =  \frac{4^n}{\sqrt{\pi n}} 
$$

Т.е., с такой скоростью растет самый большой биномиальный коэффициент. 
Это примерно посередине между элементарными неравенствами, которые мы рассмотрели выше ($ \ref{binomial1} $).
Меньше $ 4^n $ и больше $ 4^n $ делить на какое-то выражение кратное $ n $.

Оказалось, это выражение - $ \sqrt{n} $

\section{Асимптотики факториала и биномиальных коэффициентов}

\subsection{Теорема}

\begin{equation}
\label{factor2}
C_{n}^{k} \le \frac{n^k}{k!} < \left( \frac{ne}{k} \right)^k
\end{equation}

Попробуем сначала применить оценку, только что полученную с помощью формулы Стирлинга (\ref{stirling}):
$$
C_{2n}^{n} < \left( \frac{2ne}{c} \right)^n = (2e)^{n}
$$
Плохая оценка, учитывая что $ (2e)^{n} > 4^{n} $

Докажем левую часть:
$$
C_{n}^{k} = \frac{n!}{k!(n-k)!} = \frac{n(n-1)(n-2) \ldots (n-k+1)}{k!}
$$
Т.е. каждый из сомножителей в числителе меньше n

Поэтому:
$$
C_{n}^{k} \le \frac{n^k}{k!}
$$

Докажем правую часть:

Из (\ref{factor1}) нам известно, что $k! > (k/e)^k$

Поэтому:
$$
C_{n}^{k} \le \frac{n^k}{k!}
$$

Левая часть доказана.

Докажем правую часть:

Из (\ref{factor1} ) Нам известно, что $k! > \left( \frac{k}{e} \right)^k $

Поэтому:
$$
\frac{n^k}{k!} <  \frac{n^k}{k^k} e^k = \left( \frac{ne}{k} \right)^k
$$

Правая часть доказана.

Данная оценка работает хорошо только в том случае, когда $ n $ растет намного быстее, чем $ k $,
в идеале $ k $ фиксировано.

\subsection{Теорема. Уточнение оценки}

$$
\frac{n^k}{k!}(1 - \left (\frac{k(k+1)}{2n} \right)
\le C_n^k
\le \frac{n^{k}}{k!}   e^{-\frac{k(k-1)}{2n}}
$$


Земечание: \\
Пусть $k = o(\sqrt{n}) $, тогда $ \lim \frac{k}{\sqrt{n}} = 0 $

%TODO То есть у нас n - функция и k - функция?
\textbf{TODO То есть у нас n - функция и k - функция?}

Тогда 
$$
(1 - \left (\frac{k(k+1)}{2n} \right) = (1 - \frac{o(n)}{2n}) = 1
$$

И в показателе экспоненты та же картина
$$
-\frac{k(k-1)}{2n} = 0 \Rightarrow e^{-\frac{k(k-1)}{2n}} = 1
$$

Т.е. в левой и в правой части оставется $ \frac{n^k}{k!} $

Отсюда следует:

$$
C_n^k \sim \frac{n^k}{k!} \Leftrightarrow k = o(\sqrt{n}) 
$$

Следующее наблюдение, которое можно сделать - множители при $ \frac{n^k}{k!} $
с достаточно большой точностью равны:

$$
(1 - \left (\frac{k(k+1)}{2n} \right)
\approx
e^{-\frac{k(k-1)}{2n}}
$$

что неудивительно, т.к. это первые 2 слагаемых в формуле Тейлора для экспоненты.
%TODO ORLY?
\textbf{TODO Поясните плз (желательно выкладку)}

Докажем это.
Начало формулы такое же:
$$
C_{n}^{k} = \frac{n!}{k!(n-k)!} = \frac{n(n-1)(n-2) \ldots (n-k+1)}{k!} = 
$$
поделим теперь числитель и знаменатель на $ n $
$$
 = \frac{n^k}{k!}(1 - \frac{1}{n})(1 - \frac{2}{n}) \ldots (1 - \frac{k-1}{n})
$$

Для оценки сверху вспомним соображение, что $ e^x > 1 + x, x != 0 $

Выпишем все множители:

$ (1 - \frac{1}{n}) \le e^{-1/n}  $ \\
$ (1 - \frac{2}{n}) \le e^{-2/n} $ \\
$ \ldots $ \\
$ (1 - \frac{k-1}{n}) \le e^{-(k-1)/n}  $\\

Значит, все множители вместе:
$$
(1 - \frac{1}{n})(1 - \frac{2}{n}) \ldots (1 - \frac{k-1}{n}) \le 
e^{-\frac{1}{n} - \frac{2}{n} - \ldots - \frac{k-1}{n}} = 
e^{ \frac{k(k-1)}{2n} }
$$

Таким образом получли верхнюю оценку (правая часть неравенства).

Попробуем получить нижнюю:


Воспользуемся следующим соображением:

$ (1-x)(1-y) = 1-x-y+xy > 1-x-y $ \\
$ (1-x)(1-y)(1-z) > 1-x-y-z $ \\
$ (1-1/n)(1-2/n) > 1 - \frac{1}{n} - \frac{2}{n} \ldots - \frac{k-1}{n} 
= 1 - \frac{k(k-1)}{2n}$

Таким образом получли нижнюю оценку (левая часть неравенства).

\subsection{Теорема. Асимпототики при быстром росте k}

Все предыдущие оценки сделаны из соображения, что $ k $ 
растет намного медленнее $ n $

Рассмотрим случай, когда это не так:
%TODO [n \alpha] ближе к 1?
%\textbf{TODO [n \alpha] все же ближе к 1?}

Составим следующий биномиальный коэффициент:
$$
C_n^{[n \alpha]}
$$
Где $ \alpha \in (0, 1) $ и функция  $ [n \alpha] $ - целая часть аргумента $ n \alpha $. Например $([3.4] = 3) $

Или подробнее:
$$
C_{n}^{[n \alpha]} = \frac{n!}{(n - [n \alpha])! [n \alpha]!}
$$
В этом случае $ n $ и $ n \alpha $ растут примерно одинаково.

Для этой формулы хороший результат должна дать формула Стирлинта (\ref{stirling}), которая вообще хороша для случаек, когда $ n $ и $ k $ растут примерно одинаково.

Распишем биномиальный коэфиициент по формуле Стирлинга:
$$
% огромная формула!
\frac{n!}{(n - [n \alpha])! [n \alpha]!} \sim 
% числитель
\frac
{
\overbrace{ n^{n}e^{-n}\sqrt{2 \pi n} }^{n!} 
}
% знаменатель
{
\underbrace{ [n \alpha]^{[n \alpha]} e^{-[n \alpha]} \sqrt{2 \pi [n \alpha]} }_{[n \alpha]!} 
\underbrace{ (n - [n \alpha])^{n - [n \alpha]} e^{n + [n \alpha]} \sqrt{2 \pi (n - [n \alpha])} }_{ (n - [n \alpha])! } 
} = 
$$

Часть множителей сразу сокращаются, например, все $ e $ или $ \sqrt{2 \pi}  $
$$
% числитель
\frac{  n^{n} \sqrt{n} }
% знаменатель
{
[n \alpha]^{[n \alpha]} \sqrt{2 \pi [n \alpha]}  
(n - [n \alpha])^{n - [n \alpha]} \sqrt{ (n - [n \alpha])} 
} = 
$$

Функция "Целая часть" несколько усложняет сокращение.

Очевидно, $ \sqrt{n} $ или $ \sqrt{n \alpha} $ не являются существенными при оценке.

Наиболее существенными является множителиЖ
$ [n \alpha]^{[n \alpha]} $ и $ (n - [n \alpha])^{(n -[n \alpha])} $

Рассмотрим первый из них: 

Целая часть любого числа - это само число минус дробная часть.
$$
[k] = k - \epsilon
$$
? где $ \epsilon $ - дробная часть

Поэтому:
$$
[n \alpha]^{[n \alpha]} 
= (n \alpha - \epsilon)^{(n \alpha - \epsilon)}
= 
\underbrace{ (n \alpha)^{n \alpha - \epsilon}  }_{(n \alpha)^{n \alpha}}
\underbrace{(1 - \frac{\epsilon}{n \alpha})^{n \alpha - \epsilon}}_{(1 + o(1))^{n \alpha}} 
$$

Точная оценка эквивалентности этого выражения затруднена,
поэтому будем интересоваться его главной частью (см. примечания под выражением).
Чтобы применить эти выражения в оценке, небходимо поделить $[n \alpha]^{[n \alpha]} $ на какую-то функцию $ r_1(n) $, растущую не быстрее $ n $: $ 1 < r_1(n) < n $
$$
\frac{
(n \alpha)^{n \alpha} (1 + o(1))^{n \alpha}
}
{
r_1(n)
}
$$

Теперь рассмотрим выражение $ (n - [n \alpha])^{(n -[n \alpha])} $
К нему можно применить тот же прием - т.е. оценить главную часть.

Только в данном случае нужно не поделить, а умножить н какую-то функцию $ r_2(n) $, устроенную так же. как и $ r_1(n) $: 
$ 1 < r_2(n) < n $

Умножать нужно, т.к. левая часть равенства с показателем 
$ (n -[n \alpha]) $ больше превой с показателем $ (n - n \alpha)$.
Это логично, ведь вычитание в показателе числа без дробной части дает число больше, чем при вычитании с дробной 
($ (n - [3.5]) > (n - 3.5)  $).

$$
(n - [n \alpha])^{(n -[n \alpha])} = (n - n \alpha)^{(n -n \alpha)} (1 + o(1))
$$

Соберем все части оцененного выражения вместе:
$$
\frac{n^{n} \sqrt{n} r_3(n)}
{(n \alpha)^{n \alpha} n(1 - \alpha)^{n(1 - \alpha)}} \cdot
\frac{1}{(1 + o(1)) \sqrt{2 \pi (n \alpha)(n - n \alpha)}}
$$
Во второй дроби собраны части, незначительно влияющие на оценку, причем функции целой части убраны как незначащие.

В числителе присуствует функция $ r_3(n) $, такая что
$$
r_3(n) = \frac{r_1(n)}{r_2(n)}
$$
т.е. находится примерно между $ 1/n $ и $ n $

Теперь избавимся от множителей, незначительно влияющих на оценку (внесем их в некую функцию $ r_4(n) $):
$$
\sim \frac{
n^{n} r_4(n)
}{
(n \alpha)^{n \alpha} (n ( 1 - \alpha ))^{n(1-\alpha)}(1+o(1))^{n}
} =
$$

Функция $ r_4(n) $ устроена следующим образом:
$$
\frac{C_1}{n \sqrt{n}} < r_4(n) < C_2 \sqrt{n}
$$
где $ C_1, C_2 $ - некоторые константы

Преобразуем выражение дальше. Сократим все $ n $ (разность их показателей в числителе и знаменатале равна 0)

Также выражение $ (1 + o(1))^{n} $ близко к $ 1 $, поэтому его можно свободно перекидывать из знаменателя в числитель 
(а функцию $ r_4(n) $ можно внести под знак $ o(1) $).

$$
= 
\frac {
(1 + o(1))^{n} r_4(n) 
}
{ 
(\alpha^{\alpha} (1 - \alpha)^{(1 - \alpha)})^{n}
} 
= 
\frac {
(1 + o(1))^{n}
}
{ 
(\alpha^{\alpha} (1 - \alpha)^{(1 - \alpha)})^{n}
} =
\left( 
\frac{1}{\alpha^{\alpha} (1 - \alpha)^{(1 - \alpha)} } + o(1)
\right)^{n} =
$$

Введем следующее обозначение, называемое \textbf{энтропией}:
$H( \alpha) = 
- \alpha \ln{ \alpha } 
- (1 - \alpha) \ln({1 - \alpha})$

Тогда полученную оценку можно записать как
$$
C_{n}^{[n \alpha]} = (e^{H(\alpha)} + o(1) )^n
$$

Т.е., мы поняли, с какой скоростью в среднем растет подобный биномиальный коэффициент - 
за его рост отвечает функция $ H(\alpha) $

\textbf{Замечание}

\textbf{Возможна ли такая запись?}
$$
C_{n}^{[n \alpha]} 
= (e^{H(\alpha)} + o(1) )^n
= e^{n H(\alpha)}(1 + o(1) )
$$

Оценим $ (1 + o(1) )^{n} $ и попробуем найти контрпример.

Например, выражение
$$
\left( 1 + \frac{1}{\sqrt{n}} \right)^{n} 
= 
\left( 
\left( 1 + \frac{1}{\sqrt{n}} \right)^{ \sqrt{n} } 
\right)^{ \sqrt{n} } 
$$
может очень быстро расти.

%TODO вообще непонятно, ведь 1/sqrt(n) расходится?
\textbf{TODO вообще непонятно, ведь 1/sqrt(n) расходится?}

Т.е., число, ненамного большее 1 в очень большой степени может быть очень большим, и верно следующее утверждение

$$
(e^{H(\alpha)} + o(1) )^n 
\neq
e^{n H(\alpha)}(1 + o(1) )
$$

\subsection{Теорема. Асимптотики мультиномиального коэффициента}
$$
n = n_1 + n_2 + \cdots + n_k
$$
,$ k $ фиксированное число, причем $ n_i $ ведет себя следующим образом:
$$
n_i \sim \alpha_i n, \alpha_i > 0
$$

Тогда мультиномиальный коэффициент
$$
\binom{n}{n_1, n_2, \ldots, n_k} 
=\left(
e^{H(\alpha_1, \alpha_2, \ldots, \alpha_k)} + o(1)
\right)^{n}
$$

где функция $ H $ устроена следующим образом:
$$
H(\alpha_1, \alpha_2, \ldots, \alpha_k) = 
- \alpha_1 \ln \alpha_1 
- \alpha_2 \ln \alpha_2 
- \ldots
- \alpha_k \ln \alpha_k 
$$

Примем без доказательства (доказывается также по формуле Стирлинга (\ref{stirling}) )

\section{Оценка сумм последовательных биномиальных коэффициентов}

Рассматриваться будет только верхняя оценка.

Рассмотрим сумму последовательных биномиальных коэффициентов
$$
\sum_{k=0}^{m} C_n^k
$$

Докажем, что
$$
\sum\limits_{k=0}^{m} C_n^k 
\le
C_n^m \frac{n+1-m}{n+1-2m}
$$

С комбинаторной точки зрения это возможность выбрать из $ n $ элементов не более чем $ m $ элементов.

Также, очевидно
$$
m \le n/2
$$

Для начала, возьмем и поделим два последовательно идущих биномиальных коэффициента:
$$
\frac{
C_n^k
}
{C_n^{k-1}
}
=
% числитель
\frac{
(k-1)!(n-k+1)!
}
% знаменатель
{k!(n-k)!
}
= \frac{n-k+1}{k}
= \frac{n+1}{k} - 1
$$

Теперь, используя полученное выражение, поделим последовательные суммы биномиальных коэффициентов.

%TODO откуда следует? какие комбинаторные соображения
\textbf{TODO откуда следует? какие комбинаторные соображения}

$$
\sum\limits_{k=0}^{m} C_n^k 
= C_n^m \sum\limits_{k=0}^{m} \frac{C_n^k}{C_n^m} 
= C_n^m \sum\limits_{k=0}^{m} \frac{C_{n}^{m-k}}{C_n^m} 
$$

$$
C_n^m \sum\limits_{k=0}^{m} \frac{C_{n}^{m-k}}{C_n^m} = 
C_n^m
\left( 
1 +
\frac{1}{((n+1)/m - 1)} +
\frac{1}{((n+1)/m - 1)((n+1)/(m-1) - 1)} +
\frac{1}{((n+1)/m - 1)((n+1)/(m-1) - 1)((n+1)/(m-2) - 1)} + \ldots
\right) <
$$

Очевидно, каждый последующий знаменатель больше предыдущего, поэтому каждое следующее слагаемое меньше. Т.е. мы увеличим выражение, если будем писать только первое слагаемое.

$$
C_n^m
\left( 
1 +
\frac{1}{(n+1/m-1)} +
\frac{1}{(n+1/m-1)^{2}} +
\frac{1}{(n+1/m-1)^{3}} + \ldots
\right) < 
$$

Будем считать. что это бесконечная геометрическая 
прогресия (от этого сумма выражения только увеличится).
$$
C_n^m
\frac{1}{1 + \frac{1}{(n+1)/m - 1}}
$$

Упрощаем
$$
C_n^m 
\frac{\frac{n+1}{m} - 1
}
{\frac{n+1}{m} - 2
} = 
C_n^m \frac{n+1-m}{n+1-2m}
$$

Доказано

\textbf{Замечание}

Данная оценка становится грубой, если $ m $ 
приближается к $ \frac{n}{2} $

\subsection{Теорема. Энтропийная оценка}
$$
\sum\limits_{k=0}^{m} C_n^k \le e^{nH(\frac{m}{n})}
$$
$$
m \le n/2
$$

Доказательство

Возьмем уже известную нам сумму последовательных биномиальных коэффициентов и оценим ее сверху.
Для этого выберем некоторую $ t \le 1 $ 
$$
\sum\limits_{k=0}^{m} C_n^k \le \sum\limits_{k=0}^{m} C_n^k t^{k-m}
$$
Неравенство верно, т.к. для каждой $ t^{k-m} $
показатель степени $ (k-m)<0 $

Вынесем $ t^{-m} $ за знак суммы
$$
\sum\limits_{k=0}^{m} C_n^k t^{k-m}
= t^{-m} \sum\limits_{k=0}^{m} C_n^k t^{k} \le
$$

А выражение $ C_n^k t^{k} $ не что иное, как биномиальный коэффициент (если складывать не до
$ m $, а до $ n $, получаем бином Ньютона)

$$
\le t^{-m} (1 + t)^{n}
$$

$ t $ подберем так: $t = \frac{n}{n-m}$ 

$$
1+t = \frac{n}{n-m} 
$$

Тогда, раскрыв $ t $
$$
t^{-m} (1 + t)^{n} = 
% ()
\left( 
\frac{m}{n-m}
\right)^{-m}
% умножить на ()
\left( 
\frac{n}{n-m}
\right)^{n} = 
% =
\frac{ n^{n}
}{ m^{m} (n-m)^{n-m}
} =
$$
$$
% =
= \left(
\frac{1}{ \frac{m}{n}^{\frac{m}{n}} 
% *
(1-\frac{m}{n})^{1 - \frac{m}{n}} }
\right)^{n} =
e^{nH(\frac{m}{n})}
$$









\chapter{ Лекция 6 }
\section{Производящие функции}

\begin{description}
\item[Рекомендуемая литература]~ 

1. Ландо "Лекции о производящих функциях" \\
2. Грэхем, Кнут. Паташник. "Конкретная математика" \\
\end{description}

\subsection{Основные понятия}

Пусть имеется некая последовательность чисел:
$$
a_0, a_1, a_2, 3 a_3, \ldots
$$

Каждой такой последовательности формально можно сопоставить степенной ряд
$$
A(t) = a_0 + a_1 t + a_2 t^2 + a_3 t^3 + \ldots = \sum\limits_{n=0}^{\infty} a_{n} t^{n}
$$

Ряд не обладает никаким набором специфических свойств, он не обязан сходится и т.п.
Единственное обязательное свойство:
$$
A(0) = a_{0}
$$

Тогда объект $ A(t) $ называется \textbf{производящей функцией}.

Если заданы несклько производящих функицй, можно определить операции сложения и умножения
$$
A(t) = a_0 + a_2 t^2 + a_3 t^3 + \ldots
$$
$$
B(t) = b_0 + b_2 t^2 + b_3 t^3 + \ldots
$$

Сумма:
$$
A(t) + B(t) = (a_0+b_0) + (a_1+b_1) t + (a_2+b_2) t^2 + (a_3+b_3) t^3 + \ldots
$$
Произведение производящих функций является производящей функцией свёртки:
$$
A(t) B(t) = c_0 + c_2 t^2 + c_3 t^3 + \ldots = \sum_{n=0}^\infty c_n t^n
$$

где
$c_0 = a_0 b_0$	\\
$c_1 = a_0 b_1 + a_1 b_0 $ \\
$c_2 = a_0 b_2 + a_1 b_1 + a_2 b_0  $\\
$c_3 = a_0 b_3 + a_1 b_2 + a_2 b_1 + a_0 b_3$\\
$\ldots$\\

В целом:
$$
c_n = \sum_{k=0}^n a_k b_{n-k}
$$

Если мы раскроем скобки, то получится группировка степеней при соотетствующих t.

В некоторых случаях можно говорить о композиции производящих функций:
$$
A(B(t)) = a_0 + a_1 B(t) + a_2 B(t)^{2} + a_3 B(t)^{3} + \ldots =
$$
группируем по степеням $ t $
$= a_{0} + a_{1}b_{1}t $ \\
$+ (a_{1}b_{2} + a_{2}b_{1}^{2} ) t^{2} + $
$+ (a_{1}b_{3} + a_{2}2 b_{1} b_{2} + a_{3}b_{1}^{3}) t^{3} + $ \\
$\ldots$
Необходимое условие композиции: $B(0) = 0$

\textbf{Пример:}
$$
B(t) = -t; A(B(t)) = a_{0} - a_{1}t + a_{2} t^{2} - a_{3} t^{3} + \ldots
$$
$$
B(t) = t^{2}; A(B(t)) = a_{0} + a_{1} t^{2} + a_{2} t^{4} + \ldots
$$

\subsection{Теорема}
$\exists! B(t) : b_{0} = 0, b_{1} \neq 0$

Тогда $\nexists A(t), C(t) :  A(B(t)) = t, B(C(t)) = t$

Т.е. найдуться такие $ A(t), C(t) $, что их композиции равны $ t $

В некоторых случаях $ A(t) $ и $ C(t) $ могут совпадать.

Исходим из соображений, что B(t) известно, ищем $ A $ и $ C $ таким образом, чтобы 
их композиция с $ B $ оказалась равна $ t $.
Для этого достаточно, чтобы ненулевой коэффициент при $ t \neq 0$, а при всех остальных коэффициентах он должен обращаться в 0.

Попробуем подобрать прямо из формулы композиции:

$A(B(t))= a_{0} + a_{1}b_{1}t $ 
$+ (a_{1}b_{2} + a_{2}b_{1}^{2} ) t^{2} $
$+ (a_{1}b_{3} + a_{2}2 b_{1} b_{2} + a_{3}b_{1}^{3}) t^{3} + $ \\
$\ldots$

Сразу подбираем $ a_{0} = 0 $ 

Если $ b_{1} \neq 0 $, то $ a_{1} $ найдется: $a_{1} = \frac{1}{b_{1}}$

Второе выражение: $a_{1}b_{2} + a_{2}b_{1}^{2} = 0$. 

$ a_{1}, b_{1}, b_{2} $ известны,
$ a_{2} $ найдется как решение линейного уравнения и т.д. - последнее слагаемое
всегда будет иметь вид $ a_{n} b_{1}^{n} $, а предыдущие коэффициенты будут 
к этому моменту известны.


\subsection{Теорема}

$\exists! A(t) : a_{0} \neq 0$
Тогда $ \exists! B(t): A(t)B(t) = 1 $

Доказательство:

Мы знаем соотношения коэффициентов $ c $ в произведении функций:
$c_0 = a_0 b_0$	\\
$c_1 = a_0 b_1 + a_1 b_0 $ \\
$c_2 = a_0 b_2 + a_1 b_1 + a_2 b_0  $\\
$c_3 = a_0 b_3 + a_1 b_2 + a_2 b_1 + a_0 b_3$\\
$\ldots$\\

Для того, чтобы $ A(t)B(t) = 1 $ достаточно чтобы $ a_0 b_0 = 1 $ 
(находим $ b_{0} = \dfrac{1}{a_{0}} $)

Остальные же коэффициенты должны равняться 0, 
для этого последовательно находим коэффиициенты при $ b_{i} $ решением линейным уравнений.

Доказано

\subsection{Дифференцирование производящих функций}

Если есть производящая функция
$$
A(t) = a_0 + a_1 t + a_2 t^2 + a_3 t^3 + \ldots
$$

То формально ее можно продифференцировать:
$$
A(t) = a_1 + 2 a_2 t + 3 a_3 t^2 + 4 a_4 t^3 + \ldots
$$
А можно проинтегрировать:
$$
\int A(t) dt  = a_0 t + \frac{a_1 t^2}{2} + \frac{a_2 t^3}{3} + \frac{a_3 t^4}{4} + \ldots
$$

Несложно понять, что производная интеграла дает исходный ряд:
$$
(\int A(t) dt )' = A(t)
$$

Если ряды производящих функций сходятся, можно использовать соотвествующий аппарат матанализа.

\textbf{Пример:}

$ e^t = 1 + \frac{t}{1!} + \frac{t^{2}}{2!} + \frac{t^{3}}{3!} +\ldots$ \\

Этот ряд хорош тем что он всегда (и довольно быстро) сходится

Можно вспомнить тригонометрические функции (тоже всега сходятся)
$$
\sin t = \frac{t}{1!} - \frac{t^{3}}{3!} + \frac{t^{5}}{5!} - \ldots
$$
$$
\cos t = 1 - \frac{t^{2}}{2!} + \frac{t^{4}}{4!} - \ldots
$$

Еще несколько примеров рядов:

Биномиальный ряд (сходится не всегда)
В случае p натурально, он становится биномом, в обратном случае получается ряд

$$
(1+t)^p = 1 + pt + \frac{p(p-1)}{2!}t^{2} + \frac{p(p-1)(p-2)}{3!}t^{3} + \ldots
$$

$$
ln( \dfrac{1}{1-t}) = t + \dfrac{t^{2}}{2}+ \dfrac{t^{3}}{3} + \dfrac{t^{4}}{4} + \ldots
$$

Оперируя рядами, можно проверять различные выкладки, например 

$ e^{t} e^{-t} = 1 $ \\
$(1+t)^p(1+t)^q = (1+t)^{p+q}$ \\
$\sin{t}^2 + \cos{t}^2 = 1$ \\

Т.е., доказать эти равенства не для функции, а для формальных степенных рядов.

\subsection{Пример}

Простейшая последовательность: 
$1,1,1,\ldots$

$A(t) = 1+t+t^{2}+t^{3}+ \ldots = \dfrac{1}{1-t}$

\subsection{Пример. Геометрическая прогрессия}

$b,bq,bq^{2},bq^{3},\ldots$

$b + bqt + bq^2t^2 + bq^3t^3 + \ldots (1 + qt + (qt)^2 + (qt)^3 + \ldots) + \ldots 
= \dfrac{b}{1 - qt}$

\section{Использование производящих функций}

\subsection{Числа Фибоначчи}

$F_{n+2} = F_{n+1} + F_{n}$ \\
$0,1,1,2,3,5,8,13,21,34,55,89,144, \ldots$ \\

Производящая функция для чисел фибоначчи

\begin{equation}
\label{fib1}
Fib(t) = F_0 +  F_1 t + F_2 t^2 + \ldots
\end{equation}

Воспользуемся приведенным рекуррентным соотношнием для выяснения 
каких-либо свойств производящей функции.

Например, умножим на $ t^{2} $
\begin{equation}
\label{fib2}
t^2 \cdot Fib(t) = F_0 t^2 +  F_1 t^3 + F_2 t^4 + F_3 t^5 + \ldots
\end{equation}

Ту же самую производящую функию умножим на $ t $
\begin{equation}
\label{fib3}
t \cdot Fib(t) = F_0 t +  F_1 t^2 + F_2 t^3+ F_3 t^4 + \ldots
\end{equation}

В выражениях одно под другим приведены коэффициенты при определенных степенях $ (2,3,\ldots) $

Т.е., в формуле (\ref{fib1}) при коэффициентах с одинаковым показателем при $ t $ приведена сумма 
(\ref{fib2}) и (\ref{fib3})

Также получили формулу
$$
(t + t^2) Fib(t) = Fib(t) - t
$$
Разрешив относительно $ Fib(t) $, получаем
$$
Fib(t) = \dfrac{-t}{1-t-t^2}
$$

Т.е., некая рациональная функция, которую можно расписать на простейшие.

Решим квадратное уравнение

$t^2 + t + 1 = 0$

Его корни:

$\phi_{1,2} = \dfrac{-1 \pm \sqrt{5}}{2}$

Таким образом, разложение производящей функции на простейшие выглядит так:

$ Fib(t) = \dfrac{-t}{(t-\phi_1)(t-\phi_2)} =$

$ = 
\dfrac{\alpha}{(1-\phi_2)}  
+
\dfrac{\beta}{(1-\phi_1)}$ \\

$ -t = \alpha(t - \phi_1) + \beta (t - \phi_2)$ \\

Числа $ \alpha, \beta $, стоящие в числителе, можно найти, воспользовавшись 
методом неопределённых коэффициентов 
(представление  функции в виде в виде суммы элементарных дробей).

Должны совпасть коэффициенты при $ t^{1} $ и $ t^{0} $, поскольку многочлены равны


$\begin{cases}
- 1 = \alpha + \beta \\
0 = - \alpha \phi_2 - \beta \phi_1
\end{cases}$

% найдем бета
$\beta = 
\dfrac{-\alpha \phi_2}
{\phi_1} $ 

% подставим его
$ -1 = \alpha - \dfrac{\alpha  \phi_2}{\phi_1} = $ 

$ = \alpha (1 - \frac{\phi_1}{\phi_1})  $

$ = \alpha \left( \dfrac{\phi_1 - \phi_2}{\phi_1}  \right)$ \\

Вспомним, что $ \phi_1, \phi_2 $ - корни квадратного уравнения, и их разность
$ \phi_2 - \phi_1 = \sqrt{5}$


$\alpha = - \dfrac{\phi_1}
{\phi_1 - \phi_2} 
=
- \dfrac{\phi_1}
{\sqrt{5}}
$

$\beta = 
\dfrac{\phi_2}
{\phi_1 - \phi_2} 
=
\dfrac{\phi_2}
{ \sqrt{5} } 
$ 


Подаставляя в формулу для чисел Фибоначчи, получаем:


$$
Fib(t) =
\dfrac{\alpha}{(1-\phi_1)}
+
\dfrac{\beta}{(1-\phi_2)} 
=
\dfrac{ - \phi_1}{\sqrt{5} (t-\phi_1)}  
+
\dfrac{\phi_2}{\sqrt{5} (t-\phi_2)} 
=
\dfrac{1}{\sqrt{5}} 
\left( 
\dfrac{-1}{1-\frac{t}{\phi_2} }
+
\dfrac{1}{1-\frac{t}{\phi_1} }
\right)=
$$

$$
 = \dfrac{1}{\sqrt{5}} \sum\limits_{n=0}^{\infty}
( \frac{-1}{\phi_2^n} + \frac{1}{\phi_1^n} ) t^{n}  =
$$

Рассмотрим подробнее $ \phi_1 $ и $ \phi_2 $

По теореме Виета:
$ \phi_1 \phi_2  = 1$

Поэтому 

$$
= \frac{1}{\sqrt{5}} \sum\limits_{n = 0}^{\infty} 
( (- \phi_1)^{n}  + ( -\phi_1)^{n}) t^n 
= 
\sum\limits_{i = 0}^{\infty}
\left(
(\frac{1 + \sqrt{5}}{2})^n - (\frac{1 - \sqrt{5}}{2})^n 
\right) t^n
$$

В итоге, найдена формула для чисел Фибоначчи выглядит следующим образом 
(носит название формулы Бинэ)

\begin{equation}
\label{bine}
F_n = \sum\limits_{i = 0}^{\infty}
\left(
(\frac{1 + \sqrt{5}}{2})^n - (\frac{1 - \sqrt{5}}{2})^n
\right) t^n
\end{equation}

Проведем небольшую оценку.

Рассмотрим сторое слагаемое, стоящее под знаком суммы.

Заметим, что оно довольно быстро стремится к нулю (начиная с $ n = 2 $ оно меньше $ 1/2 $).

$\lim\limits_{n \to \infty} (\frac{1 - \sqrt{5}}{2}) = 0$

Отсюда мы сразу можем написать асимптотику - 
полученная формула будет эквивалентна более простой:
$$
F_n = \sum\limits_{i = 0}^{\infty}
\left(
(\frac{1 + \sqrt{5}}{2})^n - (\frac{1 - \sqrt{5}}{2})^n
\right) t^n
\sim
\left(
\frac{1 + \sqrt{5}}{2}
\right) ^ n
$$

Т.е., это рациональное число, очень близкое к какому-то целому. Это целое и будет n-м числом Фибоначчи.

Формулу можно использовать для различных приложений, например суммы $ n $ чисел Фибоначчи (сводится к геометрической прогрессии):

$$
F_1 + F_2 + F_3 + \ldots 
$$

\subsection{Теорема}

$$
\begin{pmatrix}
0 & 1 \\
1 & 1
\end{pmatrix}^{n}
=
\begin{pmatrix}
F_{n-1} & F_{n} \\
F_{n} & F_{n+1}
\end{pmatrix}
$$
Доказательство (по индукции):

Верно для $ n=0, n=1 $

Проверим для n = n + 1

$$
\begin{pmatrix}
0 & 1 \\
1 & 1
\end{pmatrix}^{n}
\cdot
\begin{pmatrix}
F_{n-1} & F_{n} \\
F_{n} & F_{n+1}
\end{pmatrix}
=
\begin{pmatrix}
F_{n} & F_{n+1} \\
F_{n-1} + F_{n} & F_{n} + F_{n+1}
\end{pmatrix}
=
\begin{pmatrix}
F_{n} & F_{n+1} \\
F_{n+1} & F_{n+2}
\end{pmatrix}
$$

\textbf{Пример:}

$$
\begin{pmatrix}
0 & 1 \\
1 & 1
\end{pmatrix}^{2n}
=
\left(
\begin{pmatrix}
0 & 1 \\
1 & 1
\end{pmatrix}^{n}
\right)^{2}
=
\begin{pmatrix}
F_{2n-1} & F_{2n} \\
F_{2n} & F_{2n+1}
\end{pmatrix}^{2}
= 
\begin{pmatrix}
F_{2n-1}^{2} + F_{n}^{2} & F_{n-1} F_{n} + F_{n} F_{n+1} \\
F_{n-1} F_{n} + F_{n} F_{n+1} & F_{n}^{2} + F_{n+1}^{2}
\end{pmatrix}
$$

Из данной матричной формулы можно получить довольно много полезных соотношений.
$$
F_{n}^{2} + F_{n+1}^{2}  = F_{2n+1}
$$
$$
F_{n}( F_{n-1} + F_{n+1} )= F_{2n}
$$

\section{Линейные рекуррентные соотношения}

Исследуем последовательности вида:

$$
a_{n+k} = c_1 a_{n+k-1} + c_2 a_{n+k-2} + \ldots + c_k a_n
$$
т.е., в которых значение $ n+k $ элемента последовательности 
зависит от $ k $ предыдущих занчений.

Соответственно, необходимо задать хотя бы $ k-1 $ 
начальных значений для рекурсии глубины $ k $

Для ЛРС мы тоже можем рассмотреть производящую функцию:

$$
A(t) = a_0 + a_1 t + a_2 t^2 + \ldots
$$

и проделать следующие действия:

домножить ее на $c_1 t$: $ c_1 t A(t) $ \\
домножить ее на $c_2 t^2$: $ c_2 t^{2} A(t) $ \\
домножить ее на $c_3 t^3$: $ c_3 t^{3} A(t) $ \\
$\ldots$ \\
домножить ее на $c_{k} t^{k}$: $ c_k t^{k} A(t) $ \\

И потом сложить все полученные соотношения.

Посмотрим, как будет устроен коэфициент при $ t^{n+k} $

В выражении $ c_1 t A(t) $ такой показатель можно получить, 
если взять из A(t) $ n+k-1$-й член ряда: \\
($ c_{1} a_{n_k-1} $).

В выражении $ c_2 t^{2} A(t)$ такой показатель можно получить, 
если взять из A(t) $ n+k-2$-й член ряда: \\
($ с_{2} a_{n_k-2} $).

И т.д. до $ n$-го члена: \\
($ c_{k} a_{n} $).

Полученные значения сложим, и по рекуррентному соотношению 
получим $ a_{n+k}$-й коэффициент, т.е. тот самый, который должен быть 
в производящей функции $ A(t) $:

$(c_0 + c_1 t + c_2 t^2 + \ldots + c_k t^k) A(t) = A(t) - P(t)$

$ P(t) $ - какой-то полином, необходимый для того, чтобы при $ t^{n+k} $
по крайней мере выполнялось условие $ n =0 $.

Т.е. степень полинома $ P(t) $ меньше $ k $.

Выведем саму производящую функцию:

$0 = A(t) - P(t) - A(t)(c_0 + c_1 t + c_2 t^2 + \ldots + c_k t^k)$

$A(t)(1 - (c_0 + c_1 t + c_2 t^2 + \ldots + c_k t^k) ) = P(t)$

$$
A(t) = \dfrac{P(t)}{Q(t)}
$$

Где $Q(t) = 1 - c_1 t - c_2 t^2 - \ldots - c_k t^k$

В данном случае производящая функция будет рациональной, причем
степень числителя меньше степени знаменателя.

Далее следует разложить функцию на простейшие дроби.

Пусть функция в общем виде:
$Q(t) = (1 - \alpha_1 t)^{m_1} (1 - \alpha_2 t)^{m_2} (1 - \alpha_n t)^{m_n}$

$1/\alpha_1, 1/\alpha_2, \ldots, 1/\alpha_n - $ корни данного многочлена 
(в общем случае могут быть и комплексными)

Тогда разложение производящей функции в общем виде:

$$
\frac{P(t)}{Q(t)} = \sum\limits_{j=1}^{n} 
\left(
\frac{b_{j_1}}{1-a_j t} 
+ \frac{b_{j_2}}{(1-a_j t)^{2}} 
+ \frac{b_{j_3}}{(1-a_j t)^{3}}
+ \ldots 
+ \frac{b_{j_m}}{(1-a_j t)^{m_j}}
\right)
$$
где $ b_{j_i} $ - какие-то коэффициенты, 
которые находятся решением соответсвующего линейного уравнения.

Рассмотрим слагаемые от 1 до $ m $

1. $ \frac{1}{1-a t} = \sum\limits_{n=0}^{\infty} {a}^{n} {t}^{n}$ - геометрическая прогрессия 

2. $ \frac{b_{j_2}}{(1-a_j t)^{2}} = \frac{1}{\alpha} \left( \frac{1}{(1-a t)} \right)' $ - очевидно, несколько преобразованная производная предыдущего слагаемого\\

3. То же самое со второй производной, и т.д.

\subsection{Пример}

Пусть имеется неограниченный запас монет достоинством 1,2,5,10 рублей.

Мы хотим узнать, каким количеством способов $ a_n $ можно сумму $ N $
представить в виде суммы доступных монет?

Найдем производящую функцию для этой последоватльности:

Начнем со случая, когда имеется только одна разновидность монет - 1 рубль.

Тогда количество способов заплатить сумму $ N $ рублей ровно одно:

Поэтому производящая функция для этой последоватльности $1,1,1,1,\ldots$:

$A_1(t) = 1 + t + t^{2} + t^{3} + \ldots = 1/(1-t)$ 
- т.е. геометрическая прогрессая со степенью 1\\

Допустим, имеется одна разновидность монет - только 2 рубля.

Теперь, если сумма нечетна, мы можем заплатить 0 способами, 
если четна - единственным способом $2,2,2,2,\ldots$

Тогда
$A_2(t) + = 1/(1-t^2)$  \\

Перемножив две получившиеся функции, заметим, что коэффициент при $ t^{n} $ равен количеству способов представить $n = 2a + b: a,b \ge 0$

Т.е., если имеется 2 вида монет - 1 и 2 рубля.

$A_{1,2} = A_1 A_2 = \frac{1}{1 - t} \frac{1}{1 - t^{2} }$ \\

По аналогии:

$A_{1,2,5} = A_1 A_2 A_5 = \frac{1}{1 - t} \frac{1}{1 - t^{2}} \frac{1}{1 - t^{5}}$ \\

$A_{1,2,5,10} = A_1 A_2 A_5 A_{10} = \frac{1}{1 - t} \frac{1}{1 - t^{2}} \frac{1}{1 - t^{5}} \frac{1}{1 - t^{10}}$ \\

\subsection{Общий пример}

Пусть имеется некое множество, состоящее из натуральных чисел
$ H \subset \mathbb{N}\setminus\{0\} $ \\

Обозначим через 
$p(H,n)$ количество способов представить $ n $ в виде суммы слагаемых, содержащихся в множестве $ H $\\

В приведенном выше случае  $ H = \{1,2,5,10\}$ \\

Можем найти производящую функцию для подобной последовательности 
(т.е., оббобщение приведенного примера)

$P(t) = \sum\limits_{k \subset H}^{} p(H,n)t^n $

В частности, производящая функция просто числа разбиений - это следующее произведение

$ P(t) = \prod\limits_{k=1}^{\infty} \frac{1}{1-t^k}$ \\

Если же мы вводим ограничение, что каждый элемент мы можем использовать не более $ d $ раз, их производящие функции

Только для 1:\\
$P(t) = \prod\limits_{k=1}^{\infty} ( 1 + t + t^{2} + ... + t^{d} )$ \\

Только для 2:\\
$P(t) = \prod\limits_{k=1}^{\infty} ( 1 + t^2 + t^{4} + ... + t^{2d} )$ \\

И в общем случае:

$P(t) = \prod\limits_{k=1}^{\infty} ( 1 + t^k + t^{2k} + ... + t^{dk} )$ \\

\subsection{Теорема Эйлера. Доказательство с помощью производящих функций}

Формулировка: количество способов представить $ N $ в виде суммы различных слагаемых равно количеству способов представить в виде нечетных.

(В предыдущей главе теорема была доказана комбинаторно).

Если каждое слагаемое используется неболее 1 раза, производящая функция разбиения:

$P(t) = \prod\limits_{k=1}^{\infty}  (1+t^k)$ \\

Если используется нечетное число слагаемых, производящая функция разбиения:

$ P(t) = \prod\limits_{k=1}^{\infty}  \frac{1}{1-t^{2k-1}} $

Доказать:
$$
\prod\limits_{k=1}^{\infty}  (1+t^k) = \prod\limits_{k=1}^{\infty}  \frac{1}{1-t^{2k-1}}
$$

Заметим, что в выражении

$ \prod\limits_{k=1}^{\infty}  (1+t^k) =
\dfrac
{\prod\limits_{k=1}^{\infty} (1-t^{2k}) }
{\prod\limits_{k=1}^{\infty} (1-t^k) }$ 

Т.е., в числителе только четные показатели степени, в знаменателе - любые.

Если их последовательно разделить, остануться только нечетные, потому что все четные сократятся, т.е. каждый множитель сокражается как $ \frac{(1-t^{k})(1+t^{k})}{(1-t^{k})} $

А это и есть исходное равенство.

\subsection{Метод перевала. Формула Харди-Рамануджана}

Если ряд производящей функции сходится, то работа с ней сильно упрощается.
Например, если есть сходящийся ряд:

$$
\sum\limits_{n = 0}^{\infty} a_n t^n
$$

Коэффициент при n-й степени находится при помощи интегральной функции.

$$
a_n  = \frac{1}{2 \pi i} \int \frac{F(t)}{t^{n+1}} dt
$$

Интеграл, охватывающий контур точки 0.

Если есть интеграл, асимптотику последовательности можно найти гораздо проще.
Существуют общие методы. позволяющие оценивать асимптотику подобных последовательностей.
Например, \textbf{формула Харди-Рамануджана}

$$
P(n) \sim \dfrac{1}{4 n \sqrt{3}} e^{\pi \sqrt{\frac{2n}{3}}}
$$
где $ P(n) $ - общее число разбиений $ n $, для которого 
$ F(t) = \prod\limits_{k = 1}^{\infty} \dfrac{1}{1 - t^{k}} $

Все примем без доказательства.


\subsection{Производящая функция бинома}

Производящая функция бинома:
$$
(1+t)^n = \sum_{k=0}^n C_n^k t^k 
$$

$$
\frac{1}{(1-t)^{n+1}} = \sum\limits_{n+k}^{n} t^k 
$$


\chapter{ Лекция 7 }
\section{Числовые последовательности}


\subsection{Числа Каталана}

Пусть имеется правильная скобочная последовательность: \\
\textbf{(()())()(())}

Причем так, что в каждый момент открывающихся скобок не менее чем закрывающихся, 
т.е. можно разбить на пары \textbf{()}, которые не пересекаются.

Найдем рекуррентное соотношение и производящую формулу для подобных скобочных структур.

пусть $ C_{n} $ - количество правильных скобочных структур для $ 2n $ скобок 
($ n $ открывающихся, $ n $ закрывающихся)

Считаем, что $ C_{0} = 1 $

Выразим $ C_{n+1} $

Для каждой открывающей скобки есть соответствующая ей открывающая. 
Тогда если мы воззьмем по отдельности этот набор(0), оставшийся после него(1), и оставшийся после удаления из него двух крайних скобок(2).

$ \overbrace{( \underbrace{(()()())}_{(1)} )}^{(0)} \overbrace{(()())}^{(2)} $

Все они будут правильными скобочными последовательностями.

Если в (1) $ k $ открывающихся скобок, то в (2) $ n-k $ открывающихся скобок.

Т.е. в (1) имеется $ C_{k} $ способов расставить скобки, а в (2) - $ C_{n-k} $

Во всей конструкции получается $ C_{k} C_{n-k} $ способов.
А т.к. разбить конструкцию на (1) и (2) мы можем в любом месте,
для получения рекуррентного соотношения надо просто сложить 
все возможные случаи разбиений

$$
C_{n+1} = C_0 C_n + C_1 C_{n-1} + C_2 C_{n-2} + \ldots + C_{n-1}C_1 + C_n C_0
$$

$$
C_{n+1} =  \sum\limits_{i = 0}^{n} C_i C_{n-i}
$$

А начальное число $ C_{0} = 1 $

Последовательность, удовлетворяющая приведенному рекуррентному соотношению,
называется \textbf{числами Каталана} 

%TODO таблица

Как видно, числа растут очень быстро.

\subsection{Комбинаторные представление чисел Каталана.}

\subsubsection{Триангуляции}

Рассмотрим некий $n+2$-угольник, и рассмотим все его триангуляции.

%TODO картинка

Оказывается, количество триангуляций $n+2$-угольника является в точности $n$-м числом Каталана.

Это верно, т.е. между триангуляциями и скобочными последовательностями существует взаимно однозначное соответсвтие (при разбиениеии $ n+2$-угольника получаем 
с одной стороны $ C_{k} $ триангуляций, с другой - $ C_{n-k} $ триангуляций).

\subsubsection{Непересекающиеся хорды на окружности}

%TODO картинка

Для 6 точек - 5 способов провести непересекающиеся хорды.

Соответственно, для $ 2n $ точек - $ n $ способов провести непересекающиеся хорды.

При разбиениеии $2n+2$-точечной окружности получаем с одной стороны $ C_{k} $ 
способов провести хорды, с другой - $ C_{n-k} $ способов).

\subsubsection{Графы без циклов}

%TODO картинка

Количество графов без циклов (неизоморфных корневых деревьев) с $ n+1 $ узлами
также является $n$-м числом Каталана.

Пусть имеется одна вершина (корень).
Всего $ n+2 $ вершины.
Произведем разбиение, что удаляется одно из ребер 
таким образом, что в оставшемся дереве остается $ k+1 $ вершина с одной стороны, 
а с другой - $ n - k + 1 $.
Тогда с одной стороны имеется $ C_{k} $ вариантов, с другой - $ C_{n-k} $ вариантов

\subsection{Производящая функция чисел Каталана}

Производящая функция выглядит так:

$$
Cat(t) = c_{0} + c_{1} t^{} + c_{2} t^{2} + \ldots 
$$

Возведем в квадрат:

$Cat(t)^{2} = c_0^2 + (c_1 c_0 + c_0 c_1)t +(c_2 c_0 + c_1 c_1 + c_0 c_2)t^2
+ (c_3 c_0 + c_2 c_1 + c_1 c_2 + c_0 c_3)  t^3 + \ldots $

Теперь воспользуемся рекуррентной формулой:

$ c_{1}= c_{2} t^{} + c_{3} t^{2} + \ldots $

Составим следующее соотношение.
$$
t Cat(t)^2 = Cat(t)-1
$$

Очевино, это квадратное уравнение относительно функции Каталана

$$
Cat(t) = \dfrac{1 \pm \sqrt{1 - 4 t}}{2 t}
$$

Правильный знак '-', иначе в 0 объект не определен. 

$$
\sum_{n=0}^{\infty} C_n t^n = \frac{1-\sqrt{1-4 t}}{2 t}
$$

Напишем формулу Тейлора для этого объекта:

$(1+x)^p = 1 + px + \dfrac{p(p-1)}{2!} x^{2} + \dfrac{p(p-1)(p-2)}{3!} x^{3} + \ldots$

$\dfrac{1-\sqrt{1-4 t}}{2 t} = \frac{1}{2t} (1 - (1-4 t)^{1/2} )$

Распишем $ (1-4 t)^{1/2} $ по формуле Тейлора:

$(1-4 t)^{1/2} = $ \\
$\dfrac{1}{2t}$
$ ( 1 -  $
% разложение в ряд Тейлора начинается здесь
$(1 $
$ + \dfrac{1}{2}(-4t) + $ \\ \\
$ + \dfrac{1/2 (1/2 - 1)}{2!} (-4t)^{2} + $ \\ \\
$ + \dfrac{1/2 (1/2 - 1)(1/2 - 2)}{3!} (-4t)^{3} + $ \\ \\
$ + \dfrac{1/2 (1/2 - 1)(1/2 - 2)(1/2 - 3)}{3!} (-4t)^{4} $ 
$ + \ldots ) = $ \\

Знак при каждом слагаемом всегда будет '+'

$ = \dfrac{1}{2t}(2t + $
$ + \dfrac{2^{2}}{2!}t^{2}  $
$ + \dfrac{2^{3}}{3!} 1\cdot3 t^{3}  $
$ + \dfrac{2^{4}}{4!} 1 \cdot 3 \cdot 5 t^{4} ) + \ldots 
+ \dfrac{2^{k}}{k!} 1 \cdot 3 \cdot 5 \cdot \ldots (2k-3) t^{k} ) + \ldots = $

Сократим на $ 2t $. 
Кроме того, произведение всех нечетных чисел называется "двойной факториал",
и обозначается $ n!! $

$ = (1 + \dfrac{2 t}{2!} t  $
$ + \dfrac{2^{2}}{3!} 3!! t^{2}  $
$ + \dfrac{2^{3}}{4!} 5!! t^{3} ) + \ldots $
$+ \dfrac{2^{k-1}}{k!} (2k-3)!! $

Таким образом, формула для $ n-$го числа Каталана

$$
c_n = \frac{2^n}{(n+1)!} (2n-1)!! = \frac{C_{2n}^n}{n+1} 
$$

Докажем последнее равенство:

$\dfrac{C_{2n}^n}{n+1}  = $
$\dfrac{(2n)!}{(n+1)n!n!} = $
$\dfrac{(2n)!}{(n+1)!n!} = $ \\

Вынесем из числителя $ (2n)! $ все нечетные множители - 
получим произведение всех четных на все нечетные, т.е. на $ (2n-1)!! $

$= \dfrac{2 \cdot 4 \cdot 6 \ldots 2n}{n!} (2n-1)!!$ \\

В числителе все четные от 2 до $ 2n $

В знаменателе произведение всех чисел от 1 до $ n $

При взаимном сокращении получается как раз $ 2^{n} $

Т.е. имеет место формула
$$\dfrac{2 \cdot 4 \cdot 6 \ldots 2n}{n!} = 2^{n}
$$ 

Итоговую формулу можно записать и так:
$$
c_n = \dfrac{C_{2n}^n}{n+1} = C_{2n}^n - C_{2n}^{n-1}
$$

Последнее равенство показывает, что формула всегда дает целое число.

Докажем, почему это верно :
$$
\dfrac{C_{2n}^n}{n+1} = C_{2n}^n - C_{2n}^{n+1}
$$
$$
\dfrac{(2n)!}{(n+1)n!} = \dfrac{(2n)!}{n!n!} - \dfrac{(2n)!}{(n+1)(n-1)!}
$$

Делим на $ (2n)! $, домножаем на $ (n+1)n! $

$$
1 = n + 1 - n
$$

Доказано

\subsection{Асимптотика для чисел Каталана}


Скорость, с которой растут числа Каталана:

$$
\dfrac{C_{2n}^n}{n+1} 
\sim \dfrac{1}{n + 1} \dfrac{4^{n}}{\sqrt{\pi n}} 
\sim \dfrac{4^{n}}{\sqrt{\pi} n^{3/2}}
$$

\section{Числа Стирлинга}

\subsection{Числа Стирлинга второго рода. Комбинаторный смысл}

числом Стирлинга второго рода из n по k, обозначаемым S(n,k), или
$\begin{Bmatrix}
n \\
k
\end{Bmatrix}$
, называется количество неупорядоченных разбиений n-элементного множества на k непустых подмножеств.


Берем множество из $ n $ элементов 
$$
1,2,3,\ldots,n
$$
и считаем количество способов
разибть его на $ k $ непустых подмножеств.

%TODO спросить про пример

Пример:

$\begin{Bmatrix}
4 \\
2
\end{Bmatrix} = 7$

%TODO спросить про таблицу

Очевидно, количество способов разбить $ n $ на 1 равно 1 (первый столбец)

Также, количество способов разбить $ n $ на $ n $ равно 1 (диагональ)

На второй диагонали - количество способов разбить $ n $ на $ n-1 $ непустое подмножество. 
Т.е., все подмножества, ктоме одного - одноэлементны, значит, это количество способов выбрать 2-элементное подмножество, а именно $ C_{n}^{2} $

Довольно сложно устроено общее количество способов разбить $ n$-элементные множества на непустые подмножества (сумма по строкам).

Носит название \textbf{чисел Белла}, и растет очень быстро.

\subsubsection{Рекуррентная формула}

Возьмем число 
$\begin{Bmatrix}
n \\
k
\end{Bmatrix}$

Возьмем элемент с номером $ n $.

1 вариант. Будем считать его отдельным подмножеством, тогда останется
$\begin{Bmatrix}
n-1 \\
k-1
\end{Bmatrix}$ способов.

2 вариант. Не будем считать его отдельным подмножеством 
(он содержится в подмножестве), тогда останется
$k \begin{Bmatrix}
n-1 \\
k
\end{Bmatrix}$, где множитель $ k $ - количество способов включить 
этот элемент в любое из подмножеств.

Сведем в рекуррентную формулу:

$$
\begin{Bmatrix}
n \\
k
\end{Bmatrix} = 
\begin{Bmatrix}
n-1 \\
k-1
\end{Bmatrix} +
k \begin{Bmatrix}
n-1 \\
k
\end{Bmatrix}
$$

\subsection{Числа Стирлинга первого рода}

Числа Стирлинга первого рода --- количество перестановок порядка n с k циклами.

Обозначается
$\begin{bmatrix}
n \\
k
\end{bmatrix}$

Берем множество из $ n $ элементов 
$$
1,2,3,\ldots,n
$$
и разбиваем его не на подмножества, а на непересекающиеся циклы.

Например,
$(a,b,c),(b,c,a),(c,a,b)$ - циклы.

%TODO спросить про пример
\textbf{TODO спросить про пример}

Пример:

Т.е. $ \{1,2,3\} $ и $ \{1,3,2\} $ - разные циклы, но одинаковые множества.

А $ \{1,2\} $ и $ \{2,1\} $ - одинаковые множества и циклы.

Понятно, что чисел 1-го рода больше, чем 2-го, т.к. одному множеству 
может соответствовать много циклов.

А именно, множеству из $ k $ элементов, бeдет соответсвовать $ (k-1)! $ циклов 
(как соотношение упорядоченных и неупорядоченных множеств)

$\begin{Bmatrix}
4 \\
2
\end{Bmatrix} = 11$


%TODO спросить про таблицу
\textbf{TODO спросить про таблицу}
Ясно, что количество способов разбить $ n $ на $ n $ циклов всего один (первая диагональ).

Количество способов разбить на 2 цикла всего равно количеству 
разбить на 2 множества (вторая диагональ).

Количество способов сделать из числа цикл - $ (n-1)! $ (первый столбец)

Количество способов сделать из числа любое число циклов - $ n!$ (сумма по строкам)

\subsubsection{Рекуррентная формула}

Возьмем элемент с номером $ n $.

1 вариант. Будем считать его образующим собственный цикл, 
тогда останется $ n-1 $ элемент, из них нужно сделать $ k-1 $ циклов
$\begin{bmatrix}
n-1 \\
k-1
\end{bmatrix}$ способов.

2 вариант. Не будем считать его отдельным циклом
(он содержится в цикле), тогда останется
$ (n-1) \begin{bmatrix}
n-1 \\
k
\end{bmatrix}$, где множитель $ (n-1) $ - количество способов включить 
этот элемент в любqой из циклов (т.е. в любое место между $ n $ элементами).

Сведем в рекуррентную формулу:

$$
\begin{bmatrix}
n \\
k
\end{bmatrix} = 
\begin{bmatrix}
n-1 \\
k-1
\end{bmatrix} +
(n-1) \begin{bmatrix}
n-1 \\
k
\end{bmatrix}
$$

Совпадает во всем с числами Стирлинга 2-го рода, кроме множителя при $\begin{bmatrix}
n-1 \\
k
\end{bmatrix}$.

\subsection{Применение}

\subsubsection{Пример 1}

Возьмем некий многочлен n-й степени, который можно записать как сумму коэффициентов (минимальный коэффициент - 1).

Сумма коэффициентов при $ x^{k} $ - числа Стирлинга 1-го рода.

$x(x+1)(x+2)(x+3)\ldots(x+n-1) = \sum\limits_{k=1}^{n} \begin{bmatrix}
n-1 \\
k
\end{bmatrix} x^{k}$

Доказательство по инфукции:

Домножим выражение на $ (x+n) $:

$x(x+1)(x+2)(x+3)\ldots(x+n-1)(x+n) = $ \\
$= \sum\limits_{k=1}^{n} \begin{bmatrix} n-1 \\ k \end{bmatrix} x^{k} (x+n) = $\\

%TODO непонятно, откуда следует
\textbf{TODO непонятно, откуда следует}

$= \sum\limits_{k=1}^{n} 
\left(
\begin{bmatrix} n-1 \\ k \end{bmatrix} x^{k+1} + n \begin{bmatrix} n \\ k \end{bmatrix} x^{k} = 
\right)
$

$= \sum\limits_{k=1}^{n+1} 
\left(
\begin{bmatrix} n \\ k-1 \end{bmatrix} x^{k} 
+ n \begin{bmatrix} n \\ k \end{bmatrix} x^{k}
\right) = 
$

$= \sum\limits_{k=1}^{n+1} 
\begin{bmatrix} n+1 \\ k \end{bmatrix} x^{k} 
$

\subsubsection{Пример 2}

Поставим обратную задачу - пусть у нас есть $ x^{n} $, как его выразить через последовательные произведения.

\textbf{Примечание}

Краткое обозаняение для последовательного произведения вида
$$
x(x-1)(x-2)(x-3)\ldots(x-n+1) = x^{\underline{n}}
$$

, а для произведения вида
$$
x(x+1)(x+2)(x+3)\ldots(x+n-1) = x^{\overline{n}}
$$

Таким образом, 

$$
x^{\overline{n}} = \begin{bmatrix} n \\ k \end{bmatrix} x^{k} 
$$

А для обратной задачи верно следующее:
$$
x^{n} = \begin{Bmatrix} n \\ k \end{Bmatrix} x^{\underline{k}} 
= \begin{Bmatrix} n \\ k \end{Bmatrix} (-1)^{n+k} x^{\overline{k}} 
$$

Числа Стирлинга 1 и 2 рода обычно и применяются для связи 
подобных последовательных произведений со степенными выражениями.

\section{Числа Белла}

Числа Белла --- количество спобобов разбить на любые подмножества n-элементное множество.

$$
b_{n} = \sum\limits_{k=1}^{n} \begin{Bmatrix} n \\ k \end{Bmatrix}
$$

\subsection{Рекуррентная формула}

Выделим последний элемент $ n+1 $
Он входит в какое-либо подмножество.
Количество способов добавить к ним этот элемент и еще $ k $ элементов - $ C_{n}^{k} $
Т.е., создано подмножество, в которое заведомо входит элемент $ n+1 $

Количество спобобов разбить оставшиеся элементы на любые непустые подмножества.
Это $ n-k$-е число Белла.
Осталось просуммировать по $ k $ и свести в рекуррентную формулу.

\begin{equation}
\label{bell_recur}
b_{n+1} = \sum\limits_{k = 0}^{n} C_{n}^{k} b_{n-k} = \sum\limits_{k = 0}^{n} C_{n}^{k} b_{k} 
\end{equation}

\subsection{Экспоненциальная производящая функция}

Есть последовательность чисел
$$
a_0, a_1, a_2, \ldots 
$$

и формальный степенной ряд
$$
A(t) = a_0 + \dfrac{a_1}{1!}t + \dfrac{a_2}{2!}t^{2} + \dfrac{a_3}{3!}t^{3} +\ldots 
$$

Если все $ a_i $ равны 1, то ряд сходится к $ e $

Если перемножить два ряда:
$A(t) = a_0 + \dfrac{a_1}{1!}t + \dfrac{a_2}{2!}t^{2} + \dfrac{a_3}{3!}t^{3} +\ldots $
$B(t) = b_0 + \dfrac{b_1}{1!}t + \dfrac{b_2}{2!}t^{2} + \dfrac{b_3}{3!}t^{3} +\ldots $

То коэффициенты произведения
$A(t) B(t) = c_0 + \dfrac{c_1}{1!}t + \dfrac{c_2}{2!}t^{2} + \dfrac{c_3}{3!}t^{3} +\ldots $

устроены следующим образом:

$\dfrac{c_n}{n!} = \dfrac{a_{0}}{0!}\dfrac{b_{n}}{n!}
+ \dfrac{a_{1}}{1!} \dfrac{b_{n-1}}{(n-1)!} 
+ \dfrac{a_{2}}{2!} \dfrac{b_{n-2}}{(n-2)!}
+ \ldots 
+ \dfrac{a_{n}}{n!} \dfrac{b_{0}}{0!} $

Разделив обе части выражения на $ n! $, получим коэффициенты - сочетания $C_{n}^{k}$
\begin{equation}
\label{exponent_gen_func}
{c_n} = C_{n}^{0} a_{0} b^{n} 
+ C_{n}^{1} a_{1} b^{n-1} 
+ C_{n}^{2} a_{2} b^{n-2} 
+ \ldots
+ C_{n}^{n} a_{n} b^{0} 
\end{equation}


\subsection{Производящая функция чисел Белла}

Внесем полученную производящую функцию в формулу (\ref{exponent_gen_func}):
$$
b_{n+1} =  \sum\limits_{k = 0}^{n} C_{n}^{k} b_{k} 
$$

Отличается она от (\ref{exponent_gen_func}) 
отсутствием коэффициента $ a $, т.е. от равен $ 1 $,
функция $ A(t) = e^{t} $, а произведение $ A(t)B(t) $ равно:
$$
A(t)B(t) = e^{t} B(t) = e^{t} \sum\limits_{n=0}^{\infty} \dfrac{b_{n+1}}{n!} t^{n}
$$

Т.е. в полученной формуле показатели степени и коэффициенты при $ b $ "опрережают" 
на 1.
Чтобы, понизить степень, возьмем производную ряда $ B(t) $

$e^{t} B(t) = B'(t)$

Получим дифференциальное уравнение 1 степени.

$e^{t} = \dfrac{B'(t)}{B(t)} = (\ln{ B(t) })' $

$\ln B(t) = e^{t} + C $

$ B(t) = e^{e^{t} + C} $

Причем константа $ C = 0 $, т.к. при t = 0 $e^{t} + C $ должен быть равнен $ 0 $.

$$
B(t) = e^{e^{t} - 1}
$$


\subsection{Формула Добинского}

$$
b_n = \dfrac{1}{e}\sum_{k=0}^\infty \dfrac{k^n}{k!}
$$

Без доказательства.

\end{document}
